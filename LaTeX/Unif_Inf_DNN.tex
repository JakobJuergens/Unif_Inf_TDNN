\documentclass[letterpaper,10pt]{article}
\usepackage[letterpaper, hmargin=1.8cm, vmargin = 2.5cm]{geometry} %Sets the page geometry
\usepackage{url}
\usepackage{hyperref}
\usepackage{multicol}
\usepackage[english]{babel}
\usepackage{setspace}
\setlength{\parskip}{1em} % Set space when paragraphs are used
\setlength{\parindent}{0in}

\usepackage{xcolor}
\usepackage{multicol}
\usepackage{float}
\usepackage{graphicx} % Package for \includegraphics
\usepackage{wrapfig} % Figure wrapping
\usepackage[T1]{fontenc} % Output font encoding for international characters
\usepackage{amssymb}
\usepackage{amsmath}
\numberwithin{equation}{section}
%%%%% Boxes Setup %%%%%
\usepackage[many]{tcolorbox}    	% for COLORED BOXES (tikz and xcolor included)
\definecolor{main}{HTML}{dedcdc}    % setting main color to be used
\definecolor{sub}{HTML}{dedcdc}     % setting sub color to be used
\tcbset{
    sharp corners,
    colback = white,
    before skip = 0.3cm,    % add extra space before the box
    after skip = 0.3cm      % add extra space after the box
}                           % setting global options for tcolorbox
\newtcolorbox{boxD}{
    colback = sub, 
    colframe = main, 
    boxrule = 0pt, 
    toprule = 0pt, % top rule weight
    bottomrule = 0pt % bottom rule weight
}

\usepackage{amsthm}
\usepackage{mathtools}
\usepackage{thmtools}
\usepackage{bbm}
\usepackage[autostyle]{csquotes}
\usepackage{doi}
\usepackage[
    backend=biber,
    style=authoryear-comp,
    natbib=true,
    sortlocale=en_US,
    url=false, 
    doi=true,
    eprint=false
]{biblatex}

\addbibresource{bibliography.bib}

% Other
\newtheorem{thm}{Theorem}
\numberwithin{thm}{section}
\newtheorem{dfn}{Definition}
\newtheorem{lem}{Lemma}
\numberwithin{lem}{section}
\newtheorem{rmk}{Remark}
\newtheorem{exmp}{Example}
\newtheorem{cor}{Corollary}
\numberwithin{cor}{section}
% \newtheorem{prop}{Proposition}
\newtheorem{asm}{Assumption}

\renewcommand\qedsymbol{$\blacksquare$}

\newcommand{\sumi}{\sum_{i=1}^n}
\newcommand{\calE}{\mathcal{E}}
\makeatother
\renewcommand{\hat}{\widehat}
\newcommand{\E}{\mathbb{E}}
\newcommand{\1}{\mathbbm{1}}
\newcommand{\Var}{\text{Var}}
\newcommand{\Cov}{\text{Cov}}
\newcommand{\rk}{\text{rk}}
\renewcommand{\P}{\mathbbm{P}}

\makeatletter
\def\blfootnote{\gdef\@thefnmark{}\@footnotetext}
\makeatother

\DeclarePairedDelimiter\floor{\lfloor}{\rfloor} %Floor function
\DeclarePairedDelimiter\ceil{\lceil}{\rceil} %Ceil function
\DeclareMathOperator*{\argmax}{arg\,max} % argmax
\DeclareMathOperator*{\argmin}{arg\,min} % argmin
\newcommand{\indep}{\perp\!\!\!\!\perp} 


\begin{document}
\singlespacing
\title{Inference for Conditional Average Treatment Effects using \\ Distributional Nearest Neighbors}
\date{Last edited: \today}
\author{Jakob R. Juergens \\ University of Wisconsin - Madison}
\maketitle
\hrule
\onehalfspacing
\begin{abstract}
    This paper presents a computationally simple method of estimating heterogeneous treatment effects based on the Two-Scale Distributional Nearest Neighbor (TDNN) estimator of \citet{demirkaya_optimal_2024}.
    As part of this analysis, I improve on conditions required for consistent variance estimation presented in the original paper and provide results for asymptotically valid pointwise inference in a nonparametric regression setup and extend the analysis to the estimation of conditional average treatment effects.
    Building on the framework of \citet{ritzwoller_simultaneous_2024}, I develop uniformly valid confidence bands for the TDNN estimator.
    I then show how to apply these to perform uniformly valid inference in both the nonparametric regression setup and the heterogeneous treatment effect setup.
    A main contribution is the development of a computationally simple method that leverages the theoretical results of the aforementioned papers.
    % \blfootnote{The author thanks Harold D. Chiang, Jack Porter {\color{red} ADD OTHERS} for invaluable feedback.}
\end{abstract}
\vspace{0.3cm}
\hrule
\singlespacing

\vspace{-0.3cm}
\begin{center}
    {\small Supplementary Material and R Package available at: \url{https://github.com/JakobJuergens/Unif_Inf_TDNN}}
\end{center}
\vspace{0.3cm}
\hrule
\singlespacing
% {\small \tableofcontents}
\thispagestyle{empty}

\pagenumbering{arabic}
\onehalfspacing

\newpage
\section{Introduction}
\hrule
	{\color{red} LOREM IPSUM}

After a short notation section, the remainder of this paper is organized as follows.
{\color{red} LOREM IPSUM}

\subsection{Notation}
\hrule
Let $[n] = \{1, \dotsc, n\}$.
Given a finite index set $\mathcal{I} \subset \mathbb{N}$, I introduce the following notational conventions.
\begin{equation}
	L_{s}(\mathcal{I}) = \left\{\left(l_1, \dotsc, l_s\right) \in \mathcal{I}^{s} \, \middle| \, \forall i \neq j: \; l_i \neq l_j\right\}
	\quad \text{and} \quad
	L_{n,s} = L_s\left([n]\right)
\end{equation}
For a data set $D_{[n]} = \left(\mathbf{Z}_1, \dotsc, \mathbf{Z}_{n}\right)$ and a vector $\ell \in L_{n,s}$, denote by $D_{[n], -\ell}$ the data set where the observations corresponding to indices in $\ell$ have been removed.
To simplify the notation in the case that a single observation (say the $i'th$ observation) is removed, I use the notation $D_{n, -i}$.
Similarly, given such a data set $D_{[n]}$ and index vector $\ell$, denote by $D_{\ell}$ the data set only consisting of the observations in $D_{[n]}$ corresponding to indices in $\ell$.
In an abuse of notation, when considering two index vectors $\ell$ and $\iota$ that do not share any entries, I denote by $\ell \cup \iota$ the concatenation of the two vectors, e.g. if $\ell = (8,2,5)$ and $\iota = (1,6)$, then $\ell \cup \iota = (8,2,5,1,6)$.

In the following, $\rightsquigarrow$ denotes convergence in distribution, while $\rightarrow_{p}$ denotes convergence in probability and $\rightarrow_{a.s.}$ denotes almost sure convergence.
The norm $\| \cdot \|_{\psi_1}$ denotes the $\psi_1$-Orlicz norm.
Recall that random variables are sub-exponential if and only if they have a finite $\psi_1$-Orlicz norm.

{\color{red} TO-DO:}
\begin{itemize}
	\item $\lesssim$ needs an explanation
\end{itemize}

\newpage
\section{Setup}\label{sec:setup}
\hrule
Throughout this paper, we will consider two distinct setups.
The first is a pure nonparametric regression setup closely mirroring the structure of \citet{demirkaya_optimal_2024}.
This setup will be very useful to illustrate the inner workings of the estimator of interest and serve as a leading example for the theoretical results.
\begin{boxD}
	\begin{asm}[Nonparametric Regression DGP]\label{asm:npr_dgp}\mbox{}\\*
		The observed data consists of an i.i.d. sample taking the following form.
		\begin{equation}\label{DGP1}
			\mathbf{D}_n = \{Z_{i} = (X_{i}, Y_{i})\}_{i = 1}^{n}
			\quad \text{from the model} \quad
			Y = \mu(X) + \varepsilon,
		\end{equation}
		where $Y \in \mathbb{R}$ is the response, $X \in \mathcal{X} \subset \mathbb{R}^k$ is a feature vector of fixed dimension $k$ distributed according to a density function $f$ with associated probability measure $\varphi$ on $\mathcal{X}$, and $\mu(x)$ is the unknown mean regression function.
		$\varepsilon$ is the unobservable model error on which we impose the following conditions.
		\begin{equation}
			\E\left[\varepsilon \, \middle| \, X\right] = 0, \quad
			\Var\left(\varepsilon \, \middle| \, X = x\right) = \sigma_{\varepsilon}^2\left(x\right)
		\end{equation}
		Let the distribution induced by this model be denoted by $P$ and thus $Z_{i} = \left(X_{i}, Y_{i}\right) \overset{\text{iid}}{\sim} P$.
	\end{asm}
\end{boxD}
In contrast to this rather statistical setup, I will consider a setting with more immediate econometric relevance: estimation of and inference on heterogeneous treatment effects in the potential outcomes framework.
This serves as a more immediately applicable version of the theoretical setup presented in \citet{ritzwoller_uniform_2024} and brings their results closes to practitioners in the field of economics.
\begin{boxD}
	\begin{asm}[Heterogeneous Treatment Effect DGP]\label{asm:hte_dgp}\mbox{}\\*
		The observed data consists of an i.i.d. sample taking the following form.
		\begin{equation}\label{DGP2}
			\begin{aligned}
				\mathbf{D}_n & = \{Z_{i} = (X_{i}, W_{i}, Y_{i})\}_{i = 1}^{n}
				\quad \text{from the model} \quad
				Y = \1(W = 0)\mu_{0}(X) + \1(W = 1)\mu_1(X) + \varepsilon,	\\
				W_{i} & \sim \operatorname{Bern}\left(\pi\left(X_{i}\right)\right)
			\end{aligned}
		\end{equation}
		where $Y \in \mathbb{R}$ is the response and $W \in \{0,1\}$ is an observed treatment indicator.
		$X \in \mathcal{X} \subset \mathbb{R}^k$ is a vector of covariates of fixed dimension $k$ distributed according to a density function $f$ with associated probability measure $\varphi$ on $\mathcal{X}$ and $\varepsilon$ is the unobservable model error on which we impose the following conditions.
		\begin{equation}
			\varepsilon \indep W \, | \, X, \quad
			\E\left[\varepsilon \, | \, X\right] = 0, \quad
			\Var\left(\varepsilon \, | \, X = x\right) = \sigma_{\varepsilon}^2\left(x\right)
		\end{equation}
		Furthermore, $\mu_0:\mathcal{X} \rightarrow \mathbb{R}$ and $\mu_1:\mathcal{X} \rightarrow \mathbb{R}$ are the two unknown potential outcome functions and $\pi:\mathcal{X} \rightarrow [0,1]$ is a function describing the probability of treatment uptake, effectively corresponding to the propensity score.
		Let the distribution induced by this model be denoted by $Q$ and thus $Z_{i} = \left(X_{i}, W_{i}, Y_{i}\right) \overset{\text{iid}}{\sim} Q$.
	\end{asm}
\end{boxD}
In this second setting, I will use the notation $\mathbf{D}^{(0)}$ and $\mathbf{D}^{(1)}$ to refer to the data subsets containing only observations with $W = 0$ and $W = 1$, respectively.
Clearly, this model can be interpreted in the context of the potential outcomes framework in the usual manner.
\begin{boxD}
	\begin{rmk}[Potential Applications]\mbox{}\\*
		From an microeconometric perspective, these two setups cover a wide array of applications.
		While nonparametric regression is itself often advantageous to answer economic questions, the real strengths show when considering the second setup.
		{\color{red} LOREM IPSUM}
	\end{rmk}
\end{boxD}
Throughout this paper, I will additionally rely on a number of assumptions that are more technical in nature.
\begin{boxD}
	\begin{asm}[Technical Assumptions]\label{asm:technical}\mbox{}\\*
		In both settings (Assumption \ref{asm:npr_dgp} and Assumption \ref{asm:hte_dgp}) the following conditions hold:
		\begin{itemize}
			\item The feature space $\mathcal{X} = \operatorname{supp}(X)$ is a bounded, compact subset of $\mathbb{R}^k$
			\item The density $f(\cdot)$ is bounded away from 0 and $\infty$
			\item $f(\cdot)$ are $\mu(\cdot)$ are four times continuously differentiable with bounded second, third, and fourth-order partial derivatives in a neighborhood of $x$
		\end{itemize}
		In the Heterogeneous Treatment Effect setting (Assumption \ref{asm:hte_dgp}), the following additional condition holds:
		\begin{itemize}
			\item $\mu_0(\cdot)$ and $\mu_1(\cdot)$ are four times continuously differentiable with bounded second, third, and fourth-order partial derivatives in a neighborhood of $x$
		\end{itemize}
	\end{asm}
\end{boxD}
There is potential to relax these assumptions at the cost of requiring both less interpretable conditions and more technically sophisticated proofs.
Additionally, we require a rather standard assumption in localized regression approaches, namely that the variance changes continuously.
\begin{boxD}
	\begin{asm}[Error Distribution Assumptions]\label{asm:errors}\mbox{}\\*
		The error terms $\varepsilon$ defined in Setup \ref{asm:npr_dgp} and Setup \ref{asm:hte_dgp}, respectively, have continuously varying variance.
		In other terms,	$\sigma^2_{\varepsilon}: \mathcal{X} \rightarrow \mathbb{R}_{>0}$ is a continuous function.
	\end{asm}
\end{boxD}
As $\mathcal{X}$ is a bounded and compact set, this implies that there exists a $\overline{\sigma}_{\varepsilon}^2 > 0$ such that for any $x \in \mathcal{X}$ we have $\sigma^{2}_{\varepsilon}\left(x\right) \leq \overline{\sigma}_{\varepsilon}^2$.
Additionally, due to the assumptions on the regression functions, this ensures the existence of seconds moments of $Y$ in both scenarios.
Furthermore, to assure that there is a sufficient number of treated and untreated observations local to each point of interest asymptotically, we require the following condition on the treatment assignment and uptake mechanism.
\begin{boxD}
	\begin{asm}[Non-Trivial Treatment Overlap]\label{asm:treatment_overlap}\mbox{}\\*
		In the Heterogeneous Treatment Effect Setup (Assumption \ref{asm:hte_dgp}), we assume that there exist a constant $\mathfrak{p} \in (0, 1/2)$ such that
		\begin{equation}
			\forall x \in \mathcal{X}: \quad 
			0 < \mathfrak{p} \leq \pi\left(x\right) \leq 1 - \mathfrak{p} < 1.
		\end{equation}
	\end{asm}
\end{boxD}
This assumption seems rather strong when considering a full universe of potential treatment recipients.
In reality we can constrain this overlap assumption to neighborhoods of points of interests $x$.
As long as there is sufficient overlap in those neighborhoods the ideas of our identification strategy continue to hold locally.
\begin{boxD}
	\begin{asm}[Stable Unit Treatment Value Assumption (SUTVA)]\label{asm:sutva}\mbox{}\\*
		For any $n$, let $\mathfrak{W}_{n}: \mathcal{X}^{n} \rightarrow \{0,1\}^{n}$ and $\mathfrak{W}_{n}^{\prime}: \mathcal{X}^{n} \rightarrow \{0,1\}^{n}$ be two functions characterizing treatment assignment among a group of $n$ potential observations.
		{\color{red} LOREM IPSUM}
	\end{asm}
\end{boxD}


\newpage
\section{Two-Scale Distributional Nearest Neighbor Estimator}\label{sec:TDNN}
\hrule
While less economically enticing, we will introduce the TDNN estimator using the simple nonparametric regression setup first.
We will do this by first considering the simpler (one-scale) distributional nearest neighbor estimator, which naturally extends to its two-scale variant as shown in \citet{demirkaya_optimal_2024}.
Then, having established the method, we will commence by adapting it to tackle the problem of estimating heterogeneous treatment effects.
As we will embed both estimation problems in the context of subsampled conditional moment regression to then build simultaneous inference procedures based on \citet{ritzwoller_simultaneous_2024}, the approach might at first seem unnatural.
However, due to the constructions that follow in Section~\ref{sec:unif_inf}, this approach will be well worth the slightly cumbersome initial presentation.

\subsection{DNN and TDNN in Nonparametric Regression}
\hrule
We can rephrase the nonparametric regression problem in terms of estimating specific conditional moments.
In the case at hand, this means that our problem can be phrased in the following way.
\begin{equation}\label{CondMomEq}
	M(x; \mu)
	= \E\left[m(Z_{i}; \mu) \, | \, X_{i} = x\right]
	= 0
	\quad \text{where} \quad
	m(Z_{i}; \mu) = Y_{i} - \mu(X_{i}).
\end{equation}
Due to the absence of nuisance parameters, conditions such as local Neyman-orthogonality vacuously hold.
We point this out to highlight a contrast that we will encounter when studying the treatment effect setting.
In the simpler non-parametric regression setting, we can approached the problem by solving the corresponding empirical conditional moment equation.
\begin{equation}\label{EmpCondMomEq}
	M_n(x; \mu, \mathbf{D}_n)
	= \sum_{i = 1}^{n}K(x, X_{i})m(Z_{i}; \mu)
	= 0
\end{equation}
In this equation, $K:\mathbb{R}^d \times \mathbb{R}^d \rightarrow \mathbb{R}$ is a data-dependent Kernel function measuring the ``distance'' between the point of interest and an observation.
Notationally, this makes the local and data-dependent approach of this procedure explicit.
One estimator that fulfills the purpose of estimating $\mu$ nonparametrically is the Distributional Nearest Neighbor (DNN) estimator.
With a name coined by \citet{demirkaya_optimal_2024}, the DNN estimator is based on important work by \citet{steele_exact_2009} and \citet{biau_rate_2010}.
Given a sample as described in Assumption~\ref{asm:npr_dgp} and a fixed feature vector $x$, we first order the sample based on the distance to the point of interest.
\begin{equation}\label{eq:ordering}
	||X_{(1)} - x||_2
	\leq ||X_{(2)} - x||_2
	\leq \dotsc
	\leq ||X_{(n)} - x||_2
\end{equation}
Here draws are broken according to the natural indices of the observations in a deterministic way to simplify the derivations going forward.
While the distance induced by the euclidean norm is a useful tool for developing an intuition for the method, the idea is not inherently connected to it.
In fact, any distance induced by a norm that captures the geometry of the feature space in a suitable way can be used to construct an analogous weighting scheme.
The generated ordering implies an associated ordering on the response variables and we denote by $Y_{(i)}$ the response corresponding to $X_{(i)}$.
Let $\rk(x; X_{i}, D)$ denote the \textit{rank} that is assigned to observation $i$ in a sample $D$ relative to a point of interest $x$, setting $\rk(x; X_{i}, D) = \infty$ if $Z_{i} \not\in D$.
Similarly, let $Y_{(1)}(x; D)$ indicate the response value of the closest neighbor in set $D$.
This enables us to define a data-driven kernel function $\kappa$ following the notation of \citet{ritzwoller_simultaneous_2024}.
\begin{equation}
	\kappa(x; Z_{i}, D, \xi)
	= \1\left(\rk(x; X_{i}, D) = 1\right)
\end{equation}
Here, $\xi$ is an additional source of randomness in the construction of the base learner that comes into play when analyzing, for example, random forests as proposed by \citet{breiman_random_2001} using the CART-algorithm described in \citet{breiman_classification_2017}.
As the DNN estimator does not incorporate such additional randomness, the term is omitted in further considerations.
In future research, additional randomness such as, for example, column subsampling could be considered, in turn making the addition of $\xi$ necessary again.
Using $\kappa$, it is straightforward to find an expression for the distance function $K$ in Equation~\ref{EmpCondMomEq} corresponding to the DNN estimator.
\begin{equation}\label{eq:data_distance}
	K(x, X_{i})
	= \binom{n}{s}^{-1} \sum_{\ell \in L_{n,s}} \1(i \in \ell)\frac{\kappa(x; Z_{i}, D_{\ell})}{s!}
	= \binom{n}{s}^{-1} \sum_{\ell \in L_{n,s}} \frac{\1\left(\rk(x; Z_{i}, D_{\ell}) = 1\right)}{s!}
\end{equation}
Inserting into Equation~\ref{EmpCondMomEq}, this gives us the following empirical conditional moment equation.
\begin{equation}
	\begin{aligned}
		M_n(x; \mu, \mathbf{D}_n)
		= \sum_{i = 1}^{n}\left(\binom{n}{s}^{-1} \sum_{\ell \in L_{n,s}} \frac{\1\left(\rk(x; Z_{i}, D_{\ell}) = 1\right)}{s!}\right)\left(Y_{i} - \mu(X_{i})\right)
		= 0
	\end{aligned}
\end{equation}
Solving this empirical conditional moment equation then yields the DNN estimator $\tilde{\mu}_{s}(x)$ with subsampling scale $s$.
Defining the kernel function, $h_{s}(x; D_{\ell}) := (s!)^{-1} Y_{(1)}(x; D_{\ell})$, it is given by the following U-statistic.
\begin{equation}\label{eq:U_stat}
	\tilde{\mu}_{s}(x; \mathbf{D}_n)
	= \binom{n}{s}^{-1} \sum_{\ell \in L_{n,s}} h_{s}(x; D_{\ell})
\end{equation}
\citet{steele_exact_2009} shows that the DNN estimator has a simple closed form representation based on the original ordered sample.
\begin{equation}\label{eq:DNN_closed_form}
	\tilde{\mu}_{s}(x; \mathbf{D}_n)
	= \binom{n}{s}^{-1} \sum_{i = 1}^{n - s + 1}\binom{n - i}{s - 1}Y_{(i)}
\end{equation}
This representation will allow me to derive computationally simple representations for the practical use of the procedures presented in this paper.
This is in contrast to most U-statistic based methods that inherently rely on evaluating the kernel on individual subsets, incurring a potentially prohibitive computational cost.
Furthermore, this representation motivates an asymptotic approximation of the weights assigned to each observation that starkly reduces the potentially computationally intensive computation of large binomial coefficients.
For this purpose let $\alpha_{s} = s/n$ leading to the following approximation of the DNN estimator using asymptotic weights.
\begin{equation}\label{eq:DNN_approx_closed_form}
	\tilde{\mu}_{s}(x; \mathbf{D}_n)
	\approx  \sum_{i = 1}^{n - s + 1} \alpha_{s} \left(1 - \alpha_{s}\right)^{i - 1} Y_{(i)}
\end{equation}
It is worthwhile to point out that the role of $s$ in the implicit bias-variance tradeoff of the DNN estimator runs counter to the role of $k$ in the usual k-NN regression.
Where a larger $k$ is usually associated with a lower variance at the cost of a higher bias, a larger $s$ does the opposite.
This is due to the fact that a higher $s$ reduces the number of observations that can occur as the closest observation in any given $s$-subset.
As a special example that illustrates the relationship, consider the DNN estimator choosing $s = n$ recovering the simple 1-NN regression estimator.
As part of their paper, \citet{demirkaya_optimal_2024} develop an explicit expression for the first-order bias term of the DNN estimator and the following distributional approximation result.
\begin{boxD}
	\begin{thm}[\citet{demirkaya_optimal_2024} - Theorem 2]\label{thm:dem2}\mbox{}\\*
		Assume that we observe data as described in Assumption~\ref{asm:npr_dgp} and that Assumption~\ref{asm:technical} is valid.
		Then, for any fixed $x \in \mathcal{X}$, we have that for some positive sequence $\omega_n$ of order $\sqrt{s/n}$
		\begin{equation}
			\frac{\tilde{\mu}_{s}(x; \mathbf{D}_n) - \mu(x) - B(s) - R(s)}{\omega_n}
			\rightsquigarrow \mathcal{N}\left(0,1\right)
		\end{equation}
		as $n,s \rightarrow \infty$ with $s = o(n)$.
		Here, $B(s)$ and $R(s)$ are defined as the following bias terms.
		\begin{align}
			B(s)
			= \Gamma(2 / k+1) \frac{f(x) \operatorname{tr}\left(\mu^{\prime \prime}(x)\right)+2 \mu^{\prime}(x)^T f^{\prime}(x)}{2 d V_d^{2 / k} f(x)^{1+2 / k}} s^{-2 / k}
			\quad \text{and} \quad
			R(s) =
			\begin{cases}
				O\left(s^{-3}\right),     & k = 1      \\
				O\left(s^{-4 / k}\right), & k \geq 2
			\end{cases}
		\end{align}
		where\dots
		\begin{multicols}{2}
			\begin{itemize}
				\item $V_d=\frac{k^{k / 2}}{\Gamma(1+k / 2)}$
				\item $\Gamma(\cdot)$ is the gamma function
				\item $\operatorname{tr}(\cdot)$ stands for the trace of a matrix
				\item $f^{\prime}(\cdot)$ and $\mu^{\prime}(\cdot)$ denote the first-order gradients of $f(\cdot)$ and $\mu(\cdot)$, respectively
				\item $f^{\prime \prime}(\cdot)$ and $\mu^{\prime \prime}(\cdot)$ represent the $d \times d$ Hessian matrices of $f(\cdot)$ and $\mu(\cdot)$, respectively
			\end{itemize}
		\end{multicols}
	\end{thm}
\end{boxD}

Starting from this set-up, \citet{demirkaya_optimal_2024} develop a novel
bias correction method for the DNN estimator that leads to appealing
finite-sample properties of the resulting Two-Scale Distributional Nearest
Neighbor (TDNN) estimator. Their method is based on the explicit formula for
the first-order bias term of the DNN estimator, which in turn allows them to
eliminate it through a clever combination of two DNN estimators. Choosing two
subsampling scales $1 \leq s_1 < s_2 \leq n$ and two corresponding weights
\begin{equation}
	w_{1}^{*}(s_1, s_2) = \frac{1}{1-(s_1/s_2)^{-2/k}}
	\quad\text{and}\quad
	w_2^{*}(s_1, s_2) = 1 - w_{1}^{*}(s_1, s_2)
\end{equation}
they define the corresponding TDNN estimator as follows.
\begin{equation}
	\hat{\mu}_{s_1, s_2}\left(x; \mathbf{D}_n\right)
	= w_{1}^{*}(s_1, s_2)\tilde{\mu}_{s_1}\left(x; \mathbf{D}_n\right) + w_2^{*}(s_1, s_2)\tilde{\mu}_{s_2}\left(x; \mathbf{D}_n\right)
\end{equation}
This leads to the elimination of the first-order bias term shown in Theorem~\ref{thm:dem2} leading to desirable finite-sample properties.
Furthermore, the authors show that this construction improves the quality of the normal approximation.

\begin{boxD}
	\begin{asm}[Bounded Ratio of Kernel-Orders]\label{asm:kernel_order_ratio}\mbox{}\\*
		There is a constant $\mathfrak{c} \in (0,1/2)$ such that the ratio of kernel orders is bounded in the following way.
		\begin{equation}
			\forall n: \quad 0 < \mathfrak{c} \leq s_1 / s_2 \leq 1 - \mathfrak{c} < 1.
		\end{equation}
	\end{asm}
\end{boxD}
We make this assumption to avoid edge cases, where asymptotically the TDNN estimator converges to one of the DNN estimators that make it up.
As this edge case is irrelevant in practice, as it would be simpler to employ the corresponding DNN estimator in the first place, this is not a practically substantial restriction.
\begin{boxD}
	\begin{thm}[\citet{demirkaya_optimal_2024} - Theorem 3]\label{thm:dem3}\mbox{}\\*
		Assume that we observe data as described in Assumption~\ref{asm:npr_dgp} and that Assumption~\ref{asm:technical} holds.
		Furthermore, let $s_1, s_2 \rightarrow \infty$ with $s_1 = o(n)$ and $s_2 = o(n)$ be such that Assumption~\ref{asm:kernel_order_ratio} holds for some $\mathfrak{c} \in (0, 1/2)$.
		Then, for any fixed $x \in \operatorname{supp}(X) \subset \mathbb{R}^d$, it holds that for some positive sequence $\sigma_n$ of order $(s_2/n)^{1/2}$,
		\begin{equation}
			\sigma_n^{-1} \left(\hat{\mu}_{s_1, s_2}\left(x; \mathbf{D}_n\right) - \mu(x) - \Lambda\right) \rightsquigarrow \mathcal{N}(0,1)
		\end{equation}
		as $n \rightarrow \infty$, where
		\begin{equation*}
			\Lambda = \begin{cases}
				O\left(s_1^{-4/d} + s_2^{-4/d}\right) & \text{for } d \geq 2 \\
				O\left(s_1^{-3} + s_2^{-3}\right)     & \text{for } d = 1    \\
			\end{cases} .
		\end{equation*}
	\end{thm}
\end{boxD}

\subsection{DNN and TDNN in Heterogeneous Treatment Effect Estimation}
\hrule
Motivated by the nonparametric regression setup, we set out to apply the underlying idea in the context of heterogeneous treatment effects.
Similarly to before, we start by specifying a moment corresponding to our object of interest, taking into account the additional factors that come into play.
Due to the presence of a high-dimensional nuisance parameter in the form of the function $q$, it is natural to apply the concepts of DDML (DML).
This approach closely follows the leading example of \citet{ritzwoller_simultaneous_2024}.
The main goal at this stage is to construct a highly practical method based on their ideas that leverages the computational simplicity of the distributional nearest-neighbor framework.\\

While considering the problem of point-estimation of a conditional average
treatment effect given a feature vector $x$,
$\operatorname{CATE}(x) = \E\left[Y_{i}\left(W_{i} = 1\right) -
		Y_{i}\left(W_{i} = 0\right) \, \middle| \, X_{i} = x\right]$, we
will employ a Neyman-orthogonal score function to curtail the influence of the
nuisance parameters on our estimation.
\begin{equation}
	\begin{aligned}
		M(x; \operatorname{CATE}, \mu, p)
		 & = \E\left[m\left(Z_{i}; \operatorname{CATE}, \mu, \pi\right) \, \middle| \, X_{i} = x\right]
		= 0
		\quad \text{where} \quad                                                                                                                                                                                       \\
		m\left(Z_{i}; \operatorname{CATE}, \mu, \pi\right)
		 & = \mu_1\left(X_{i}\right) - \mu_0\left(X_{i}\right) + \beta\left(W_{i}, X_{i}\right)\left(Y_{i} - \mu_{W_{i}}\left(X_{i}\right)\right) - \operatorname{CATE}\left(X_{i}\right)
	\end{aligned}
\end{equation}
Here, we make use of the following notation, that is common in the potential
outcomes framework, and the well-known Horvitz-Thompson weight.
\begin{equation}
	\text{for }  w = 1,2: \quad \mu_w\left(x\right) = \E\left[Y_{i} \, \middle| \, W_{i} = w, \; X_{i} = x\right]
	\quad \text{and} \quad
	\beta\left(w, x\right) = \frac{w}{\pi\left(x\right)} - \frac{1 - w}{1 - \pi\left(x\right)}
\end{equation}
As a shorthand notation, we will furthermore use $m\left(Z_{i}; \mu, \pi\right) = m\left(Z_{i}; \operatorname{CATE}, \mu, \pi\right) + \operatorname{CATE}\left(X_{i}\right)$.
This notation will mainly be used to shorten the presentation of proofs in the appendix.
Proceeding in an analogous fashion to the nonparametric regression setup leads us to the following empirical moment equation, where $\hat{\mu}$ and $\hat{\pi}$ are first-stage estimators and $K$ is the data-driven kernel function defined in Equation~\ref{eq:data_distance}.
\begin{equation}
	\begin{aligned}
		M_{n}\left(x; \hat{\mu}, \hat{\pi}\right)
		= \sum_{i = 1}^{n} K(x, X_{i}) m\left(Z_{i}; \hat{\mu}, \hat{\pi}\right)
		= 0
	\end{aligned}
\end{equation}
However, due to the presence of infinite-dimensional nuisance parameters, it becomes attractive to proceed by using this weighted empirical moment equation embedded into the DML2 estimator of \citet{chernozhukov_doubledebiased_2018}.
Applying these ideas to the context of estimating the CATE has been previously explored, for example by \citet{semenova_debiased_2021}
For the sake of simplicity, we will assume that $m = n/K$, i.e.\ the desired number of observations in each fold, is an integer going forward.
\begin{boxD}
	\begin{dfn}{(T)DNN-DML2 CATE-Estimator}\label{def:CATE_DNN_DML}\mbox{}\\*
		To estimate the Conditional Average Treatment Effect at a point of interest $x \in \mathcal{X}$, we proceed as follows.
		\begin{enumerate}
			\item Take a $K$-fold random partition $\mathcal{I} = \left(I_k\right)_{k = 1}^{K}$
			      of the observation indices $[n]$ such that the size of each fold $I_k$ is $m =
				      n/K$. For each $k \in [K]$, define $I_{k}^{C} = [n] \backslash I_k$.
			      Furthermore, for the observation being assigned rank $i \in [n]$, denote by
			      $k(i)$ the fold that the observation appears in.
			\item For each $k \in [K]$, use the DNN estimator on the data set
			      $\mathbf{D}_{I_k^C}$\dots
			      \begin{enumerate}
				      \item to estimate the nuisance parameters $\mu_0$ and $\mu_1$:
				            \begin{equation}
					            \hat{\mu}_{k,s}^{w}\left(x\right) = \hat{\mu}_{w,s}\left(x; \mathbf{D}_{I_k^{C}}^{(w)}\right) \quad \text{for } w=0,1
				            \end{equation}
				      \item if $\pi$ is unknown, i.e.\ we are not in a randomized experiment setting,
				            additionally estimate $\pi$
				            \begin{equation}
					            \hat{\pi}_{k,s}\left(x\right) = \hat{\mu}_{s}\left(x; \mathbf{D}_{I_k^{C}}\right) \quad \text{where the predicted variable is $W$}
				            \end{equation}
			      \end{enumerate}
			\item Construct the estimator $\widehat{\operatorname{CATE}}\left(x\right)$
			      as the solution to the following equation.
			      \begin{equation}
				      \begin{aligned}
					      0 \quad = \quad & \sum_{k = 1}^{K} \sum_{i \in I_k} K(x, X_{i}) m\left(Z_{i}; \widehat{\operatorname{CATE}}\left(x\right), \hat{\mu}_{k,s}, \hat{\pi}_{k,s}\right)                                                                \\
					      %
					      = \quad         & \sum_{i = 1}^{n - s + 1} \left[\frac{\binom{n-i}{s-1}}{\binom{n}{s}} \sum_{k = 1}^{K} \1\left(i \in I_{k}\right)  m\left(Z_{(i)}; \widehat{\operatorname{CATE}}\left(x\right), \hat{\mu}_{k,s}, \hat{\pi}_{k,s}\right)\right] \\
					      %
					      = \quad         & \sum_{i = 1}^{n - s + 1} \left[\frac{\binom{n-i}{s-1}}{\binom{n}{s}} m\left(Z_{(i)}; \widehat{\operatorname{CATE}}\left(x\right), \hat{\mu}_{k(i),s}, \hat{\pi}_{k(i),s}\right)\right]
				      \end{aligned}
			      \end{equation}
		\end{enumerate}
	\end{dfn}
\end{boxD}
This description shows the case of the DNN estimator.
Observe, that the weights $K(x, X_{i})$ chosen in the second step are chosen according to the full sample - not according to the chosen folds.
The corresponding TDNN-based estimator is defined analogously, employing the TDNN estimator in the first-stage estimation procedure and using the corresponding weights of the TDNN-estimator in the second stage.
It should be pointed out that the use of the TDNN estimator in the estimation of the propensity score can have the potentially adverse property of generating estimates outside the unit interval.
This is due to the presence of negative weights for specific combinations of subsampling scales.
Thus, restricting the procedure to rely on the DNN estimator for the estimation of propensity scores in the first stage might be desirable.
Specifically, using a lower choice of subsampling scale for this estimation step can help avoid extreme values in the Neyman orthogonal score function due to estimated propensity scores close to zero or one.
This is due to the fact that a lower subsampling scale averages over a larger number of observations and can thus contribute to better smoothing properties for the propensity score.

Plugging in for the score function in the equation that defines the estimator, we can observe the following.
\begin{equation}
	\begin{aligned}
		\widehat{\operatorname{CATE}}\left(x\right) & = \sum_{i = 1}^{n - s + 1} \frac{\binom{n-i}{s-1}}{\binom{n}{s}}
		\left[\hat{\mu}_{k(i),s}^{1}\left(X_{(i)}\right) - \hat{\mu}_{k(i),s}^{0}\left(X_{(i)}\right) + \hat{\beta}_{k(i),s}\left(W_{(i)}, X_{(i)}\right)\left(Y_{(i)} - \hat{\mu}^{W_{(i)}}_{k(i),s}\left(X_{(i)}\right)\right)\right]
	\end{aligned}
\end{equation}
Thus, given first-stage estimates of the nuisance parameters, we have a closed form representation of the CATE-estimator for a given partition of $[n]$.
Furthermore, given these first-stage estimates, the evaluation of the CATE-estimator at a different point of interest is merely a reweighting of the terms corresponding to different observations.
Considering the first stage estimates, we can recognize that the estimation of $\mu^{0}$ and $\mu^{1}$ is effectively a nonparametric regression problem as previously described where we used the reduced data sets $\mathbf{D}^{(0)}$ and $\mathbf{D}^{(1)}$, respectively.
In contrast, the estimation of $\pi$ that is necessary in nearly all contexts but randomized experiments can be described further due to the binary outcome.
For that purpose, let $Z_{(i|k)}$ denote the $i$'th closest observation in fold $k$ akin to the construction shown in Equation~\ref{eq:ordering} relative to $x$ but with respect to the data in fold $k$.
\begin{equation}
	\begin{aligned}
		\hat{\pi}_{k,s}\left(x\right)
		= \sum_{i = 1}^{n - m - s + 1} \frac{\binom{n - m - i}{s - 1}}{\binom{n - m}{s}} W_{(i|k)}
	\end{aligned}
\end{equation}
What these equations show is that the main computational cost associated with these methods comes from having to construct multiple orderings of the sample of interest.
The essential strength of this approach: It is not necessary to solve any complex optimization problems to obtain the estimator.
Furthermore, due to the prevalence of constructing orderings of data with respect to the euclidean norm, this is a well-studied problem with efficient algorithms available of the shelf.

As an extension, we can consider a leave-one-out estimation analog for the functional nuisance parameters, where these are estimated at each observation based on all other available observations.
This approach eliminates the randomness inherent to the crossvalidation procedure while preserving the advantages obtained through the usage of DML ideas.
This leads to the following estimator.
\begin{boxD}
	\begin{dfn}{(T)DNN-LOO-DML CATE-Estimator}\label{def:CATE_DNN_LOO_DML}\mbox{}\\*
		To estimate the Conditional Average Treatment Effect at a point of interest $x \in \mathcal{X}$, we proceed as follows.
		\begin{enumerate}
			\item For each observation $i$, use the (T)DNN estimator on the data set $\mathbf{D}_{n,-i}$\dots
			      \begin{enumerate}
				      \item to estimate the nuisance parameters $\mu_0$ and $\mu_1$:
				            \begin{equation}
					            \tilde{\mu}_{s}^{w}\left(x\right) 
								= \hat{\mu}_{w,s}\left(x; \mathbf{D}_{n,-i}^{(w)}\right) \quad \text{for } w=0,1
				            \end{equation}
				      \item if $\pi$ is unknown, i.e.\ we are not in a randomized experiment setting,
				            additionally estimate $\pi$
				            \begin{equation}
					            \tilde{\pi}_{s}\left(x\right) 
								= \hat{\mu}_{s}\left(x; \mathbf{D}_{n,-i}\right) \quad \text{where the predicted variable is $W$}
				            \end{equation}
			      \end{enumerate}
			\item Construct the estimator $\widetilde{\operatorname{CATE}}\left(x\right)$ as
			      \begin{equation}
					\begin{aligned}
						\widetilde{\operatorname{CATE}}\left(x\right) & = \sum_{i = 1}^{n - s + 1} \frac{\binom{n-i}{s-1}}{\binom{n}{s}}
						\left[\tilde{\mu}_{s}^{1}\left(X_{(i)}\right) - \tilde{\mu}_{s}^{0}\left(X_{(i)}\right) + \tilde{\beta}_{s}\left(W_{(i)}, X_{(i)}\right)\left(Y_{(i)} - \tilde{\mu}^{W_{(i)}}_{s}\left(X_{(i)}\right)\right)\right]
					\end{aligned}
				\end{equation}
		\end{enumerate}
	\end{dfn}
\end{boxD}

\newpage
\section{Pointwise Inference for the TDNN Estimator}\label{sec:pw_inf}
\hrule
To perform inference in the regression setup, \citet{demirkaya_optimal_2024} introduce variance estimators based on the Jackknife and Bootstrap.
However, as they point out, their consistency results rely on a likely suboptimal rate condition for the subsampling scale.
While Theorem \ref{thm:dem3} allows $s_2$ to be of the order $o(n)$, the variance estimators rely on the considerably stronger condition that $s_2 = o(n^{1/3})$.
Establishing consistent variance estimation under weaker assumptions on the subsampling rates could broaden the scope of the TDNN estimator for inferential purposes considerably.
Furthermore, it can contribute to a better balance between variance and bias as the choice of the kernel orders is crucial when considering the finite sample properties of the estimator.
In this paper, I will focus specifically on variance estimators based on the Jackknife and show consistency results under $s = o(n)$.
This is motivated by the closed form representation of the estimators in question leading to computationally simple formulas for the exact Jackknife variance estimators.

\subsection{Jackknife Variance Estimators for Nonparametric Regression}\label{Var_Ests}
\hrule

Define the following variance we need to estimate to perform pointwise inference at a point of interest $\mathbf{x}$.
\begin{equation}\label{eq:TDNN_Var}
	\omega^{2}\left(\mathbf{x}; \mathbf{D}_n\right)
	= \Var_{D}\left(\hat{\mu}_{s_1, s_2}\left(\mathbf{x}; \mathbf{D}_n\right)\right)
\end{equation}
We denote by $\mathbf{D}_{n, -i}$ the data set $\mathbf{D}_n$ after removing the $i$'th observation.
Then, the proposed Jackknife variance estimator takes the following form.
\begin{equation}\label{eq:JK_Var_Est}
	\hat{\omega}_{JK}^2\left(\mathbf{x}; \mathbf{D}_n\right)
	= \frac{n-1}{n} \sum_{i = 1}^{n}\left(\hat{\mu}_{s_1, s_2}\left(\mathbf{x}; \mathbf{D}_{n, -i}\right) - \hat{\mu}_{s_1, s_2}\left(\mathbf{x}; \mathbf{D}_{n}\right)\right)^2
\end{equation}

\begin{boxD}
	\begin{thm}[Closed Form Expression for the Jackknife-Variance Estimator]\label{thm:JK_closed_form}\mbox{}\\*
		The Jackknife variance estimator for the DNN estimator has the following convenient closed-form representations.
		\begin{equation}
			{\color{red} LOREM IPSUM}
		\end{equation}
		Similarly, the Jackknife variance estimator for the TDNN estimator admits the following representation.
		\begin{equation}
			{\color{red} LOREM IPSUM}
		\end{equation}
	\end{thm}
\end{boxD}

\begin{equation}\label{eq:JKD_Var_Est}
	\hat{\omega}_{JKD}^2\left(\mathbf{x}; d, \mathbf{D}_n\right)
	= \frac{n-d}{d}\binom{n}{d}^{-1} \sum_{\ell \in L_{n,d}}
	\left(\hat{\mu}_{s_1, s_2}\left(\mathbf{x}; \mathbf{D}_{n, -\ell}\right)
	- \hat{\mu}_{s_1, s_2}\left(\mathbf{x}; \mathbf{D}_{n}\right)
	\right)^2
\end{equation}

In this section, we will loosen that restrictive condition to make use of the attractive performance of U-statistics with large subsampling rates in the context of inference.
The PIJK variance estimator applied to the TDNN estimator is as follows.
\begin{equation}\label{eq:PIJK_Var_Est}
	\hat{\omega}_{PI}^2\left(\mathbf{x}; \mathbf{D}_n\right)
	= \frac{s_2^2}{n^2}\sum_{i = 1}^{n}\left[\left(
		\binom{n-1}{s-1}^{-1} \sum_{\ell \in L_{s_2-1}([n]\backslash\{i\})} h_{s_1, s_2}\left(\mathbf{x}; D_{\ell \cup \{i\}}\right)\right)
		- \hat{\mu}_{s_1, s_2}\left(\mathbf{x}; \mathbf{D}_n\right)\right]^2
\end{equation}

{\color{red} LOREM IPSUM}

Analysing the kernel of the TDNN estimators, it can be shown that the conditions of Theorem 6 of \citet{peng_bias_2021} apply under the regime $s_2 = o(n)$.
Thus, we obtain the following result.
% \begin{boxD}
% 	\begin{lem}[\citet{peng_bias_2021} - Theorem 6]\label{thm:peng21_6}\mbox{}\\*
% 		Let $X_1, \ldots, X_n$ be i.i.d. from $\mathrm{P}$ and $\mathrm{U}_{n, k, \omega}$ be a generalized complete $U$ statistic with kernel $s\left(X_1, \ldots, X_k ; \omega\right)$.
	
% 		Let $\theta=\E[s], \zeta_{1, \omega}=\Var\left(\E\left[s \mid X_1\right]\right)$ and $\zeta_k=\Var(s)$.
% 		If $\frac{k}{n}\left(\frac{\zeta_k}{k \zeta 1, \omega}-1\right) \rightarrow 0$, then
% 		\begin{equation}
% 			\frac{\mathrm{ps-IJ}_{U}^\omega}{\Var\left(\mathrm{U}_{n, k, \omega}\right)}
% 			\longrightarrow_{p} 1 .
% 		\end{equation}
% 	\end{lem}
% \end{boxD}
\begin{boxD}
	\begin{thm}[Pseudo-Infinitesimal Jackknife Variance Estimator Consistency]\label{thm:PI_JK_Cons}\mbox{}\\*
		Let $0 < c_1 \leq s_1/s_2 \leq c_2 < 1$ and $s_2 = o(n)$, then
		\begin{equation}
			\frac{\hat{\omega}_{PI}^2\left(\mathbf{x}; \mathbf{D}_n\right)}{\omega^{2}\left(\mathbf{x}; \mathbf{D}_n\right)} \longrightarrow_{p} 1.
		\end{equation}
	\end{thm}
\end{boxD}


In an analogous fashion to Theorems 5 and 6 from \citet{demirkaya_optimal_2024}, we furthermore obtain the following consistency results for the presented variance estimators.
As they point out, proving these results goes beyond the techniques presented in \citet{arvesen_jackknifing_1969}, instead relying on results for infinite-order U-statistics.
Following the ideas from \citet{peng_bias_2021}, we then obtain the following results on the Jackknife and Bootstrap variance estimators respectively.
As part of the proof of these results, we obtain general results on the consistency of Jackknife and Bootstrap variance estimators for infinite-order U-statistics beyond the TDNN estimator.
\begin{boxD}
	\begin{thm}[Jackknife Variance Estimator Consistency]\label{thm:JK_Cons}\mbox{}\\*
		Let $0 < c_1 \leq s_1/s_2 \leq c_2 < 1$ and $s_2 = o(n)$, then
		\begin{equation}
			\frac{\hat{\omega}_{JK}^2\left(\mathbf{x}; \mathbf{D}_n\right)}{\omega^{2}\left(\mathbf{x}; \mathbf{D}_n\right)} \longrightarrow_{p} 1.
		\end{equation}
	\end{thm}
\end{boxD}

\begin{boxD}
	\begin{thm}[delete-d Jackknife Variance Estimator Consistency]\label{thm:JKD_Cons}\mbox{}\\*
		Let $0 < c_1 \leq s_1/s_2 \leq c_2 < 1$, $s_2 = o(n)$, and $d = o(n)$, then
		\begin{equation}
			\frac{\hat{\omega}_{JKD}^2\left(\mathbf{x}; d, \mathbf{D}_n\right)}{\omega^{2}\left(\mathbf{x}; \mathbf{D}_n\right)} \longrightarrow_{p} 1.
		\end{equation}
	\end{thm}
\end{boxD}

\subsection{Variance Estimation for the (T)DNN-DML2 CATE Estimator}\label{CATE_Var_Ests}
\hrule

{\color{red}
Ideas:
\begin{itemize}
	\item Ignoring the occurrence of left-out observation in nuisance parameter estimation and do basic Jackknife - does this lead to bias?
	\item Leave Fold-Out Bootstrap with slowly diverging number of folds ($k \rightarrow \infty, \;m = o(n)$) - Effectively a variant of delete-d bootstrap
	\item Leave out two folds in the estimator's first step. 
	Then use each previously left out fold for Jackknife construction to eliminate contamination from nuisance parameters
	\item Modify approach presented in \citet{ritzwoller_uniform_2024} Appendix F.4 - modified half-sample k-fold cross-split bootstrap root
\end{itemize}
}

A fitting variance estimator given the context of this paper in the literature can be obtained by modifying a construction presented in \citet{ritzwoller_uniform_2024}.
Specifically, the procedure is based on a variation of the approach presented in Appendix F.4 of the aforementioned paper and makes use of a carefully constructed bootstrap-root.
Thus, we need to introduce some additional notation.
\begin{boxD}
	\begin{dfn}[Crossfitting Half-Sample]\label{def:CF_HSample}\mbox{}\\*
		{\color{red} LOREM IPSUM}
	\end{dfn}
\end{boxD}
This bootstrap root will take the following structure.
\begin{equation}
	R^{*}_{n,s}\left(\mathbf{x}\right)
	= \overline{\operatorname{CATE}}_{H(\mathcal{I})}\left(\mathbf{x}\right)
	- \widehat{\operatorname{CATE}}\left(\mathbf{x}\right)
\end{equation}
Here, $\overline{\operatorname{CATE}}_{H\left(\mathcal{I}\right)}\left(\mathbf{x}\right)$ is the solution to the following equation, where $\mathcal{I}$ is a fixed partition of $[n]$ and $H$ is a fixed half-sample corresponding to $\mathcal{I}$.
\begin{equation}
	\begin{aligned}
		0 & = \sum_{k = 1}^{K} \sum_{i \in H_k} K(\mathbf{x}, \mathbf{X}_i \, | \, H) m\left(\mathbf{Z_i}; \overline{CATE}_{H\left(\mathcal{I}\right)}\left(\mathbf{x}\right), \hat{\mu}_{k,s}, \hat{\pi}_{k,s}\right)\\
		%
	 	& = \sum_{i = 1}^{n/2 - s + 1} \left[\frac{\binom{n/2-i}{s-1}}{\binom{n/2}{s}} m\left(\mathbf{Z}_{(i \, | \, H)}; \overline{\operatorname{CATE}}_{H\left(\mathcal{I}\right)}\left(\mathbf{x}\right), \hat{\mu}_{k(i \, | \, H),s}, \hat{\pi}_{k(i\, | \, H),s}\right)\right]
	\end{aligned}
\end{equation}
Plugging in for the moment under consideration once more, we find the following.
\begin{equation}
	\begin{aligned}
		{\color{red} LOREM IPSUM}
	\end{aligned}
\end{equation}

\begin{boxD}
	\begin{thm}[Consistent Variance Estimation for the (T)DNN-DML2 CATE Estimator]\label{thm:CATE_DNN_DML_Var_Est}\mbox{}\\*
		{\color{red} LOREM IPSUM}
	\end{thm}
\end{boxD}

\subsection{Pointwise Inference with the TDNN Estimator}
\hrule
\begin{boxD}
	\begin{thm}[Pointwise Inference in Nonparametric Regression]\label{thm:pw_inf_TDNN}\mbox{}\\*
		{\color{red} LOREM IPSUM}
	\end{thm}
\end{boxD}

\begin{boxD}
	\begin{thm}[Pointwise Inference in Heterogeneous Treatment Effect Estimation]\label{thm:pw_inf_TDNN_HTE}\mbox{}\\*
		{\color{red} LOREM IPSUM}
	\end{thm}
\end{boxD}


\newpage
\section{Uniform Inference for the TDNN Estimator}\label{sec:unif_inf}
\hrule

Noteworthy properties of $\kappa$ are its permutational symmetry in $D_{\ell}$ and that $\kappa$ does not consider the response variable when assigning weights to the observations under consideration.
The latter immediately implies a property that has been called ``Honesty'' by \citet{wager_estimation_2018}.

\begin{boxD}
	\begin{dfn}[Symmetry and Honesty - Adapted from \citet{ritzwoller_simultaneous_2024}]\label{Symmetry_Honesty}\mbox{}
		\begin{enumerate}
			\item The kernel $\kappa\left(\cdot, \cdot, D_{\ell}\right)$ is Honest in the sense that
				  $$\kappa\left(x, X_{i}, D_{\ell}\right) \indep m\left(Z_{i} ; \mu\right) \mid X_{i}, D_{\ell,-i},$$
				  where $\indep$ denotes conditional independence.
			\item The kernel $\kappa\left(\cdot, \cdot, D_{\ell}\right)$ is positive and satisfies the restriction
				  $\sum_{i \in s} \kappa\left(\cdot, X_{i}, D_{\ell}\right)=1$ almost surely.
				  Moreover, the kernel $\kappa\left(\cdot, X_{i}, D_{\ell}\right)$ is invariant to permutations of the data $D_{\ell}.$
		\end{enumerate}
	\end{dfn}
\end{boxD}

\hrule
Absent from \citet{demirkaya_optimal_2024} is a way to construct uniformly valid confidence bands around the TDNN estimator.
Luckily, as a byproduct of considering the methods from \citet{ritzwoller_simultaneous_2024}, procedures for simultaneous inference can be developed relatively easily.

To consider this problem in detail we first introduce additional notation.
Instead of a single point of interest, previously denoted by $x$, we will consider a vector of $p$ points of interest denoted by $x^{(p)} \in \left(\operatorname{supp}\left(X\right)\right)^{p}$.
Consequently, the $j$-th entry of $x^{(p)}$ will be denoted by $x^{(p)}_{j}$.
In an abuse of notation, let functions (such as $\mu$ or the DNN/TDNN estimators) evaluated at $x^{(p)}$ denote the vector of corresponding function values evaluated at the point, respectively.
It should be pointed out that, due to the local definition of the kernel in the estimators, this does not translate to the evaluation of the same function at different points in the most immediate sense.
To summarize the kind of object we want to construct, we define a simultaneous confidence region for the TDNN estimator in the following way following closely the notation of \citet{ritzwoller_simultaneous_2024}.

\begin{boxD}
	\begin{dfn}[Uniform Confidence Regions]\mbox{}\\*
		A confidence region for the TDNN (or DNN) estimators that is uniformly valid at the rate $r_{n,d}$ is a family of random intervals
		\begin{equation}
			\hat{\mathcal{C}}\left(x^{(p)}\right)
			:= \left\{\hat{C}(x^{(p)}_{j})
			= \left[c_{L}(x^{(p)}_{j}), c_{U}(x^{(p)}_{j})\right]\, : \, j \in [p]\right\}
		\end{equation}
		based on the observed data, such that
	
		\begin{equation}
			\sup_{P \in \mathbf{P}} \left| P\left(\mu(x^{(d)}) \in \hat{\mathcal{C}}\left(x^{(d)}\right)\right) \right| \leq r_{n,d}
		\end{equation}
		for some sequence $r_{n,d}$, where $\mathbf{P}$ is some statistical family containing $P$.
	\end{dfn}
\end{boxD}

\subsection{Low-Level}
In our pursuit of constructing simultaneous confidence regions for the TDNN estimator, we return to the results from \citet{ritzwoller_simultaneous_2024} in their high-dimensional form.

\begin{boxD}
	\begin{thm}[\citet{ritzwoller_simultaneous_2024} - Theorem 4.1]\label{thm:rit4_1}\mbox{}\\*
		For any sequence of kernel orders $b=b_n$, where
		\begin{equation}
			\frac{1}{n} \frac{\nu_j^2}{\sigma_{b, j}^2} \rightarrow 0
			\quad \text{as} \quad
			n \rightarrow \infty,
		\end{equation}
		we have that
		\begin{equation}
			\sqrt{\frac{n}{\sigma_{b, j}^2 b^2}} \binom{n}{b}^{-1} \sum_{\mathbf{s} \in \mathbf{S}_{n, b}} u\left(x^{(p)}_{j} ; D_{\mathbf{s}}\right) \rightsquigarrow \mathcal{N}(0,1),
			\quad \text{as} \quad
			n \rightarrow \infty.
		\end{equation}
	\end{thm}
\end{boxD}

\begin{boxD}
	\begin{thm}[Adapted from \citet{ritzwoller_simultaneous_2024} - Theorem 4.2]\label{thm:rit4_2}\mbox{}\\*
		Define the terms
		\begin{equation}
			\bar{\psi}_{s_2}^2
			= \max_{j \in[p]}\left\{\nu_j^2- s_2 \sigma_{s_2, j}^2\right\}
			%
			\quad \text {and} \quad
			%
			\underline{\sigma}_{s_2}^2
			= \min_{j \in[p]} \sigma_{s_2, j}^2.
		\end{equation}
		If the kernel function $h_{s_1, s_2}\left(x ; D_{\ell}\right)$ satisfies the bound
		\begin{equation}
			\left\|h_{s_1, s_2}\left(x ; D_{\ell}\right)\right\|_{\psi_1} \leq \phi
		\end{equation}
		for each $j$ in $[d]$, then
		\begin{equation}
			\sqrt{\frac{n}{s_2^2 \underline{\sigma}_{s_2}^2}}
			\left\|\hat{\mu}_{s_1, s_2}(x^{(p)}; \mathbf{D}_n) - \mu(x^{(p)}) - \frac{s_2}{n} \sum_{i=1}^n h^{(1)}_{s_1, s_2}(x^{(p)}; \mathbf{z}_{i})\right\|_{\infty}
			%
			= \sqrt{\frac{n}{s_2^2 \underline{\sigma}_{s_2}^2}} \left\|\operatorname{HR}_{s_1, s_2}(x^{(p)}; \mathbf{D}_n)\right\|_{\infty}
			%
			\lesssim \xi_{n, s_2},
		\end{equation}
		where
		\begin{equation}
			\xi_{n, s_2}
			= \left(\frac{C s_2 \log(p n)}{n}\right)^{s_2 / 2}\left(\left(\frac{n \bar{\psi}_{s_2}^2}{{s_2}^2 \underline{\sigma}_{s_2}^2}\right)^{1 / 2}+\left(\frac{\phi^2 s_2 \log ^4(p n)}{\underline{\sigma}_{s_2}^2}\right)^{1 / 2}\right),
		\end{equation}
		with probability greater than $1-C / n$.
	\end{thm}
\end{boxD}

\subsection{High-Level}
Recent advances in the field of simultaneous inference for infinite-order U-statistics, specifically \citet{ritzwoller_simultaneous_2024}, and careful analysis of the Hoeffding projections of different orders will be the cornerstones in developing simultaneous inference methods.
The authors' approach to constructing simultaneous confidence regions is based on the half-sample bootstrap root.

\begin{boxD}
	\begin{dfn}[Half-Sample Bootstrap Root Approximation - \citet{ritzwoller_simultaneous_2024}]\mbox{}\\*
		The Half-Sample Bootstrap Root Approximation of the sampling distribution of the root
		\begin{equation}
			R\left(x^{(p)}; \mathbf{D}_n\right)
			:= \hat{\mu}\left(x^{(p)}; \mathbf{D}_n\right) - \mu(x^{(p)})
		\end{equation}
		is given by the conditional distribution of the half-sample bootstrap root
		\begin{equation}
			R^{*}\left(x^{(p)}; \mathbf{D}_n\right)
			:= \hat{\mu}\left(x^{(p)}; D_l\right) - \hat{\mu}\left(x^{(p)}; \mathbf{D}_n\right)
		\end{equation}
		where $l$ denotes a random element from $L_{n, n/2}$.
	\end{dfn}
\end{boxD}

Next, to standardize the relevant quantities, we introduce a corresponding studentized process.
\begin{equation}
	\hat{\lambda}_{j}^{2}\left(x^{(p)}; \mathbf{D}_n\right) = \Var\left(\sqrt{n} R^{*}(x^{(p)}_{j}; \mathbf{D}_n) \, | \, \mathbf{D}_n\right)
	\quad \text{and} \quad
	\hat{\Lambda}_n\left(x^{(p)}; \mathbf{D}_n\right) = \operatorname{diag}\left(\left\{\hat{\lambda}_{j}^{2}\left(x^{(p)}; \mathbf{D}_n\right)\right\}_{j = 1}^{p}\right)
\end{equation}
\begin{equation}
	\hat{S}^{*}\left(x^{(p)}; \mathbf{D}_n\right)
	:= \sqrt{n} \, \Big\| \left(\hat{\Lambda}_n\left(x^{(p)}; \mathbf{D}_n\right)\right)^{-1/2} R^{*}\left(x^{(p)}; \mathbf{D}_n\right)\Big\|_{2}
\end{equation}
Let $\operatorname{cv}\left(\alpha; \mathbf{D}_n\right)$ denote the $1-\alpha$ quantile of the distribution of $\hat{S}^{*}\left(x^{(p)}; \mathbf{D}_n\right)$.
As the authors point out specifically, and as indicated by the more explicit notation chosen in this presentation, this is a quantile of the conditional distribution given the data $\mathbf{D}_n$.
Given this construction, the simultaneous confidence region developed in \citet{ritzwoller_simultaneous_2024} adapted to the TDNN estimator takes the following form.

\begin{boxD}
	\begin{thm}[Uniform Confidence Region - \citet{ritzwoller_simultaneous_2024}]\mbox{}\\*
		Define the intervals
		\begin{equation}
			\hat{\mathcal{C}}\left(x^{(p)}_j; \mathbf{D}_n\right)
			:= \hat{\mu}\left(x^{(p)}_{j}; \mathbf{D}_n\right) \pm
			n^{-1/2} \hat{\lambda}_{j}\left(x^{(p)}; \mathbf{D}_n\right)\operatorname{cv}\left(\alpha; \mathbf{D}_n\right)
		\end{equation}
		The $\alpha$-level simultaneous confidence region for $\mu\left(x^{(p)}\right)$ is given by $\hat{\mathcal{C}}\left(x^{(p)}\right)$.
	\end{thm}
\end{boxD}

To justify the use of this simultaneous confidence region, it remains to be shown if and how the other conditions for the inner workings of this procedure apply to the TDNN estimator.
This is substantially simplified due to the absence of a nuisance parameter.
Thus, consider the following conditions from \cite{ritzwoller_uniform_2024} that are simplified to fit the problem at hand.

\begin{boxD}
	\begin{dfn}[Shrinkage and Incrementality - Adapted from \citet{ritzwoller_simultaneous_2024}]\mbox{}\\*
		We say that the kernel $\kappa\left(\cdot, \cdot, D_{\ell}\right)$ has a simultaneous shrinkage rate $\epsilon_b$ if
		\begin{equation}
			\sup_{P \in \mathbf{P}} \sup_{j \in[p]}
			\E\left[\max \left\{\left\|X_{i}-x^{(p)}_{j}\right\|_{2}: \kappa\left(x^{(p)}_{j}, X_{i}, D_{\ell}\right)>0\right\}\right]
			\leq \epsilon_b .
		\end{equation}
		We say that a kernel $\kappa\left(\cdot, \cdot, D_{\ell}\right)$ is uniformly incremental if
		\begin{equation}
			\inf_{P \in \mathbf{P}} \sup_{j \in[p]}
			\Var\left(\E\left[\sum_{i \in \ell} \kappa\left(x^{(p)}_{j}, X_{i}, D_{\ell}\right) m\left(Z_{i} ; \mu\right) \mid l \in \ell, Z_l = Z\right]\right)
			\gtrsim b^{-1}
		\end{equation}
		where $Z$ is an independent random variable with distribution $P$.
	\end{dfn}
\end{boxD}

Translating these properties to suit the TDNN regression problem, we obtain the following conditions that need to be verified.
First, to verify simultaneous shrinkage at a rate $\epsilon_b$, the following remains to be shown.
\begin{equation}
	\sup_{P \in \mathbf{P}} \sup_{j \in[p]}
	\E\left[\max \left\{\left\|X_{i}-x^{(p)}_{j}\right\|_{2}:
	\rk(x^{(p)}_{j}; X_{i}, D_{\ell}) = 1\right\}\right]
	\leq \epsilon_b
\end{equation}
Second, for simultaneous incrementality, we need to show the following.
\begin{equation}
	\begin{aligned}
		  & \inf_{P \in \mathbf{P}} \sup_{j \in[p]}
		\Var\left(\E\left[
				\sum_{i \in \ell} \1\left(\rk(x^{(p)}_{j}; X_{i}, D_{\ell}) = 1\right) \left(Y_{i} - \mu\left(X_{i}\right)\right) \mid l \in \ell, Z_l = Z\right]
		\right)                                     \\
		%
		= & \inf_{P \in \mathbf{P}} \sup_{j \in[p]}
		\Var\left(
		\sum_{i \in \ell}\E\left[\1\left(\rk(x^{(p)}_{j}; X_{i}, D_{\ell}) = 1\right) \varepsilon_i \mid l \in \ell, Z_l = Z\right]
		\right)                                     \\
		%
		= & \inf_{P \in \mathbf{P}} \sup_{j \in[p]}
		\Var\left(
		\sum_{i = 1}^{s}\E\left[\1\left(\rk(x^{(p)}_{j}; X_{i}, D_{1:s}) = 1\right) \varepsilon_i \mid l \in [s], Z_l = Z\right]
		\right)                                     \\
		%
		= & \inf_{P \in \mathbf{P}} \sup_{j \in[p]}
		s^2 \cdot \Var\left(
		\E\left[\1\left(\rk(x^{(p)}_{j}; X_1, D_{1:s}) = 1\right) \varepsilon_1 \mid l \in [s], Z_l = Z\right]
		\right)
		\gtrsim b^{-1}
	\end{aligned}
\end{equation}

To verify these assumptions, recent theory developed in \citet{peng_rates_2022} is of great help.
Specifically, the following Proposition and its proof are helpful in showing the desired simultaneous incrementality property.

	{\color{red} LOREM IPSUM}

\begin{boxD}
	\begin{asm}[Boundedness - Adapted from \citet{ritzwoller_simultaneous_2024}]\mbox{}\\*
		The absolute value of the function $m(\cdot; \mu)$ is bounded by the constant $(\theta+1) \phi$ almost surely.
		\begin{equation}
			|m(Z_{i} ; \mu)|
			= |Y_{i} - \mu(X_{i})|
			= |\varepsilon_i|
			\leq (\theta+1) \phi
			\quad \text{a.s.}
		\end{equation}
	\end{asm}
\end{boxD}
To follow the notational conventions, we will further define the two functions $m^{(1)}(Z_{i}; \mu) = - \mu(X_{i})$ and $m^{(2)}(Z_{i}) = Y_{i}$.
As the authors point out, the boundedness condition can easily be replaced by a condition on the subexponential norm.
This, being more in line with the assumptions of \citet{demirkaya_optimal_2024}, is a desirable substitution.
Thus, we will instead consider the following assumption and fill in parts of the proofs that hinge on boundedness for ease of exposition in the original paper.

\begin{boxD}
	\begin{asm}[Sub-Exponential Norm Bound]\mbox{}\\*

	\end{asm}	
\end{boxD}


{\color{red} LOREM IPSUM}

% 
% \begin{assumption}[Consistency]

% \end{assumption}

% {\color{red} LOREM IPSUM}

\begin{boxD}
	\begin{asm}[Moment Smoothness - Adapted from \citet{ritzwoller_simultaneous_2024}]\mbox{}\\*
		Define the moments
		\begin{equation}
			M^{(1)}(x ; \mu)
			= \E\left[m^{(1)}\left(Z_{i} ; \mu\right) \mid X_{i}= x\right]
			\quad \text{and} \quad
			M^{(2)}(x)
			= \E\left[m^{(2)}\left(Z_{i}\right) \mid X_{i} = x\right],
		\end{equation}
		associated with the functions $m^{(1)}(\cdot ; \mu)$ and $m^{(2)}(\cdot)$.
		Plugging in yields the following functions.
		\begin{equation}
			M^{(1)}(x ; \mu)
			= -\mu(x)
			\quad \text{and} \quad
			M^{(2)}(x)
			= \mu(x).
		\end{equation}
		Both moments are uniformly Lipschitz in their first component, in the sense that
		\begin{equation}
			\forall x, x^{\prime} \in \operatorname{supp}\left(X\right): \quad
			\sup _{P \in \mathbf{P}}
			\left|\mu(x)-\mu\left(x^{\prime}\right)\right|
			\lesssim\left\|x-x^{\prime}\right\|_{2}.
		\end{equation}
		and $M^{(1)}$ is bounded below in the following sense
		\begin{equation}
			\inf_{P \in \mathbf{P}} \inf_{j \in [p]} \Big|M^{(1)}\left(x^{(p)}_{j}\right) \Big|
			= \inf_{P \in \mathbf{P}} \inf_{j \in [p]} \Big|\mu\left(x^{(p)}_{j}\right) \Big| \geq c
		\end{equation}
		for some positive constant $c$.
	\end{asm}
\end{boxD}


The Lipschitz continuity part of this assumption translates directly into a Lipschitz continuity assumption on the unknown nonparametric regression function.
The boundedness assumption is
	{\color{red} LOREM IPSUM}

\subsection{Uniform Inference with the TDNN Estimator}
\begin{boxD}
	\begin{thm}[Uniform Inference in Nonparametric Regression]\label{thm:unif_inf_TDNN}\mbox{}\\*
		{\color{red} LOREM IPSUM}
	\end{thm}
\end{boxD}

\begin{boxD}
	\begin{thm}[Uniform Inference in Heterogeneous Treatment Effect Estimation]\label{thm:unif_inf_TDNN_HTE}\mbox{}\\*
		{\color{red} LOREM IPSUM}
	\end{thm}
\end{boxD}


\newpage
\section{Simulations}\label{sec:simulations}
\hrule

Having developed theoretical results concerning simultaneous inference methods for the TDNN estimator, we will proceed by testing their properties in several simulation studies.

\subsection{Nonparametric Regression}
\hrule
To investigate the practicality of the nonparametric regression estimators presented in this paper, we consider a collection of setups.
First, we focus on illustrating the bias correcting properties of the TDNN estimator by replicating some of the findings of \citet{demirkaya_optimal_2024}.
One such promising example is shown in Figure~\ref{fig:TDNN_bias_cor} highlighting the potential improvements obtainable by combining multiple subsampling scales.
\begin{figure}[H]
	\centering
	\includegraphics[width = \textwidth]{../Code/Simulations/Graphics/TDNN_DNN.pdf}
	\caption{Comparison of the DNN ($s = 20$) and TDNN ($s_1 = 20, s_2 = 50$) Estimators for different sample sizes.
		The dashed line indicates the value of the unknown regression function at the point of interest.
		Simulation Setup replicates Setting 1 from \citet{demirkaya_optimal_2024} for 10000 Monte Carlo Replications.}
	\label{fig:TDNN_bias_cor}
\end{figure}
As can be seen in Figure~\ref{fig:TDNN_bias_cor}, a suitable choice of subsampling scales can effectively reduce the bias of the TDNN estimator when compared with the DNN estimator.
This reinforces the idea that the TDNN estimator can be a useful tool in practice that has potential to improve on well-established nearest neighbor methods.

\newpage
As a second, potentially more illustrative example, we consider the estimation of a function of two arguments.
Specifically, we consider the function $\mu(x) = 5 \cdot \left(\cos(x_1) + \cos(x_2)\right)$ on ${[0,1]}^2$ with heteroskedastic error terms whose variance is determined by $\sigma_{\varepsilon}^2(x) = \frac{1}{16}{\left(x_1^2 + x_2^2\right)}^2$.
The resulting surface is depicted in Figure~\ref{fig:reg_surface}.
\begin{figure}[H]
	\centering
	\includegraphics[width = \textwidth]{../Graphics/Reg_Exmp1.pdf}
	\caption{Value of the Regression Function (left) and Variance of the Error Term (right)}
	\label{fig:reg_surface}
\end{figure}
To analyze the behavior of the estimator in this setting, we run a number of Monte-Carlo simulations each consisting of 10000 simulation runs.
While the theoretical analysis was of purely asymptotic nature, these simulation results can provide a modicum of guidance when it comes to choices such as the kernel orders employed in the estimation procedure.
Each run consists of 10000 observations that are uniformly distributed on ${[0,1]}^2$, we find the following concerning the estimators bias and variance given different kernel orders $s_1$ and $s_2$.
\begin{figure}[H]
	\centering
	\includegraphics[width = \textwidth]{../Code/Simulations/Graphics/Reg_Exp1/TDNN/Plot_TDNN_n10000s_10_25.RDS.pdf}
	\caption{Approximate Bias (left) and Variance (right) of the TDNN Estimator with $s_1 = 10$ and $s_2 = 25$}
	\label{fig:est_bias_var_2}
\end{figure}
\begin{figure}[H]
	\centering
	\includegraphics[width = \textwidth]{../Code/Simulations/Graphics/Reg_Exp1/TDNN/Plot_TDNN_n10000s_1000_2500.RDS.pdf}
	\caption{Approximate Bias (left) and Variance (right) of the TDNN Estimator with $s_1 = 1000$ and $s_2 = 2500$}
	\label{fig:est_bias_var}
\end{figure}


\subsection{CATE-Estimation}
\hrule

\begin{figure}[H]
	\centering
	\includegraphics[width = \textwidth]{../Graphics/CATE_Exmp1.pdf}
	\caption{Value of the Regression Functions $\mu_0$ (upper) and $\mu_1$ (lower).	Error term structure remains unchanged.}
	\label{fig:CATE_surfaces}
\end{figure}

{\color{red} LOREM IPSUM}

\newpage
\section{Application}\label{sec:application}
\hrule
{\color{red} LOREM IPSUM}

\newpage
\section{Conclusion}\label{sec:conclusion}
\hrule
{\color{red} LOREM IPSUM}

\newpage
\printbibliography

\newpage
\appendix

\newpage
\section{The DNN Estimators as a Generalized U-Statistics}
\hrule

As most of the theoretical results in \citet{demirkaya_optimal_2024} rely on representations as a U-statistic, it is helpful to introduce additional concepts and notation at this stage.
In analogy to the main part of this paper, I will first start with the regression context and then go on to consider the CATE estimation scenario separately.
Recalling Equation~\ref{eq:U_stat}, the DNN and TDNN estimators can be expressed in the following U-statistic form and are thus a type of generalized complete U-statistic as introduced by \citet{peng_rates_2022}.
\begin{equation}
	\tilde{\mu}_{s}(x; \mathbf{D}_n)
	= \binom{n}{s}^{-1} \sum_{\ell \in L_{n,s}} h_{s}(x; D_{\ell})
	\quad \text{and} \quad
	\hat{\mu}_{s_1, s_2}(x; \mathbf{D}_n)
	= \binom{n}{s}^{-1} \sum_{\ell \in L_{n,s_2}} h_{s_1, s_2}(x; D_{\ell})
\end{equation}
It is worth pointing out that in contrast to the DNN estimator, the kernel for the TDNN estimator is of order $s_2 > s_1$.
The authors derive an explicit formula for the kernel that shows the connection between the DNN and TDNN estimators.
This connection will prove useful going forward.
\begin{boxD}
	\begin{lem}[Kernel of TDNN Estimator - Adapted from Lemma 8 of \citet{demirkaya_optimal_2024}]\label{lem:dem8}\mbox{}\\*
		The kernel of the TDNN estimator takes the following form.
		\begin{equation}
			\begin{aligned}
				h_{s_1, s_2}\left(x; D\right)
				 & = w_{1}^{*}\left[\binom{s_2}{s_1}^{-1}\sum_{\ell \in L_{s_2, s_1}} h_{s_1}\left(x; D_{\ell}\right)\right] + w_{2}^{*} h_{s_2}\left(x; D\right) \\
				 & = w_{1}^{*} \tilde{\mu}_{s_1}\left(x; D\right) + w_{2}^{*} h_{s_2}\left(x; D\right)                                                            \\
			\end{aligned}
		\end{equation}
	\end{lem}
\end{boxD}
Borrowing the notational conventions from \citet{lee_u-statistics_2019}, I additionally introduce the following notation.
\begin{equation}\label{eq:psi_s_c}
	\psi_{s}^{c}(x; \mathbf{z}_{1}, \dotsc, \mathbf{z}_{c})
	= \E_{D}\left[h_{s}\left(x; D\right) \, | \,  Z_1 = \mathbf{z}_{1}, \dotsc, Z_c = \mathbf{z}_{c}\right]
\end{equation}
\begin{equation}
	h_{s}^{(1)}\left(x; \mathbf{z}_{1}\right)
	= \psi_{s}^{1}(x; \mathbf{z}_{1}) - \mu(x)
\end{equation}
\begin{equation}
	h_{s}^{(c)}\left(x; \mathbf{z}_{1}, \dotsc, \mathbf{z}_{c}\right)
	= \psi_{s}^{c}(x; \mathbf{z}_{1}, \dotsc, \mathbf{z}_{c}) - \sum_{j = 1}^{c-1}\left(\sum_{\ell \in L_{n,j}}h_{s}^{(j)}(x; \mathbf{z}_{\ell})\right) - \mu(x)
	\quad \text{for } c = 2, \dotsc, s
\end{equation}
In contrast to the notational inspiration, the subsampling size $s$ is made explicit.
Since we are dealing with an infinite-order U-statistic, $s$ will be diverging with $n$.
Completely analogous, define the corresponding objects for the TDNN estimator.
For the DNN estimator and any $1 \leq c \leq s$, define
\begin{equation}\label{eq:xi_s_c}
	\xi_{s}^{c}\left(x\right)
	= \Var_{1:c}\left(\psi_{s}^{c}(x; Z_{1}, \dotsc, Z_{c})\right)
\end{equation}
where $Z_{c+1}^{\prime}, \ldots, Z_n^{\prime}$ are i.i.d.\ from $P$ and independent of $Z_1, \ldots, Z_n$ and thus
$\xi_{s}^{s}\left(x\right) = \Var\left(h_s\left(x; Z_1, \ldots, Z_s\right)\right)$.
% Then, I have the following result from the original paper.
Similarly, for the TDNN estimator and any $1 \leq c \leq s_2$, let
\begin{equation}\label{eq:zeta_s1s2_c}
	\zeta_{s_1, s_2}^{c}\left(x\right)
	= \Var_{1:c}\left(\psi_{s_1, s_2}^{c}(x; Z_{1}, \dotsc, Z_{c})\right)
\end{equation}
with an analogous definition of $Z^{\prime}$.
As a byproduct (or main purpose depending on the perspective) these terms can be used to derive the Hoeffding decomposition of the TDNN estimator.
\begin{equation}\label{eq:H_projection}
	\begin{aligned}
		H_{s}^{c}\left(x; \mathbf{D}_n\right)
		= \binom{n}{c}^{-1} \sum_{\ell \in L_{n,c}} h^{(c)}_{s}(x; D_{\ell})
		\quad \text{and} \quad
		H_{s_1, s_2}^{c}\left(x; \mathbf{D}_n\right)
		= \binom{n}{c}^{-1} \sum_{\ell \in L_{n,c}} h^{(c)}_{s_1, s_2}(x; D_{\ell})
	\end{aligned}
\end{equation}
These projection terms can then be used to construct the following Hoeffding decompositions.
\begin{equation}\label{eq:H_Decomp}
	\tilde{\mu}_{s}\left(x; \mathbf{D}_n\right)
	= \mu(x) + \sum_{j = 1}^{s}\binom{s}{j}H_{s}^{j}\left(x; \mathbf{D}_n\right)\\
	\quad \text{and} \quad
	\hat{\mu}_{s_1, s_2}\left(x; \mathbf{D}_n\right)
	= \mu(x) + \sum_{j = 1}^{s_2}\binom{s_2}{j}H_{s_1, s_2}^{j}\left(x; \mathbf{D}_n\right)
\end{equation}

Standard results for U-statistics (see, for example, \citet{lee_u-statistics_2019}) now give us a number of useful results.
First, an immediate result on the expectations of the Hoeffding-projection kernels.
\begin{align}\label{eq:H_k_expectation}
	\forall c = 1,2,\dotsc, j-1: \quad & \E_{D}\left[h_{s_1, s_2}^{(j)}\left(x; D\right) \, | \, Z_1 = \mathbf{z}_1, \dotsc, Z_c = \mathbf{z}_c\right] = 0
	\quad \text{and} \quad
	\E_{D}\left[h_{s_1, s_2}^{(j)}\left(x; D\right)\right] = 0
\end{align}
Second, I obtain a useful variance decomposition in terms of the Hoeffding-projection variances.
\begin{align}\label{eq:Var_decomp}
	\Var_{D}\left(\hat{\mu}_{s_1, s_2}\left(x; D\right)\right)
	 & = \sum_{j = 1}^{s_2} \binom{s_2}{j}^2 \Var_{D}\left(H_{s_1, s_2}^{j}\left(x; D\right)\right) \\
	%
	\Var_{D}\left(H_{s_1, s_2}^{j}\left(x; D\right)\right)
	 & = \binom{n}{j}^{-1} \Var_{D}\left(h_{s_1, s_2}^{(j)}\left(x; D\right)\right)
	=: \binom{n}{j}^{-1} V_{s_1, s_2}^{j}\left(x\right)
\end{align}
Third, the following equivalent expression for the kernel variance.
\begin{equation}\label{eq:k_var}
	\zeta_{s_1, s_2}^{s_2}\left(x\right)
	= \Var_{D}\left(h_{s_1, s_2}\left(x; D\right)\right)
	= \sum_{j = 1}^{s_2} \binom{s_2}{j}V_{s_1, s_2}^{j}
\end{equation}

\subsection{CATE-Estimators as Generalized U-Statistics}
\hrule

Given estimates of the functional nuisance parameters, the proposed CATE estimators can be analyzed as generalized U-statistics in the same way as in the nonparametric regression context.
As most of the theoretical results on these estimators will similarly rely on Hoeffding projection arguments, I will introduce analogous notation in this more general scenario.
First, observe that the DNN-DML2 CATE estimator can be rewritten as follows to explicitly show its construction as a generalized U-statistic.
\begin{equation}
    \begin{aligned}
        \widehat{\operatorname{CATE}}\left(x; \hat{\mu}, \hat{\pi}\right) 
        & = \sum_{i = 1}^{n - s + 1} \frac{\binom{n-i}{s-1}}{\binom{n}{s}}
		\left[\hat{\mu}_{k(i),s}^{1}\left(X_{(i)}\right) - \hat{\mu}_{k(i),s}^{0}\left(X_{(i)}\right) + \hat{\beta}_{k(i),s}\left(W_{(i)}, X_{(i)}\right)\left(Y_{(i)} - \hat{\mu}^{W_{(i)}}_{k(i),s}\left(X_{(i)}\right)\right)\right] \\
        %
        & = \binom{n}{s}^{-1} \sum_{\ell \in L_{n,s}}
        \underbrace{\sum_{i = 1}^{n}\frac{\1\left(\rk(x; Z_{i}, D_{\ell}) = 1\right)}{s!} 
        \left[\hat{\mu}_{k_i,s}^{1}\left(X_{i}\right) - \hat{\mu}_{k_i,s}^{0}\left(X_{i}\right) + \hat{\beta}_{k_i,s}\left(W_{i}, X_{i}\right)\left(Y_{i} - \hat{\mu}^{W_{i}}_{k_i,s}\left(X_{i}\right)\right)\right]}_{\chi_{s}(x; \mathbf{D}_{\ell}, \hat{\mu}, \hat{\pi})} \\
        %
        & = \binom{n}{s}^{-1} \sum_{\ell \in L_{n,s}}\chi_{s}(x; \mathbf{D}_{\ell}, \hat{\mu}, \hat{\pi})
    \end{aligned}
\end{equation}
In the same fashion as in the previous case, I can now define the Hoeffding decomposition for the estimator.
Note that in this step, I continue to take the estimates of the functional nuisance parameters as exogenously given.
\begin{equation}
    \vartheta_{s}^{c}\left(x; \mathbf{z}_{1}, \dotsc, \mathbf{z}_{c}, \hat{\mu}, \hat{\pi}\right)
    & = \E_{D}\left[\chi_{s}(x; \mathbf{D}, \hat{\mu}, \hat{\pi}) \, \middle| \, Z_1 = \mathbf{z}_1, \dotsc, Z_c = \mathbf{z}_c, \hat{\mu}, \hat{\pi}\right]
\end{equation}
\begin{equation}
    \chi_{s}^{(1)}\left(x; \mathbf{z}_{1}, \hat{\mu}, \hat{\pi}\right)
	= \vartheta_{s}^{1}\left(x; \mathbf{z}_{1}, \hat{\mu}, \hat{\pi}\right)
    - \E_{D}\left[\chi_{s}(x; \mathbf{D}, \hat{\mu}, \hat{\pi}) \, \middle| \, \hat{\mu}, \hat{\pi}\right]
\end{equation}
As before, I define the higher-order projection terms, i.e. for $c = 2, \dotsc, s$, in the following way.
\begin{equation}
    \chi_{s}^{(c)}\left(x; \mathbf{z}_{1}, \dotsc, \mathbf{z}_{c}, \hat{\mu}, \hat{\pi}\right)
	& = \vartheta_{s}^{c}(x; \mathbf{z}_{1}, \dotsc, \mathbf{z}_{c}, \hat{\mu}, \hat{\pi}) 
    - \sum_{j = 1}^{c-1}\left(\sum_{\ell \in L_{n,j}} \chi_{s}^{(j)}(x; \mathbf{z}_{\ell}, \hat{\mu}, \hat{\pi})\right) 
    - \E_{D}\left[\chi_{s}(x; \mathbf{D}, \hat{\mu}, \hat{\pi}) \, \middle| \, \hat{\mu}, \hat{\pi}\right]
\end{equation}
In anticipation of arguments involving empirical process theory, I define as follows.
\begin{equation}
    \chi_{s, 0}(x; \mathbf{D}_{[s]})
    = \chi_{s}(x; \mathbf{D}_{[s]}, \mu, \pi)
\end{equation}
\begin{equation}
    \vartheta_{s,0}^{c}\left(x; \mathbf{z}_{1}, \dotsc, \mathbf{z}_{c}\right)
    & = \vartheta_{s}^{c}\left(x; \mathbf{z}_{1}, \dotsc, \mathbf{z}_{c}, \mu, \pi\right)
\end{equation}
\begin{equation}
    \chi_{s,0}^{(1)}\left(x; \mathbf{z}_{1}\right)
	= \chi_{s}^{(1)}\left(x; \mathbf{z}_{1}, \mu, \pi\right)
    = \vartheta_{s,0}^{1}\left(x; \mathbf{z}_{1}\right) 
    - \E_{D}\left[\chi_{s,0}\left(x; \mathbf{D}\right)\right]
\end{equation}
\begin{equation}
    \begin{aligned}
        \chi_{s,0}^{(c)}\left(x; \mathbf{z}_{1}, \dotsc, \mathbf{z}_{c}\right)
	    & = \chi_{s}^{(c)}\left(x; \mathbf{z}_{1}, \dotsc, \mathbf{z}_{c}, \mu, \pi\right) \\
        %
        & = \vartheta_{s,0}^{c}(x; \mathbf{z}_{1}, \dotsc, \mathbf{z}_{c}) 
        - \sum_{j = 1}^{c-1}\left(\sum_{\ell \in L_{n,j}} \chi_{s,0}^{(j)}(x; \mathbf{z}_{\ell})\right) 
        - \E_{D}\left[\chi_{s,0}^{(c)}\left(x; Z_{1}, \dotsc, Z_{c}\right)\right]
    \end{aligned}
\end{equation}
Furthermore, in an overloading of notation, I define as follows.
\begin{equation}
    \vartheta_{s}^{c}\left(x; \mathbf{z}_{1}, \dotsc, \mathbf{z}_{c}\right)
    & = \E_{D}\left[\chi_{s}(x; \mathbf{D}, \hat{\mu}, \hat{\pi}) \, \middle| \, Z_1 = \mathbf{z}_1, \dotsc, Z_c = \mathbf{z}_c\right]
\end{equation}
\begin{equation}
    \chi_{s}^{(1)}\left(x; \mathbf{z}_{1}\right)
	= \vartheta_{s}^{1}\left(x; \mathbf{z}_{1}\right)
    - \E_{D}\left[\chi_{s}(x; \mathbf{D}, \hat{\mu}, \hat{\pi})\right]
\end{equation}
\begin{equation}
    \chi_{s}^{(c)}\left(x; \mathbf{z}_{1}, \dotsc, \mathbf{z}_{c}\right)
	& = \vartheta_{s}^{c}(x; \mathbf{z}_{1}, \dotsc, \mathbf{z}_{c}) 
    - \sum_{j = 1}^{c-1}\left(\sum_{\ell \in L_{n,j}} \chi_{s}^{(j)}(x; \mathbf{z}_{\ell})\right) 
    - \E_{D}\left[\chi_{s}(x; \mathbf{D}, \hat{\mu}, \hat{\pi})\right]
\end{equation}
To explicitly point out the difference between these definitions, note that we are now not taking the estimates of the functional nuisance parameters as exogenously given.
Instead, we also consider the corresponding expectations with respect to the first-stage estimates.
These definitions allow us to use the following decomposition where now I acknowledge the randomness in the first-stage estimation.
\begin{equation}\label{eq:DNNDML2_ResidDecomp}
    \begin{aligned}
        \chi_{s}^{(c)}\left(x; \mathbf{D}_{\ell}, \hat{\mu}, \hat{\pi}\right)
        & = \underbrace{\chi_{s}^{(c)}\left(x; \mathbf{D}_{\ell}, \hat{\mu}, \hat{\pi}\right) - \chi_{s}^{(c)}\left(x; \mathbf{D}_{\ell}\right)}_{A_{c}\left(x; \mathbf{D}_{\ell}\right)}
        + \underbrace{\chi_{s}^{(c)}\left(x; \mathbf{D}_{\ell}\right) - \chi_{s,0}^{(c)}\left(x; \mathbf{D}_{\ell}\right)}_{B_{c}\left(x; \mathbf{D}_{\ell}\right)} \\
        & + \underbrace{\chi_{s,0}^{(c)}\left(x; \mathbf{D}_{\ell}\right) - \E_{D}\left[\chi_{s,0}^{(c)}\left(x; \mathbf{D}_{\ell}\right)\right]}_{C_{c}\left(x; \mathbf{D}_{\ell}\right)}
        + \underbrace{\E_{D}\left[\chi_{s,0}^{(c)}\left(x; \mathbf{D}_{\ell}\right)\right]}_{D_{c}\left(x\right)}
    \end{aligned}
\end{equation}
Considering this decomposition will be of great usefulness when analyzing the asymptotic properties of the CATE estimators.

Plugging into the full expression for the CATE estimator, the following is obtained.
\begin{equation}\label{eq:DNNDML2_Decomp}
    \begin{aligned}
        \widehat{\operatorname{CATE}}\left(x\right)
        & = \E_{D}\left[\widehat{\operatorname{CATE}}\left(x\right)\right] 
        + \sum_{j = 1}^{s} \binom{s}{j} \binom{n}{j}^{-1}\sum_{\ell \in L_{n,j}} \left[A_{j}\left(x; \mathbf{D}_{\ell}\right) + B_{j}\left(x; \mathbf{D}_{\ell}\right) + C_{j}\left(x; \mathbf{D}_{\ell}\right) + D_{j}\left(x\right)\right]\\
        %
        & = \underbrace{\E_{D}\left[\widehat{\operatorname{CATE}}\left(x\right)\right]}_{\text{Centering-Term}}
        + \underbrace{\frac{s}{n} \sum_{i = 1}^{n} \left[A_{1}\left(x; Z_{i}\right) + B_{1}\left(x; Z_{i}\right) + C_{1}\left(x; Z_{i}\right) + D_{1}\left(x\right)\right]}_{\text{Pseudo-H\'ajek-Projection}}\\
        & \quad \quad + \underbrace{\sum_{j = 2}^{s} \binom{s}{j} \binom{n}{j}^{-1}\sum_{\ell \in L_{n,j}} \left[A_{j}\left(x; \mathbf{D}_{\ell}\right) + B_{j}\left(x; \mathbf{D}_{\ell}\right) + C_{j}\left(x; \mathbf{D}_{\ell}\right) + D_{j}\left(x\right)\right]}_{\text{Pseudo-H\'ajek-Residual}}
    \end{aligned}
\end{equation}
I call these terms pseudo-H\'ajek projection and pseudo-H\'ajek residual because of the additional randomness due to the first-stage estimation of the functional nuisance parameters.
An integral part of showing the asymptotic properties of the estimator is to show that a number of terms in this representation are asymptotically negligible as long as the first-stage estimator converges fast enough.

\newpage
\section{Useful Results}
\hrule

\begin{lem}[\citet{demirkaya_optimal_2024} - Lemma 12]\label{lem:dem12}\mbox{}\\*
	Let $D = \{Z_1, \dotsc, Z_s\}$ an i.i.d.\ sample drawn from $P$.
	The indicator functions $\kappa\left(x; Z_{i}, D\right)$ satisfy the following properties.
	\begin{enumerate}
		\item For any $i \neq j$, we have $\kappa\left(x; Z_{i}, D\right) \kappa\left(x; Z_{j}, D\right)=0$ with probability one;
		\item $\sum_{i=1}^{s} \kappa\left(x; Z_{i}, D\right)=1$;
		\item $\forall i \in [s]: \quad \E_{1:s}\left[\kappa\left(x; Z_{i}, D\right)\right]=s^{-1}$
		\item $\E_{2: s}\left[\kappa\left(x; Z_1, D\right)\right]=\left\{1-\varphi\left(B\left(x,\left\|X_1-x\right\|\right)\right)\right\}^{s-1}$
	\end{enumerate}
	Here $\E_{i: s}$ denotes the expectation with respect to $\left\{Z_{i}, Z_{i+1}, \dotsc, Z_s\right\}$.
	Furthermore, $\varphi$ denotes the probability measure on $\mathbb{R}^{d}$ induced by the random vector $X$.
\end{lem}

\hrule

\begin{lem}[\citet{demirkaya_optimal_2024} - Lemma 13]\label{lem:dem13}\mbox{}\\*
	For any $L^1$ function $f$ that is continuous at $x$, it holds that
	\begin{equation}
		\lim _{s \rightarrow \infty} \E_{1}\left[f\left(X_1\right) s \E_{2:s}\left[\kappa(x; Z_1, D)\right]\right]
		= f(x).
	\end{equation}
\end{lem}

\hrule

\begin{lem}\label{lem:limit_res}\mbox{}\\*
	As a consequence of Lemma~\ref{lem:dem13}, we find the following limit results in the nonparametric regression setup.
	\begin{equation}
		\begin{aligned}
			\lim_{s \rightarrow \infty} \E_{1}\left[Y_1 s \E_{2:s}\left[\kappa(x; Z_1, D)\right]\right]
			& = \mu\left(x\right)
		\end{aligned}
	\end{equation}
	\begin{equation}
		\begin{aligned}
			\lim_{s \rightarrow \infty} \E_{1}\left[Y_1^2 s \E_{2:s}\left[\kappa(x; Z_1, D)\right]\right]
			& = \mu^2\left(x\right) + \sigma_{\varepsilon}^{2}(x)
			\leq \mu^2\left(x\right) + \overline{\sigma}_{\varepsilon}^{2}
		\end{aligned}
	\end{equation}
	Similarly, in the CATE estimation setup, we can make the following observations.
	\begin{equation}
		\begin{aligned}
			\lim_{s \rightarrow \infty} \E_{1}\left[m\left(Z_{1}; \mu, \pi\right) s \E_{2:s}\left[\kappa(x; Z_1, D)\right]\right]
			& = \mu_1\left(x\right) - \mu_0\left(x\right)
		\end{aligned}
	\end{equation}          
	\begin{equation}
		\begin{aligned}
			\lim_{s \rightarrow \infty} \E_{1}\left[m^2\left(Z_{1}; \mu, \pi\right) s \E_{2:s}\left[\kappa(x; Z_1, D)\right]\right]
			& = \left(\mu_1\left(x\right) - \mu_0\left(x\right)\right)^2 + \frac{\sigma_{\varepsilon}^2(x)}{\pi\left(x\right)\left(1 - \pi\left(x\right)\right)}\\
			%
			& \leq \left(\mu_1\left(x\right) - \mu_0\left(x\right)\right)^2 + \frac{\overline{\sigma}_{\varepsilon}^2}{\mathfrak{p}\left(1 - \mathfrak{p}\right)}
		\end{aligned}
	\end{equation}
\end{lem}

\hrule

\begin{proof}[Proof of Lemma~\ref{lem:limit_res}]
	Starting with the first limit, we find the following.
	\begin{equation}
		\begin{aligned}
			\E_{1}\left[Y_1 s \E_{2:s}\left[\kappa(x; Z_1, D)\right]\right]
			& = \E_{1}\left[\left(\mu\left(X_1\right) + \varepsilon_1\right) s \E_{2:s}\left[\kappa(x; Z_1, D)\right]\right]\\
			%
			& = \E_{1}\left[\left(\mu\left(X_1\right) + \E\left[\varepsilon_1 \, \middle| \, X_1\right]\right) s \E_{2:s}\left[\kappa(x; Z_1, D)\right]\right]\\
			%
			& = \E_{1}\left[\mu\left(X_1\right) s \E_{2:s}\left[\kappa(x; Z_1, D)\right]\right]
			\overset{\text{(Lem~\ref{lem:dem13})}}{\longrightarrow} \mu\left(x\right)
			\quad \text{as} \quad s \rightarrow \infty
		\end{aligned}
	\end{equation}
	Similarly, when considering the second limit, we can make the following observation.
	\begin{equation}
		\begin{aligned}
			\E_{1}\left[Y_1^2 s \E_{2:s}\left[\kappa(x; Z_1, D)\right]\right]
			& = \E_{1}\left[\left(\mu\left(X_1\right) + \varepsilon_1\right)^2 s \E_{2:s}\left[\kappa(x; Z_1, D)\right]\right]\\
			%
			& = \E_{1}\left[\left(\mu^2\left(X_1\right) + 2\mu\left(X_1\right)\varepsilon_1 + \varepsilon_1^2\right)s \E_{2:s}\left[\kappa(x; Z_1, D)\right]\right]\\
			%
			& = \E_{1}\left[\left(\mu^2\left(X_1\right) + 2\mu\left(X_1\right) \E\left[\varepsilon_1 \, \middle| \, X_1\right] + \E\left[\varepsilon_1^2 \, \middle| \, X_1\right]\right)
			s \E_{2:s}\left[\kappa(x; Z_1, D)\right]\right] \\
			%
			& = \E_{1}\left[\left(\mu^2\left(X_1\right) +\sigma_{\varepsilon}^{2}(X_1)\right) s \E_{2:s}\left[\kappa(x; Z_1, D)\right]\right] \\
			%
			& \overset{\text{(Lem~\ref{lem:dem13})}}{\longrightarrow} \mu^2\left(x\right) +\sigma_{\varepsilon}^{2}(x)
			\quad \text{as} \quad s \rightarrow \infty
		\end{aligned}
	\end{equation}
	In the CATE estimation setting, we can proceed analogously.
	\begin{equation}
		\begin{aligned}
			\E_{1}\left[m\left(Z_{1}; \mu, \pi\right) s \E_{2:s}\left[\kappa(x; Z_1, D)\right]\right]
			& = \E_{1}\left[\left(\mu_1\left(X_{1}\right) - \mu_0\left(X_{1}\right) + \beta\left(W_{1}, X_{1}\right)\varepsilon_{i}\right) s \E_{2:s}\left[\kappa(x; Z_1, D)\right]\right] \\
			%
			& = \E_{1}\left[\left(\mu_1\left(X_{1}\right) - \mu_0\left(X_{1}\right) + \beta\left(W_{1}, X_{1}\right)\E\left[\varepsilon_{i} \, \middle| \, \right]\right) s \E_{2:s}\left[\kappa(x; Z_1, D)\right]\right] \\
			%
			& = \E_{1}\left[\left(\mu_1\left(X_{1}\right) - \mu_0\left(X_{1}\right)\right) s \E_{2:s}\left[\kappa(x; Z_1, D)\right]\right] \\
			%
			& \overset{\text{(Lem~\ref{lem:dem13})}}{\longrightarrow} \mu_1\left(x\right) - \mu_0\left(x\right)
			\quad \text{as} \quad s \rightarrow \infty
		\end{aligned}
	\end{equation}
	Similarly, we can find the following.
	\begin{equation}
		\begin{aligned}
			& \E_{1}\left[m^2\left(Z_{i}; \mu, \pi\right) s \E_{2:s}\left[\kappa(x; Z_1, D)\right]\right]
			= \E_{1}\left[\left(\mu_1\left(X_{1}\right) - \mu_0\left(X_{1}\right) + \beta\left(W_{1}, X_{1}\right)\varepsilon_{i}\right)^2 s \E_{2:s}\left[\kappa(x; Z_1, D)\right]\right] \\
			%
			& \quad = \E_{1}\left[\left(\mu_1\left(X_{i}\right) - \mu_0\left(X_{1}\right)\right)^2 s \E_{2:s}\left[\kappa(x; Z_1, D)\right]\right]
			+ \underbrace{\E_{1}\left[\left(\mu_1\left(X_{i}\right) - \mu_0\left(X_{1}\right)\right)\beta\left(W_{1}, X_{1}\right)\E\left[\varepsilon_{1} \, \middle| \, X_1\right] s \E_{2:s}\left[\kappa(x; Z_1, D)\right]\right]}_{=0} \\
			& \quad \quad + \E_{1}\left[\left(\beta\left(W_{1}, X_{1}\right)\varepsilon_{i}\right)^2 s \E_{2:s}\left[\kappa(x; Z_1, D)\right]\right]\\
			%
			& \quad =  \E_{1}\left[\left(\mu_1\left(X_{1}\right) - \mu_0\left(X_{1}\right)\right)^2 s \E_{2:s}\left[\kappa(x; Z_1, D)\right]\right]
			+ \E_{1}\left[\left(\frac{W_{1}}{\pi\left(X_1\right)} - \frac{1 - W_{1}}{1 - \pi\left(X_1\right)}\right)^2 \varepsilon_{1}^2 s \E_{2:s}\left[\kappa(x; Z_1, D)\right]\right]\\
			%
			& \quad = \underbrace{\E_{1}\left[\left(\mu_1\left(X_{1}\right) - \mu_0\left(X_{1}\right)\right)^2 s \E_{2:s}\left[\kappa(x; Z_1, D)\right]\right]}_{\longrightarrow \left(\mu_1\left(x\right) - \mu_0\left(x\right)\right)^2 \quad \text{as} \quad s \rightarrow \infty}
			+ \underbrace{\E_{1}\left[\E\left[\left(\frac{W_{1}}{\pi\left(X_1\right)} - \frac{1 - W_{1}}{1 - \pi\left(X_1\right)}\right)^2 \varepsilon_{1}^2 \, \middle| \, X_1\right] s \E_{2:s}\left[\kappa(x; Z_1, D)\right]\right]}_{(B)}
		\end{aligned}
	\end{equation}
	Continuing with the second term, marked by $(B)$, we find the following.
	\begin{equation}
		\begin{aligned}
			(B) 
			& = \E_{1}\left[\E\left[\left(\frac{W_{1}}{\pi\left(X_1\right)} - \frac{1 - W_{1}}{1 - \pi\left(X_1\right)}\right)^2 \varepsilon_{1}^2 \, \middle| \, X_1\right] s \E_{2:s}\left[\kappa(x; Z_1, D)\right]\right]\\
			%
			& = \E_{1}\left[\frac{\sigma_{\varepsilon}^2(X_1) \cdot s \E_{2:s}\left[\kappa(x; Z_1, D)\right]}{\pi^2\left(X_1\right)\left(1 - \pi\left(X_1\right)\right)^2} \cdot 
			\E\left[\left(W_{1}\left(1 - \pi\left(X_1\right)\right) - \left(1 - W_{1}\right)\pi\left(X_1\right)\right)^2 \, \middle| \, X_1\right] \right]\\
		\end{aligned}
	\end{equation}
	Observe that $W_1(1-W_1) = 0$, $W_1^2 = W_1$, and $(1-W_1)^2 = 1 - W_1$, which allows us to use the following simplification.
	\begin{equation}
		\begin{aligned}
			(B) 
			& = \E_{1}\left[\frac{\sigma_{\varepsilon}^2(X_1) \cdot s \E_{2:s}\left[\kappa(x; Z_1, D)\right]}{\pi^2\left(X_1\right)\left(1 - \pi\left(X_1\right)\right)^2} \cdot 
			\E\left[W_{1}\left(1 - \pi\left(X_1\right)\right)^2 + \left(1 - W_{1}\right)\pi^2\left(X_1\right) \, \middle| \, X_1\right] \right]\\
			%
			& = \E_{1}\left[\frac{\sigma_{\varepsilon}^2(X_1) \cdot s \E_{2:s}\left[\kappa(x; Z_1, D)\right]}{\pi^2\left(X_1\right)\left(1 - \pi\left(X_1\right)\right)^2} \cdot 
			\left(\pi(X_1)\left(1 - \pi\left(X_1\right)\right)^2 + (1 - \pi(X_1))\pi^2\left(X_1\right) \right)\right]\\
			%
			& = \E_{1}\left[\frac{\sigma_{\varepsilon}^2(X_1) \cdot s \E_{2:s}\left[\kappa(x; Z_1, D)\right]}{\pi\left(X_1\right)\left(1 - \pi\left(X_1\right)\right)}\right]
			\overset{\text{(Lem~\ref{lem:dem13})}}{\longrightarrow} \frac{\sigma_{\varepsilon}^2(x)}{\pi\left(x\right)\left(1 - \pi\left(x\right)\right)}
			\quad \text{as} \quad s \rightarrow \infty
		\end{aligned}
	\end{equation}
	Recombining the terms of interest, we find the desired limit bound.
	\begin{equation}
		\begin{aligned}
			\E_{1}\left[m^2\left(Z_{i}; \mu, \pi\right) s \E_{2:s}\left[\kappa(x; Z_1, D)\right]\right]
			\overset{\text{(Lem~\ref{lem:dem13})}}{\longrightarrow} \left(\mu_1\left(x\right) - \mu_0\left(x\right)\right)^2 + \frac{\sigma_{\varepsilon}^2(x)}{\pi\left(x\right)\left(1 - \pi\left(x\right)\right)}
			\quad \text{as} \quad s \rightarrow \infty
		\end{aligned}
	\end{equation}
\end{proof}

\hrule

\begin{lem}\label{lem:kern_ineq}\mbox{}\\*
	Fix sample size $n$, subsampling scale $s$, and $c$ such that $0 < c \leq s \leq n$.
	Let $D = \left\{Z_1, Z_2, \dotsc, Z_c, Z_{c+1}, \dotsc Z_s \right\}$ be an i.i.d.\ data set drawn from $P$ as described in Setup~\ref{asm:npr_dgp}.
	Let $D^{\prime} = \left\{Z_1, Z_2, \dotsc, Z_c, Z_{c+1}^{\prime}, \dotsc Z_s^{\prime} \right\}$ be a second data set that shares the first $c$ observations with $D$.
	The remaining $s - c$ observations of $D^{\prime}$, i.e.\ $\left\{Z_{c+1}^{\prime}, \dotsc Z_s^{\prime} \right\}$, are i.i.d.\ draws from $P$ that are independent of $D$.

	Then, the following inequality holds.
	\begin{equation}
		\begin{aligned}
			\E_{D, D^{\prime}}\left[Y_{1}Y_{c+1}^{\prime} \, c(s-c) \, \kappa\left(x; Z_{1}, D\right)\kappa\left(x; Z_{c+1}^{\prime}, D^{\prime}\right)\right]
			& \leq {\color{red} LOREM IPSUM}
		\end{aligned}
	\end{equation}

	Similarly, in the CATE estimation setting (Setup~\ref{asm:hte_dgp}), i.e.\ replacing observations drawn from $P$ by observations drawn from $Q$, the following inequality holds.
	\begin{equation}
		\begin{aligned}
			\E_{D, D^{\prime}}\left[m\left(Z_{1}; \mu, \pi\right) m\left(Z_{c+1}^{\prime}; \mu, \pi\right) \, c(s-c) \, \kappa\left(x; Z_{1}, D\right)\kappa\left(x; Z_{c+1}^{\prime}, D^{\prime}\right)\right]
			& \leq {\color{red} LOREM IPSUM}
		\end{aligned}
	\end{equation}
\end{lem}

\hrule

\begin{proof}[Proof of Lemma~\ref{lem:kern_ineq}]\mbox{}\\*
	Consider first the following argument.
	\begin{equation}
		\begin{aligned}
			& \E_{D, D^{\prime}}\left[Y_{1}Y_{c+1}^{\prime} \, c(s-c) \, \kappa\left(x; Z_{1}, D\right)\kappa\left(x; Z_{c+1}^{\prime}, D^{\prime}\right)\right]
			= {\color{red} LOREM IPSUM}
		\end{aligned}
	\end{equation}
	{\color{red} LOREM IPSUM}
\end{proof}

\hrule

\begin{lem}[\citet{peng_bias_2021} - Lemma 1]\label{lem:peng1}\mbox{}\\*
	Suppose that $\sum X_{i}^2 \xrightarrow{p} 1, \sum \E\left[X_{i}^2\right] \rightarrow 1$, and $\sum_{i=1}^n \E\left[Y_{i}^2\right] \rightarrow 0$, then
	\begin{equation}
		\sum\left[X_{i}+Y_{i}\right]^2 \xrightarrow{p} 1 \quad \text { and } \E\left[\sum\left(X_{i}+Y_{i}\right)^2\right] \rightarrow 1.
	\end{equation}
\end{lem}

\hrule

\begin{lem}[Honesty of the DNN/TDNN Estimators]\label{lem:honesty}\mbox{}\\*
	The DNN and TDNN estimator kernels $\kappa\left(\cdot, \cdot, D_{\ell}\right)$ are Honest in the sense of \citet{wager_estimation_2018}.
	\begin{equation*}
		\kappa\left(x, X_{i}, D_{\ell}\right) \indep Y_{i} \mid X_{i}, D_{\ell,-i},
	\end{equation*}
	where $\indep$ denotes conditional independence and $D_{\ell,-i} = \{Z_l \, | \, l \in \ell \backslash \{i\}\}$.
\end{lem}

\newpage
\section{Proofs for Results in Section~\ref{sec:TDNN}}
\hrule

\newpage
\section{Proofs for Results in Section~\ref{sec:pw_inf}}
\hrule

\subsection{Closed Form Representations}
\hrule

\begin{proof}[Proof of Theorem~\ref{thm:JK_closed_form}]\mbox{}\\*
	Recall the closed form representation of the DNN estimator as presented in Equation~\ref{eq:DNN_closed_form} and its asymptotic approximation in Equation~\ref{eq:DNN_approx_closed_form}.
	\begin{equation}
		\tilde{\mu}_{s}(x; \mathbf{D}_n)
		= \binom{n}{s}^{-1} \sum_{i = 1}^{n - s + 1}\binom{n - i}{s - 1}Y_{(i)}
		\approx \sum_{i = 1}^{n - s + 1} \alpha_{s} {\left(1 - \alpha_{s}\right)}^{i - 1} Y_{(i)}
	\end{equation}
	Plugging into the Jackknife variance estimator for the DNN estimator now gives us the following where we assume that $n$ is sufficiently large for $n - s + 1$ to be larger than $s$.
	\begin{equation}
		\begin{aligned}
			\hat{\omega}_{\text{JK}}^{2}
			 & = \frac{n - 1}{n}\sum_{i = 1}^{n} {\left(\tilde{\mu}_{s}(x; \mathbf{D}_{n, -i}) - \tilde{\mu}_{s}(x; \mathbf{D}_n)\right)}^2 \\
			%
			 & =
		\end{aligned}
	\end{equation}
	Even more simple, we can use the approximate weights to find the following representation.
	For this purpose recall that $\alpha_{s} = s/n$ and define $\tilde{\alpha}_s = s/(n-1)$.
	Thus, $\tilde{\alpha}_s = \frac{n}{n-1}\alpha_{s}$.
	\begin{equation}
		\begin{aligned}
			\hat{\omega}_{\text{JK}}^{2}
			 & = \frac{n - 1}{n}\sum_{i = 1}^{n} {\left(\tilde{\mu}_{s}(x; \mathbf{D}_{n, -i}) - \tilde{\mu}_{s}(x; \mathbf{D}_n)\right)}^2 \\
			 %
			 & \approx \frac{n - 1}{n} \left[
				\sum_{i = 1}^{n - s + 1} \left(
				\sum_{j = 1}^{i - 1}\left(\tilde{\alpha}_{s}{\left(1 - \tilde{\alpha}_{s}\right)}^{j - 1} - \alpha_{s}{\left(1 - \alpha_{s}\right)}^{j - 1}
				\right) Y_{(j)} \right. \right.\\
				& \quad \quad \quad \quad \quad +  \left.\sum_{j = i + 1}^{n - s + 2}\left(\tilde{\alpha}_{s}{\left(1 - \tilde{\alpha}_{s}\right)}^{j - 1} - \alpha_{s}{\left(1 - \alpha_{s}\right)}^{j}
				\right) Y_{(j)}
				-  \alpha_{s}{\left(1 - \alpha_{s}\right)}^{i - 1} Y_{(i)}
				\right)^2\\
				& \quad \quad \left. + \sum_{i = n - s + 2}^{n} {\left(\sum_{j = 1}^{n - s + 1} 
					\left(\tilde{\alpha}_{s}{\left(1 - \tilde{\alpha}_{s}\right)}^{j - 1} - \alpha_{s}{\left(1 - \alpha_{s}\right)}^{j - 1}
					\right) Y_{(j)}\right)}^2
				\right]\\
				%
				& = \alpha_{s}^2 \cdot \frac{n - 1}{n} \left[
					\sum_{i = 1}^{n - s + 1} \left(
					\sum_{j = 1}^{i - 1}\left(\frac{n}{n-1}{\left(\frac{n - 1 - s}{n - 1}\right)}^{j - 1} - {\left(\frac{n - s}{n}\right)}^{j - 1}
					\right) Y_{(j)} \right. \right.\\
					& \quad \quad \quad \quad \quad +  \left.\sum_{j = i + 1}^{n - s + 2}\left(\frac{n}{n-1}{\left(\frac{n - 1 - s}{n - 1}\right)}^{j - 1} - {\left(\frac{n - s}{n}\right)}^{j}
					\right) Y_{(j)}
					-  {\left(\frac{n - s}{n}\right)}^{i - 1} Y_{(i)}
					\right)^2\\
					& \quad \quad \left. + \sum_{i = n - s + 2}^{n} {\left(\sum_{j = 1}^{n - s + 1} 
						\left(\frac{n}{n-1}{\left(\frac{n - 1 - s}{n - 1}\right)}^{j - 1} - {\left(\frac{n - s}{n}\right)}^{j - 1}
						\right) Y_{(j)}\right)}^2
					\right]
		\end{aligned}
	\end{equation}

	The closed form of the Jackknife variance estimator for the TDNN estimator follows from the same approach.

		{\color{red} LOREM IPSUM}
\end{proof}

\newpage
\subsection{NPR - Kernel (Conditional) Expectations}\label{subsec:KernelCondExp}
\hrule
As part of deriving consistency results for the variance estimators under consideration, we need to do a careful analysis of the Kernel of the DNN and TDNN estimators.
In this section of the appendix we will thus derive the expectations of the kernel and its corresponding H\'ajek projection.
First, we start with the nonparametric regression setup.
\vspace{0.5cm}
\hrule

\begin{lem}[NPR - DNN Kernel Expectation]\label{lem:DNN_k_exp}\mbox{}\\*
	Let $x$ denote a point of interest.
	Then
	\begin{equation}
		\E_D\left[h_s\left(x; D\right)\right]
		= \E_{1}\left[\mu\left(X_1\right) s{\left(1 - \psi\left(B\left(x, \|X_1 - x\|\right)\right)\right)}^{s-1}\right]\\
		\longrightarrow \mu(x) \quad \text{as} \quad s \rightarrow \infty
	\end{equation}
\end{lem}
\hrule
\begin{proof}[Proof of Lemma~\ref{lem:DNN_k_exp}]
	This result follows immediately from Lemma~\ref{lem:limit_res}.
\end{proof}

\hrule

\begin{lem}[NPR - DNN Haj\'ek Kernel Expectation]\label{lem:psi_s_1}\mbox{}\\*
	Let $z_1 = (x_1, y_1)$ denote a specific realization of $Z$ and $x$ denote a point of interest.
	Then
	\begin{equation}
		\psi_{s}^{1}\left(x; z_1\right)
		= \varepsilon_1 \E_D\left[\kappa\left(x; Z_1, D\right)\, \Big| \, X_1 = x_1 \right]
		+ \E_{D}\left[\sum_{i = 1}^{s} \kappa\left(x; Z_{i}, D\right) \mu(X_{i})\, \Big| \, X_1 = x_1 \right]
	\end{equation}
\end{lem}
\hrule
\begin{proof}[Proof of Lemma~\ref{lem:psi_s_1}]
	\begin{equation}
		\begin{aligned}
			\psi_{s}^{1}\left(x; z_1\right)
			 & = \E_{D}\left[h_{s}\left(x; D\right) \, | \, Z_1 = z_1 \right]
			= \E_{D}\left[\sum_{i = 1}^{s} \kappa\left(x; Z_{i}, D\right) Y_{i} \, \Big| \, Z_1 = z_1 \right] \\
			%
			 & = \E_{D}\left[\left(\mu(x_1) + \varepsilon_1\right)\kappa\left(x; Z_1, D\right)
			+ \sum_{i = 2}^{s} \kappa\left(x; Z_{i}, D\right) \mu(X_{i})\, \Big| \, Z_1 = z_1 \right]         \\
			%
			 & = \varepsilon_1 \E_D\left[\kappa\left(x; Z_1, D\right)\, \Big| \, X_1 = x_1 \right]
			+ \E_{D}\left[\sum_{i = 1}^{s} \kappa\left(x; Z_{i}, D\right) \mu(X_{i})\, \Big| \, X_1 = x_1 \right]
		\end{aligned}
	\end{equation}
\end{proof}


\newpage
\subsection{CATE - Kernel (Conditional) Expectations}\label{sec:CATE_exp}
\hrule
Next, we address the CATE estimation setup, where we first consider the scenario where the nuisance parameters are assumed to be known a priori.
In a second step, we will show that asymptotically, the estimation of nuisance parameters as described in Definition \ref{def:CATE_DNN_DML}, does not alter the asymptotic analysis of the estimator.
For clarity, we point out that in contexts relating to the estimation of the conditional average treatment effect, the kernel or score function $h_s$ could hypothetically signify the first or second stage kernel.
As the first stage is effectively covered by the nonparametric regression setup, we will take $h_s$ in these contexts to mean the kernel weighted Neyman-orthogonal score associated with the CATE.
\vspace{0.5cm}
\hrule

\begin{boxD}
    \begin{lem}[CATE - DNN Kernel Expectation]\label{lem:CATE_DNN_k_exp}\mbox{}\\*
	Let $x$ denote a point of interest.
	Then
	\begin{equation}
		\begin{aligned}
			\E_D\left[h_s\left(x; D\right)\right]
			& = \E_{1}\left[m(Z_1, \eta_0)
			s\E_{2:s}\left[\kappa(x; Z_1, D)\right]\right]\\
			%
			& \longrightarrow \theta_{0}(x) \quad \text{as} \quad s \rightarrow \infty
		\end{aligned}
	\end{equation}
\end{lem}
\end{boxD}

\begin{proof}[Proof of Lemma~\ref{lem:CATE_DNN_k_exp}]
	This result follows immediately from Lemma~\ref{lem:dem13} and the following observation.
	\begin{equation}
		\begin{aligned}
			\E_{1}\left[m\left(Z_{1}; \eta_{0}\right) s \E_{2:s}\left[\kappa(x; Z_1, D)\right]\right]
			 & = \E_{1}\left[\left(\mu_{0}^{1}\left(X_{1}\right) - \mu_{0}^{0}\left(X_{1}\right) + \beta\left(W_{1}, X_{1}\right)\varepsilon_{i}\right) s \E_{2:s}\left[\kappa(x; Z_1, D)\right]\right]                                \\
			%
			 & = \E_{1}\left[\left(\mu_{0}^{1}\left(X_{1}\right) - \mu_{0}^{0}\left(X_{1}\right) + \beta\left(W_{1}, X_{1}\right)\E\left[\varepsilon_1 \, \middle| \, X_1\right]\right) s \E_{2:s}\left[\kappa(x; Z_1, D)\right]\right] \\
			%
			 & = \E_{1}\left[\left(\mu_{0}^{1}\left(X_{1}\right) - \mu_{0}^{0}\left(X_{1}\right)\right) s \E_{2:s}\left[\kappa(x; Z_1, D)\right]\right]                                                                                \\
			%
			 & \overset{\text{(Lem~\ref{lem:dem13})}}{\longrightarrow} \mu_{0}^{1}\left(x\right) - \mu_{0}^{0}\left(x\right)
             = \theta_{0}(x)
			\quad \text{as} \quad s \rightarrow \infty
		\end{aligned}
	\end{equation}
\end{proof}

\begin{boxD}
    \begin{lem}[CATE - DNN Haj\'ek Kernel Expectation]\label{lem:CATE_chi_s_1}\mbox{}\\*
	Let $z_1 = (x_1, W_{1}, y_1)$ denote a specific realization of $Z$ and $x$ denote a point of interest.
	Then
	\begin{equation}
		\psi_{s}^{1}\left(x; z_1\right)
		= \beta\left(W_{1}, X_1\right)\varepsilon_{1} \cdot \E\left[\kappa\left(x; Z_1, D\right) \; \middle| \; X_1 = x_1\right]
		+ \E_{D}\left[\sum_{i = 2}^{s} \kappa\left(x; Z_{i}, D\right) \left(\mu_{0}^{1}\left(X_{i}\right) - \mu_{0}^{0}\left(X_{i}\right)\right)
		\, \Big| \, Z_1 = z_1 \right]
	\end{equation}
\end{lem}
\end{boxD}

\begin{proof}[Proof of Lemma~\ref{lem:CATE_chi_s_1}]
	\begin{equation}
		\begin{aligned}
			\psi_{s}^{1}\left(x; z_1\right)
			 & = \E_{D}\left[\chi_{s,0}\left(x; D\right) \, | \, Z_1 = z_1 \right] 
			  = \E_{D}\left[\sum_{i = 1}^{s} \kappa\left(x; Z_{i}, D\right) m(Z_i, \eta_0)
			 \, \Big| \, Z_1 = z_1 \right]\\
			 %
			 & = \left(\mu_{0}^{1}\left(X_1\right) - \mu_{0}^{0}\left(X_1\right) + \beta\left(W_{1}, X_1\right)\varepsilon_{1}\right)\E\left[\kappa\left(x; Z_1, D\right) \; \middle| \; X_1 = x_1\right]\\
			 & \quad + \E_{D}\left[\sum_{i = 2}^{s} \kappa\left(x; Z_{i}, D\right) \left(\mu_{0}^{1}\left(X_{i}\right) - \mu_{0}^{0}\left(X_{i}\right)\right)
			 \, \Big| \, Z_1 = z_1 \right]\\
			 %
			 & = \beta\left(W_{1}, X_1\right)\varepsilon_{1} \cdot \E\left[\kappa\left(x; Z_1, D\right) \; \middle| \; X_1 = x_1\right]
			 + \E_{D}\left[\sum_{i = 2}^{s} \kappa\left(x; Z_{i}, D\right) \left(\mu_{0}^{1}\left(X_{i}\right) - \mu_{0}^{0}\left(X_{i}\right)\right)
			 \, \Big| \, Z_1 = z_1 \right]
		\end{aligned}
	\end{equation}
\end{proof}

% \begin{lem}[TDNN Haj\'ek Kernel Expectation]\label{lem:psi_s1s2_1}\mbox{}\\*
% 	Let $z_1 = (x_1, y_1)$ denote a specific realization of $Z$ and $x$ denote a point of interest.
% 	Let $D = \left\{Z_1, \dotsc, Z_{s_2} \right\}$ and $D^{\prime} = \left\{Z_1^{\prime}, \dotsc, Z_{s_1}^{\prime} \right\}$ denote two independent and i.i.d.\ samples drawn from $P$.
% 	Furthermore, let $X \sim P$ and $X \indep D,D^{\prime}$.
% 	Then
% 	\begin{equation}
% 		\begin{aligned}
% 			\psi_{s_1, s_2}^{1}\left(x; z_1\right)
% 			 & = w_{1}^{*} \left(
% 			\frac{s_2}{s_1}\left(\varepsilon_1 \E_{D^{\prime}}\left[\kappa\left(x; Z_1^{\prime}, D^{\prime}\right)\, \Big| \, X_1^{\prime} = x_1 \right]
% 				+ \E_{D^{\prime}}\left[\sum_{i = 1}^{s} \kappa\left(x; Z_{i}^{\prime}, D^{\prime}\right) \mu(X_{i}^{\prime})\, \Big| \, X_1 ^{\prime}= x_1 \right]\right)
% 			+ \frac{s_2}{s_2 - s_1}\E\left[\mu(X)\right]\right)                                                                                          \\
% 			 & \quad + w_{2}^{*} \left(\varepsilon_1 \E_D\left[\kappa\left(x; Z_1, D\right)\, \Big| \, X_1 = x_1 \right]
% 			+ \E_{D}\left[\sum_{i = 1}^{s} \kappa\left(x; Z_{i}, D\right) \mu(X_{i})\, \Big| \, X_1 = x_1 \right]\right) \\
% 		\end{aligned}
% 	\end{equation}
% \end{lem}

% \begin{proof}[Proof of Lemma~\ref{lem:psi_s1s2_1}]
% 	\begin{equation}
% 		\begin{aligned}
% 			\psi_{s_1, s_2}^{1}\left(x; z_1\right)
% 			 & = \E_{D}\left[w_{1}^{*} \tilde{\mu}_{s_1}\left(x; D\right)
% 				+ w_{2}^{*} h_{s_2}\left(x; D\right)\, | \, Z_1 = z_1\right]
% 			= \E_{D}\left[h_{s_1, s_2}\left(x; D\right) \, | \, Z_1 = z_1 \right]                                                             \\
% 			%
% 			 & = w_{1}^{*} \E_{D}\left[\tilde{\mu}_{s_1}\left(x; D\right)\, | \, Z_1 = z_1\right]
% 			+ w_{2}^{*} \E_D\left[h_{s_2}\left(x; D\right)\, | \, Z_1 = z_1\right]                                                            \\
% 			%
% 			 & = w_{1}^{*} \E_{D}\left[\binom{s_2}{s_1}^{-1}\sum_{\ell \in L_{s_2, s_1}}h_{s_1}\left(x; D_\ell\right)\, | \, Z_1 = z_1\right]
% 			+ w_{2}^{*} \E_D\left[h_{s_2}\left(x; D\right)\, | \, Z_1 = z_1\right]                                                            \\
% 			%
% 			 & = w_{1}^{*} \binom{s_2}{s_1}^{-1}\left(\E_{D}\left[
% 				\sum_{\ell \in L_{s_1 - 1}\left([s_2]\backslash \{1\}\right)}h_{s_1}\left(x; D_{\ell \cup 1}\right)
% 				\sum_{\ell \in L_{s_1}\left([s_2]\backslash \{1\}\right)}h_{s_1}\left(x; D_\ell\right)
% 				\, | \, Z_1 = z_1\right]
% 			\right)                                                                                                                                                      \\
% 			 & \quad + w_{2}^{*} \E_D\left[h_{s_2}\left(x; D\right)\, | \, Z_1 = z_1\right]                                                   \\
% 			%
% 			 & = w_{1}^{*} \binom{s_2}{s_1}^{-1}\left(
% 			\binom{s_2-1}{s_1-1}\E_{1:s_1}\left[h_{s_1}\left(x; D_{[s_1]}\right) \, | \, Z_1 = z_1 \right]
% 			+ \binom{s_2-1}{s_1}\E_{2:(s_1+1)}\left[h_{s_1}\left(x; D_{2:(s_1+1)}\right)\right]
% 			\right)                                                                                                                                                      \\
% 			 & \quad + w_{2}^{*} \E_D\left[h_{s_2}\left(x; D\right)\, | \, Z_1 = z_1\right]                                                   \\
% 			%
% 			 & = w_{1}^{*} \binom{s_2}{s_1}^{-1}\left(
% 			\binom{s_2-1}{s_1-1}\psi_{s_1}^{1}\left(x; \mathbf{z_1}\right)
% 			+ \binom{s_2-1}{s_1}\E_{2:(s_1+1)}\left[h_{s_1}\left(x; D_{2:(s_1+1)}\right)\right]
% 			\right) + w_{2}^{*} \psi_{s_2}^{1}\left(x; \mathbf{z_1}\right)
% 		\end{aligned}
% 	\end{equation}
% 	Using Lemmas~\ref{lem:DNN_k_exp} and \ref{lem:psi_s_1}, we can further simplify this term significantly.
% 	\begin{equation}
% 		\begin{aligned}
% 			\psi_{s_1, s_2}^{1}\left(x; z_1\right)
% 			 & = w_{1}^{*} \left(\frac{s_2}{s_1}\psi_{s_1}^{1}\left(x; \mathbf{z_1}\right)
% 			+ \frac{s_2}{s_2 - s_1}\E\left[\mu(X)\right]\right)
% 			+ w_{2}^{*} \psi_{s_2}^{1}\left(x; \mathbf{z_1}\right)                                                                                       \\
% 			%
% 			 & = w_{1}^{*} \left(
% 			\frac{s_2}{s_1}\left(\varepsilon_1 \E_{D^{\prime}}\left[\kappa\left(x; Z_1, D^{\prime}\right)\, \Big| \, X_1 = x_1 \right]
% 				+ \E_{D^{\prime}}\left[\sum_{i = 1}^{s} \kappa\left(x; Z_{i}, D^{\prime}\right) \mu(X_{i})\, \Big| \, X_1 = x_1 \right]\right)
% 			+ \frac{s_2}{s_2 - s_1}\E\left[\mu(X)\right]\right)                                                                                          \\
% 			 & \quad + w_{2}^{*} \left(\varepsilon_1 \E_D\left[\kappa\left(x; Z_1, D\right)\, \Big| \, X_1 = x_1 \right]
% 			+ \E_{D}\left[\sum_{i = 1}^{s} \kappa\left(x; Z_{i}, D\right) \mu(X_{i})\, \Big| \, X_1 = x_1 \right]\right) \\
% 			%
% 			 & \quad = {\color{red} LOREM IPSUM}
% 		\end{aligned}
% 	\end{equation}
% \end{proof}


\newpage
\subsection{NPR - Kernel Variances \& Covariances}
\hrule
Similar to the previous section of proofs, we will continue by analyzing the variances and covariances of the kernels under consideration.
These results will play an important role in the derivation of consistency properties for the variance estimators.
Similar to the previous part, we will first consider the nonparametric regression setup and then proceed to the conditional average treatment effect setup.
\vspace{0.5cm}
\hrule
\begin{lem}[Adapted from \citet{demirkaya_optimal_2024}]\label{lem:omega_s}\mbox{}\\*
	Let $D = \{Z_1, \dotsc, Z_{s}\}$ be a vector of i.i.d.\ random variables drawn from $P$.
	Furthermore, let
	\begin{equation}
		\Omega_{s}\left(x\right)
		= \E\left[h_{s}^{2}\left(x; Z_1, \ldots,  Z_{s}\right)\right].
	\end{equation}
	Then,
	\begin{equation}
		\Omega_{s}\left(x\right)
		= \E_{1}\left[\left(\mu\left(X_1\right)+ \varepsilon_1\right)^2 s \E_{2:s}\left[\kappa\left(x; Z_1, D\right)\right]\right]
		\lesssim \mu^2(x) + \overline{\sigma}_{\varepsilon}^2 + o(1)
		\quad \text{as} \quad s \rightarrow \infty.
	\end{equation}
\end{lem}
\hrule
\begin{proof}[Proof of Lemma~\ref{lem:omega_s}]\mbox{}\\*
    This result follows immediately from Lemma~\ref{lem:dem13} and the following observation.
	\begin{equation}
		\begin{aligned}
			\Omega_{s}\left(x\right)
			 & = \E\left[h_{s}^{2}\left(x; Z_1, \ldots,  Z_{s}\right)\right]
			= \E_{D}\left[\left(\sum_{i = 1}^{s}\kappa\left(x; Z_{i}, D\right)Y_{i}\right)^2\right]
			= \E_{D}\left[\sum_{i = 1}^{s}\sum_{j = 1}^{s}\left(\kappa\left(x; Z_{i}, D\right)\kappa\left(x; Z_{j}, D\right)Y_{i}Y_{j}\right)\right] \\
			%
			 & = \E_{D}\left[s \kappa\left(x; Z_{1}, D\right)Y_{1}^2\right]
			= \E_{1}\left[Y_{1}^2 s \E_{2:s}\left[\kappa\left(x; Z_{1}, D\right)\right]\right]       
            = \E_{1}\left[\left(\mu\left(X_1\right) + \varepsilon_1\right)^2 s \E_{2:s}\left[\kappa(x; Z_1, D)\right]\right] \\
			%
			 & = \E_{1}\left[\left(\mu^2\left(X_1\right) + 2\mu\left(X_1\right)\varepsilon_1 + \varepsilon_1^2\right)s \E_{2:s}\left[\kappa(x; Z_1, D)\right]\right]                       \\
			%
			 & = \E_{1}\left[\left(\mu^2\left(X_1\right) + 2\mu\left(X_1\right) \E\left[\varepsilon_1 \, \middle| \, X_1\right] + \E\left[\varepsilon_1^2 \, \middle| \, X_1\right]\right)
			s \E_{2:s}\left[\kappa(x; Z_1, D)\right]\right]                                                                                                                                \\
			%
			 & = \E_{1}\left[\left(\mu^2\left(X_1\right) +\sigma_{\varepsilon}^{2}(X_1)\right) s \E_{2:s}\left[\kappa(x; Z_1, D)\right]\right]              \overset{\text{(Lem~\ref{lem:dem13})}}{\longrightarrow} \mu^2\left(x\right) +\sigma_{\varepsilon}^{2}(x)
			\quad \text{as} \quad s \rightarrow \infty
		\end{aligned}
	\end{equation}
    Furthermore, we have the following inequality.
    \begin{equation}
        \mu^2(x) + \sigma_{\varepsilon}^2(x) \leq \mu^2\left(x\right) + \overline{\sigma}_{\varepsilon}^{2}
    \end{equation}
	Thus, we obtain the desired result.
\end{proof}

\hrule

\begin{lem}\label{lem:omega_sc}\mbox{}\\*
	Let $D = \{Z_1, \dotsc, Z_{s}\}$ be a vector of i.i.d.\ random variables drawn from $P$.
	Let $D^{\prime} = \{Z_1, \dotsc, Z_{c}, Z_{c+1}^{\prime}, \dotsc,  Z_{s}^{\prime}\}$ where $Z_{c+1}^{\prime}, \dotsc,  Z_{s}^{\prime}$ are i.i.d.\ draws from $P$ that are independent of $D$.
	Furthermore, let
	\begin{equation}
		\Omega_{s}^{c}\left(x\right)
		= \E\left[h_{s}\left(x; Z_1, \ldots, Z_{c}, Z_{c+1}, \ldots, Z_{s}\right) \cdot
			h_{s}\left(x; Z_1, \ldots,Z_{c}, Z_{c+1}^{\prime}, \ldots, Z_{s}^{\prime}\right)\right].
	\end{equation}
	Then,
	\begin{equation}
		\Omega_{s}^{c}\left(x\right)
		\lesssim \frac{s^2 + cs  - c^2}{s^2} \mu^2(x) + (c/s) \overline{\sigma}_{\varepsilon}^2 + o(1)
		\quad \text{for} \quad s \quad \text{sufficiently large}
	\end{equation}
	and thus
	\begin{equation}
		\Omega_{s}^{c}\left(x\right)
		\lesssim \mu^2(x) + \overline{\sigma}_{\varepsilon}^2 + o(1)
		\quad \text{as} \quad s \rightarrow \infty.
	\end{equation}
\end{lem}
\hrule
\begin{proof}[Proof of Lemma~\ref{lem:omega_sc}]
	\begin{equation}
		\begin{aligned}
			\Omega_{s}^{c}\left(x\right)
			 & = \E\left[h_{s}\left(x; Z_1, \ldots, Z_{c}, Z_{c+1}, \ldots, Z_{s}\right) \cdot
			h_{s}\left(x; Z_1, \ldots,Z_{c}, Z_{c+1}^{\prime}, \ldots, Z_{s}^{\prime}\right)\right]                                                                \\
			%
			 & = \E_{D, D^{\prime}}\left[
				\left(\sum_{i = 1}^{s}\kappa\left(x; Z_{i}, D\right)Y_{i}\right)
				\left(\sum_{j = 1}^{c}\kappa\left(x; Z_{j}, D^{\prime}\right)Y_{j}
				+ \sum_{j = c+1}^{s}\kappa\left(x; Z_{j}^{\prime}, D^{\prime}\right)Y_{j}^{\prime}\right)
			\right]                                                                                                                                                                                             \\
			%
			 & = \E_{D, D^{\prime}}\left[\sum_{i = 1}^{c}\sum_{j = 1}^{c}\kappa\left(x; Z_{i}, D\right)\kappa\left(x; Z_{j}, D^{\prime}\right)Y_{i}Y_{j}\right]
			+  \E_{D, D^{\prime}}\left[\sum_{i = 1}^{c}\sum_{j = c+1}^{s}\kappa\left(x; Z_{i}, D\right)\kappa\left(x; Z_{j}^{\prime}, D^{\prime}\right)Y_{i}Y_{j}^{\prime}\right]   \\
			 & \quad + \E_{D, D^{\prime}}\left[\sum_{i = c+1}^{s}\sum_{j = 1}^{c}\kappa\left(x; Z_{i}, D\right)\kappa\left(x; Z_{j}, D^{\prime}\right)Y_{i}Y_{j}\right]
			+  \E_{D, D^{\prime}}\left[\sum_{i = c+1}^{s}\sum_{j = c+1}^{s}\kappa\left(x; Z_{i}, D\right)\kappa\left(x; Z_{j}^{\prime}, D^{\prime}\right)Y_{i}Y_{j}^{\prime}\right] \\
			%
			 & = \underbrace{\E_{D, D^{\prime}}\left[c \kappa\left(x; Z_{1}, D\right)\kappa\left(x; Z_{1}, D^{\prime}\right)Y_{1}^{2}\right]}_{(A)}
			+ \underbrace{\E_{D, D^{\prime}}\left[c(s-c) \kappa\left(x; Z_{1}, D\right)\kappa\left(x; Z_{c+1}^{\prime}, D^{\prime}\right)Y_{1}Y_{c+1}^{\prime}\right]}_{(B)}                           \\
			 & \quad + \underbrace{\E_{D, D^{\prime}}\left[c(s-c) \kappa\left(x; Z_{c+1}, D\right)\kappa\left(x; Z_{1}, D^{\prime}\right)Y_{c+1}Y_{1}\right]}_{(C)}                                    \\
			 & \quad + \underbrace{\E_{D, D^{\prime}}\left[(s-c)^2 \kappa\left(x; Z_{c+1}, D\right)\kappa\left(x; Z_{c+1}^{\prime}, D^{\prime}\right)Y_{c+1}Y_{c+1}^{\prime}\right]}_{(D)}
		\end{aligned}
	\end{equation}
	Starting from this decomposition, we will analyze the terms one by one.
	First, by Lemma~\ref{lem:limit_res}, we find the following.
	\begin{equation}
		\begin{aligned}
            (A) & = 
			\E_{D, D^{\prime}}\left[c\kappa\left(x; Z_{1}, D\right)\kappa\left(x; Z_{1}, D^{\prime}\right)Y_{1}^{2}\right]
			= (c/s) \E_{1}\left[Y_1^2 s \E_{2:s}\left[\kappa\left(x; Z_{1}, D\right)\kappa\left(x; Z_{1}, D^{\prime}\right)\right]\right] \\
			%
			 & \leq (c/s) \E_{1}\left[Y_1^2 s\E_{2:s}\left[\kappa\left(x; Z_{1}, D\right)\right]\right]                  
			\overset{\text{(Lem~\ref{lem:limit_res})}}{\lesssim} (c/s)\left(\mu^2(x) + \sigma_{\varepsilon}(x)\right) + o(1)\\
		\end{aligned}
	\end{equation}
	Similarly, we can find that:
	\begin{equation}
		\begin{aligned}
            (B)
			 & = \E_{D, D^{\prime}}\left[c(s-c) \kappa\left(x; Z_{1}, D\right)\kappa\left(x; Z_{c+1}^{\prime}, D^{\prime}\right)Y_{1}Y_{c+1}^{\prime}\right]                                          
			\overset{\text{Lem \ref{lem:npr_kern_ineq1}}}{\lesssim} \frac{c(s-c)}{s^2}\mu^2(x) + o(1)
		\end{aligned}
	\end{equation}
	Following analogous steps, we find the same result for the third term.
	\begin{equation}
		\begin{aligned}
			(C)
            = \E_{D, D^{\prime}}\left[c(s-c) \kappa\left(x; Z_{c+1}, D\right)\kappa\left(x; Z_{1}, D^{\prime}\right)Y_{c+1}Y_{1}\right]
			\overset{\text{Lem \ref{lem:npr_kern_ineq1}}}{\lesssim} \frac{c(s-c)}{s^2}\mu^2(x) + o(1)
		\end{aligned}
	\end{equation}
	The fourth term can be asymptotically bounded in the following way.
	\begin{equation}
		\begin{aligned}
			(D)
            & = \E_{D, D^{\prime}}\left[(s-c)^2 \kappa\left(x; Z_{c+1}, D\right)\kappa\left(x; Z_{c+1}^{\prime}, D^{\prime}\right)Y_{c+1}Y_{c+1}^{\prime}\right]                                           \\
			%
			 & = \E_{D, D^{\prime}}\left[\mu(X_{c+1})\mu(X_{c+1}^{\prime})\, (s-c)^2 \, \kappa\left(x; Z_{c+1}, D\right)\kappa\left(x; Z_{c+1}^{\prime}, D^{\prime}\right)\right] \\
			%
			 & \leq \E_{D}\left[|\mu(X_{c+1})|\, (s-c) \, \kappa\left(x; Z_{c+1}, D\right)\right]
			\E_{D^{\prime}}\left[|\mu(X_{c+1}^{\prime})|\, (s-c) \, \kappa\left(x; Z_{c+1}^{\prime}, D^{\prime}\right)\right]                                                                                      \\
			%
			 & = \frac{(s-c)^2}{s^2} \E_{D}\left[|\mu(X_{c+1})|\, s \, \kappa\left(x; Z_{c+1}, D\right)\right]
			\E_{D^{\prime}}\left[|\mu(X_{c+1}^{\prime})|\, s \, \kappa\left(x; Z_{c+1}^{\prime}, D^{\prime}\right)\right]                                                                                          \\
			%
			 & \quad = \frac{(s-c)^2}{s^2} \left(\E_{D}\left[|\mu(X_{c+1})|\, s \, \kappa\left(x; Z_{c+1}, D\right)\right]\right)^2                                                                        
			\lesssim \frac{(s-c)^2}{s^2}\mu^2(x) + o(1)
		\end{aligned}
	\end{equation}
	The result of Lemma~\ref{lem:omega_sc} follows immediately by summing up the asymptotic bounds for the individual terms.
\end{proof}

\newpage
\begin{lem}\label{lem:upsilon_s}\mbox{}\\*
	Let $D = \{Z_1, \dotsc, Z_{s_2}\}$ be a vector of i.i.d.\ random variables drawn from $P$ for $s_2 > s_1$.
	Furthermore, let
	\begin{equation}
		\Upsilon_{s_1, s_2}\left(x\right)
		= \E\left[h_{s_1}\left(x; Z_1, \ldots,  Z_{s_1}\right) \cdot
			h_{s_2}\left(x; Z_1, \ldots,Z_{s_1}, \ldots, Z_{s_2}\right)\right].
	\end{equation}
	Then,
	\begin{equation}
		\Upsilon_{s_1, s_2}\left(x\right)
		\lesssim \mu^{2}\left(x\right) + \overline{\sigma}^2_{\varepsilon} + o(1)
		\quad \text{as} \quad s_1, s_2 \rightarrow \infty
		\quad \text{with} \quad
		0 < \mathfrak{c} \leq s_1 / s_2 \leq 1 - \mathfrak{c} < 1.
	\end{equation}
\end{lem}
\hrule
\begin{proof}[Proof of Lemma~\ref{lem:upsilon_s}]
	\begin{equation}
		\begin{aligned}
			\Upsilon_{s_1, s_2}\left(x\right)
			 & = \E\left[h_{s_1}\left(x; Z_1, \ldots,  Z_{s_1}\right) \cdot
			h_{s_2}\left(x; Z_1, \ldots,Z_{s_1}, \ldots, Z_{s_2}\right)\right]                                                                             \\
			%
			 & = \E_{D}\left[
				\left(\sum_{i = 1}^{s_1} \kappa(x; Z_{i}, D_{[s_1]})Y_{i}\right)
				\left(\sum_{j = 1}^{s_1}\kappa(x; Z_{j}, D)Y_j + \sum_{j = s_1 + 1}^{s_2}\kappa(x; Z_{j}, D)Y_j\right)
			\right]                                                                                                                                                                            \\
			%
			 & = \E_{D}\left[\sum_{i = 1}^{s_1} \kappa(x; Z_{i}, D) Y_{i}^2\right]
			+ \E_{D}\left[\sum_{i = 1}^{s_1}\sum_{j = s_1 + 1}^{s_2}\kappa(x; Z_{i}, D_{[s_1]})\kappa(x; Z_{j}, D) Y_{i} Y_j\right]                              \\
			%
			 & = \E_{D}\left[Y_1^2 \, s_1 \, \kappa(x; Z_1, D)\right]
			+ \E_{D}\left[Y_{1} Y_{s_2} \, s_1 (s_2 - s_1) \, \kappa(x; Z_1, D_{[s_1]})\kappa(x; Z_{s_2}, D)\right]                                        \\
			%
			 & = \E_{D}\left[\left(\mu^2(X_1) + \sigma^2_{\varepsilon}(X_1)\right) \, s_1 \, \kappa(x; Z_1, D)\right]
			+ \E_{D}\left[\mu(X_1) \mu(X_{s_2}) \, s_1 (s_2 - s_1) \, \kappa(x; Z_1, D_{[s_1]})\kappa(x; Z_{s_2}, D)\right]              \\
			%
			 & = \frac{s_1}{s_2}\E_{D}\left[\left(\mu^2(X_1) + \sigma^2_{\varepsilon}(X_1)\right) \, s_1 \, \kappa(x; Z_1, D)\right]
			+ \frac{s_2 - s_1}{s_2}\E_{D}\left[\mu(X_1) \mu(X_{s_2}) \, s_1 s_2 \, \kappa(x; Z_1, D_{[s_1]})\kappa(x; Z_{s_2}, D)\right] \\
			%
			 & \leq \frac{s_1}{s_2} \E_{D}\left[\left(\mu^2(X_1) + \sigma^2_{\varepsilon}(X_1)\right) \, s_2 \, \kappa(x; Z_1, D)\right]                   \\
			 & \quad \quad + \frac{s_2 - s_1}{s_2}\E_{D}\left[|\mu(X_1)| \, s_1 \, \kappa(x; Z_1, D_{[s_1]})\right]
			\E_{D}\left[|\mu(X_{s_2})| \, s_2 \, \kappa(x; Z_{s_2}, D)\right]                                                                                       \\
			%
			 & \lesssim \mu^{2}\left(x\right) + \sigma^2_{\varepsilon}(x) + o(1)
			\leq \mu^{2}\left(x\right) + \overline{\sigma}^2_{\varepsilon} + o(1).
		\end{aligned}
	\end{equation}
\end{proof}

\newpage
\begin{lem}\label{lem:upsilon_sc}\mbox{}\\*
	Let $D = \{Z_1, \dotsc, Z_{s_2}\}$ be a vector of i.i.d.\ random variables drawn from $P$ for $s_2 > s_1$.
	Let $D^{\prime} = \{Z_1, \dotsc, Z_{c}, Z_{c+1}^{\prime}, \dotsc,  Z_{s_1}^{\prime}\}$ where $Z_{c+1}^{\prime}, \dotsc,  Z_{s_1}^{\prime}$ are i.i.d.\ draws from $P$ that are independent of $D$.
	Furthermore, let
	\begin{equation}
		\Upsilon_{s_1, s_2}^{c}\left(x\right)
		= \E\left[h_{s_1}\left(x; Z_1, \ldots, Z_c, Z^{\prime}_{c+1}, \ldots,  Z^{\prime}_{s_1}\right) \cdot
			h_{s_2}\left(x; Z_1, \ldots, Z_{s_2}\right)\right].
	\end{equation}
	Then,
	\begin{equation}
		\begin{aligned}
			 & \Upsilon_{s_1, s_2}^{c}\left(x\right)
			\lesssim \frac{c s_2 - c^2 + s_1 s_2}{s_1 s_2}\mu^2(x) + (c/s_1) \overline{\sigma}^2_{\varepsilon} + o(1) \\
			%
			 & \text{for} \quad s_1, s_2 \quad \text{sufficiently large}
			\quad \text{with} \quad
			0 < \mathfrak{c} \leq s_1 / s_2 \leq 1 - \mathfrak{c} < 1
		\end{aligned}
	\end{equation}
	and thus
	\begin{equation}
		\Upsilon_{s_1, s_2}^{c}\left(x\right)
		\lesssim \mu^2(x) + o(1)
		\quad \text{as} \quad s_1, s_2 \rightarrow \infty
		\quad \text{with} \quad
		0 < \mathfrak{c} \leq s_1 / s_2 \leq 1 - \mathfrak{c} < 1.
	\end{equation}
\end{lem}
\hrule
\begin{proof}[Proof of Lemma~\ref{lem:upsilon_sc}]
	\begin{equation}
		\begin{aligned}
			\Upsilon_{s_1, s_2}^{c}\left(x\right)
			 & = \E\left[h_{s_1}\left(x; Z_1, \ldots, Z_c, Z^{\prime}_{c+1}, \ldots,  Z^{\prime}_{s_1}\right) \cdot
			h_{s_2}\left(x; Z_1, \ldots, Z_{s_2}\right)\right]                                                                                                                       \\
			%                                                                     
			 & = \E_{D, D^{\prime}}\left[
				\left(\sum_{i = 1}^{c} \kappa(x; Z_{i}, D^{\prime})Y_{i} + \sum_{i = c+1}^{s_1} \kappa(x; Z_{i}^{\prime}, D^{\prime})Y_{i}^{\prime}\right)
				\left(\sum_{j = 1}^{c} \kappa(x; Z_{j}, D)Y_j + \sum_{j = c+1}^{s_2}\kappa(x; Z_{j}, D)Y_j \right)
			\right]                                                                                                                                                                                             \\
			%
			 & = \underbrace{\E_{D, D^{\prime}}\left[\sum_{i = 1}^{c}\sum_{j = 1}^{c} \kappa(x; Z_{i}, D^{\prime})\kappa(x; Z_{j}, D)Y_{i} Y_j\right]}_{(A)}
			+ \underbrace{\E_{D, D^{\prime}}\left[\left(\sum_{i = 1}^{c}\kappa(x; Z_{i}, D^{\prime}) Y_{i}\right) \left(\sum_{j = c+1}^{s_2} \kappa(x; Z_{j}, D) Y_j\right)\right]}_{(B)} \\
			 & \quad + \underbrace{\E_{D, D^{\prime}}\left[\sum_{i = c+1}^{s_1}\sum_{j = 1}^{c} \kappa(x; Z_{i}^{\prime}, D^{\prime})\kappa(x; Z_{j}, D)Y_{i}^{\prime} Y_j\right]}_{(C)}
			+ \underbrace{\E_{D, D^{\prime}}\left[\left(\sum_{i = c+1}^{s_1}\kappa(x; Z_{i}^{\prime}, D^{\prime}) Y_{i}^{\prime}\right)
					\left(\sum_{j = c+1}^{s_2} \kappa(x; Z_{j}, D) Y_j\right)\right]}_{(D)}
		\end{aligned}
	\end{equation}
	Again, we have four terms to analyze individually.
	\begin{equation}
		\begin{aligned}
			(A)
			 & = \E_{D, D^{\prime}}\left[\sum_{i = 1}^{c}\sum_{j = 1}^{c} \kappa(x; Z_{i}, D^{\prime})\kappa(x; Z_{j}, D)Y_{i} Y_j\right]\\
			 %
			& = \E_{D, D^{\prime}}\left[\sum_{i = 1}^{c} Y_{i}^2 \kappa(x; Z_{i}, D^{\prime})\kappa(x; Z_{i}, D)\right]                                                                       \\
			%
			 & = \E_{D, D^{\prime}}\left[Y_{1}^2 \, c \, \kappa(x; Z_1, D^{\prime})\kappa(x; Z_1, D)\right]
			= \E_{D, D^{\prime}}\left[\left(\mu^2(X_{1}) + \sigma^2_{\varepsilon}(X_{1})\right) \, c \, \kappa(x; Z_1, D_{[c]})\kappa(x; Z_1, D^{\prime}_{c+1:s_1})\right] \\
			%
			 & = \E_{D}\left[\left(\mu^2(X_{1}) + \sigma^2_{\varepsilon}(X_{1})\right) \, c \, \kappa(x; Z_1, D)\right]
			= \frac{c}{s_1} \E_{D}\left[\left(\mu^2(X_{1}) + \sigma^2_{\varepsilon}(X_{1})\right) \, s_1 \, \kappa(x; Z_1, D)\right]                                               \\
			%
			 & \lesssim (c/s_1)(\mu^2(x) + \sigma^2_{\varepsilon}(x)) + o(1)
			\leq (c/s_1)(\mu^2(x) + \overline{\sigma}^2_{\varepsilon}) + o(1)
		\end{aligned}
	\end{equation}
	Considering the second term, we find the following.
	\begin{equation}
		\begin{aligned}
			(B)
			 & = \E_{D, D^{\prime}}\left[\left(\sum_{i = 1}^{c}\kappa(x; Z_{i}, D^{\prime}) Y_{i}\right)\left(\sum_{j = c+1}^{s_2} \kappa(x; Z_{j}, D) Y_j\right)\right]
			= \E_{D, D^{\prime}}\left[\sum_{i = 1}^{c}\sum_{j = c+1}^{s_2}Y_{i} Y_j \kappa(x; Z_{i}, D^{\prime})\kappa(x; Z_{j}, D)\right]                               \\
			%
			 & = \E_{D, D^{\prime}}\left[c(s_2 - c) \, Y_1 Y_{s_1} \kappa(x; Z_1, D^{\prime})\kappa(x; Z_{s_2}, D)\right]
			= \frac{c (s_2 - c)}{s_1 s_2}\E_{D, D^{\prime}}\left[Y_1 Y_{s_2} \, s_1 s_2 \,\kappa(x; Z_1, D^{\prime})\kappa(x; Z_{s_2}, D)\right]                   \\
			%
			 & \leq \frac{c (s_2 - c)}{s_1 s_2}
			\E_{D^{\prime}}\left[|\mu(X_1)| \, s_1  \,\kappa(x; Z_1, D^{\prime})\right]
			\E_{D}\left[|\mu(X_{s_2})| \, s_2  \,\kappa(x; Z_{s_2}, D)\right]                                                                                       \\
			%
			 & \lesssim \frac{c(s_2 - c)}{s_1 s_2} \mu^2(x) + o(1)
		\end{aligned}
	\end{equation}
	Similarly, by simplifying the third term, we find the following.
	\begin{equation}
		\begin{aligned}
			(C)
			 & = \E_{D, D^{\prime}}\left[\sum_{i = c+1}^{s_1}\sum_{j = 1}^{c} \kappa(x; Z_{i}^{\prime}, D^{\prime})\kappa(x; Z_{j}, D)Y_{i}^{\prime} Y_j\right]
			= \E_{D, D^{\prime}}\left[Y_{s_1}^{\prime} Y_{1} \, (s_1 - c)c \, \kappa(x; Z_{s_1}^{\prime}, D^{\prime})\kappa(x; Z_1, D)\right]                                                    \\
			% 
			 & = \frac{(s_1 - c)c}{s_1 s_2}\E_{D, D^{\prime}}\left[\mu(X_{s_1}^{\prime})\mu(X_1) \, s_1 s_2 \, \kappa(x; Z_{s_1}^{\prime}, D^{\prime})\kappa(x; Z_1, D)\right] \\
			%
			 & \leq \frac{(s_1 - c)c}{s_1 s_2}
			\E_{D}\left[|\mu(X_{s_1}^{\prime})| \, s_1 \, \kappa(x; Z_{s_1}^{\prime}, D^{\prime})\right]
			\E_{D}\left[|\mu(X_1)| \, s_2 \, \kappa(x; Z_1, D)\right]                                                                                                                             \\
			%
			 & \lesssim \frac{(s_1 - c)c}{s_1 s_2} \mu^2(x) + o(1)
		\end{aligned}
	\end{equation}
	Lastly, concerning the fourth term, observe the following.
	\begin{equation}
		\begin{aligned}
			(D)
			 & = \E_{D, D^{\prime}}\left[\left(\sum_{i = c+1}^{s_1}\kappa(x; Z_{i}^{\prime}, D^{\prime}) Y_{i}^{\prime}\right)
				\left(\sum_{j = c+1}^{s_2} \kappa(x; Z_{j}, D) Y_j\right)\right]
			= \E_{D, D^{\prime}}\left[\sum_{i = c+1}^{s_1}\sum_{j = c+1}^{s_2}\kappa(x; Z_{i}^{\prime}, D^{\prime}) \kappa(x; Z_{j}, D)  Y_{i}^{\prime}Y_j\right]                                                        \\
			%
			 & = \E_{D, D^{\prime}}\left[\mu(X_{s_1}^{\prime}) \mu(X_{s_2}) \, (s_1 - c)(s_2 -c) \,\kappa(x; Z_{s_1}^{\prime}, D^{\prime}) \kappa(x; Z_{s_2}, D)  \right]                        \\                                                                                                                                                                       \\
			%
			 & = \frac{(s_1 - c)(s_2 -c)}{s_1 s_2}\E_{D, D^{\prime}}\left[\mu(X_{s_1}^{\prime}) \mu(X_{s_2}) \, s_1 s_2 \,\kappa(x; Z_{s_1}^{\prime}, D^{\prime}) \kappa(x; Z_{s_2}, D)  \right] \\                                                                                                                                                                       \\
			%
			 & \leq \frac{(s_1 - c)(s_2 -c)}{s_1 s_2}
			\E_{D^{\prime}}\left[|\mu(X_{s_1}^{\prime})| \, s_1 \,\kappa(x; Z_{s_1}^{\prime}, D^{\prime})   \right]
			\E_{D}\left[ |\mu(X_{s_2})| \, s_2 \, \kappa(x; Z_{s_2}, D)  \right]                                                                                                                                    \\
			%
			 & \lesssim \frac{(s_1 - c)(s_2 -c)}{s_1 s_2}\mu^2(x) + o(1)
		\end{aligned}
	\end{equation}
\end{proof}

\newpage
\begin{lem}[Kernel Variance of the TDNN Kernel]\label{lem:Var_TDNN_k}
	For the kernel of the TDNN estimator with subsampling scales $s_1$ and $s_2$, it holds that
	\begin{equation}
		\zeta_{s_1, s_2}^{s_2}\left(x\right)
		\lesssim \mu^2(x) + \overline{\sigma}_{\varepsilon} + o(1)
		\quad \text{as} \quad s_1, s_2 \rightarrow \infty
		\quad \text{with} \quad
		0 < \mathfrak{c} \leq s_1 / s_2 \leq 1 - \mathfrak{c} < 1.
	\end{equation}
\end{lem}
\hrule
\begin{proof}[Proof of Lemma~\ref{lem:Var_TDNN_k}]
	Consider first the following decomposition.
	\begin{equation}
		\begin{aligned}
			\zeta_{s_1, s_2}^{s_2}\left(x\right)
			 & = \Var\left(h_{s_1, s_2}\left(x; Z_1, \ldots, Z_{s_2}\right)\right)
			= \Var_{D}\left(h_{s_1, s_2}\left(x; D\right)\right)                                     \\
			%
			 & \leq \E_{D}\left[h_{s_1, s_2}^{2}\left(x; D\right)\right]
			= \E_{D}\left[
				\left(w_{1}^{*}\tilde{\mu}_{s_1}\left(x; D\right) + w_{2}^{*} h_{s_2}\left(x; D\right)\right)^2
			\right]                                                                                           \\
			%
			 & = \left(w_{1}^{*}\right)^2\E_{D}\left[\tilde{\mu}_{s_1}^{2}\left(x; D\right)\right]
			+ 2 w_{1}^{*}w_{2}^{*} \E_{D}\left[\tilde{\mu}_{s_1}\left(x; D\right) h_{s_2}\left(x; D\right)\right]
			+ \left(w_{2}^{*}\right)^2\Omega_{s_2}
		\end{aligned}
	\end{equation}

	Then, observe the following.
	\begin{equation}
		\begin{aligned}
			\E_{D}\left[\tilde{\mu}_{s_1}^{2}\left(x; D\right)\right]
			 & = \E_D\left[\left(\binom{s_2}{s_1}^{-1}\sum_{\ell \in L_{s_2, s_1}} h_{s_1}\left(x; D_{\ell}\right)\right)^2\right]
			= \binom{s_2}{s_1}^{-2} \E_{D}\left[\sum_{\iota, \iota' \in L_{s_2, s_1}}h_{s_1}\left(x; D_{\iota}\right)h_{s_1}\left(x; D_{\iota'}\right)\right] \\
			%
			 & = \binom{s_2}{s_1}^{-2} \sum_{c = 0}^{s_1} \binom{s_2}{s_1}\binom{s_1}{c}\binom{s_2 - s_1}{s_1 - c} \Omega_{s_1}^{c}
			= \binom{s_2}{s_1}^{-1} \sum_{c = 0}^{s_1} \binom{s_1}{c}\binom{s_2 - s_1}{s_1 - c} \Omega_{s_1}^{c}                                                                \\
			%
			 & \lesssim \Omega_{s_1}
			\lesssim \mu(x)^2 + \sigma_{\varepsilon}^2 + o(1)
			\quad \text{as} \quad s \rightarrow \infty
		\end{aligned}
	\end{equation}

	Recall that by Lemma~\ref{lem:omega_s}, we have the following.
	\begin{equation}
		\begin{aligned}
			\Omega_{s_2}
			\lesssim \mu(x)^2 + \sigma_{\varepsilon}^2 + o(1)
			\quad \text{as} \quad s \rightarrow \infty
		\end{aligned}
	\end{equation}

	Lastly, consider the following.
	\begin{equation}
		\begin{aligned}
			\E_{D}\left[\tilde{\mu}_{s_1}\left(x; D\right) h_{s_2}\left(x; D\right)\right]
			 & = \E_D\left[\binom{s_2}{s_1}^{-1}\sum_{\ell \in L_{s_2, s_1}} h_{s_1}\left(x; D_{\ell}\right)h_{s_2}\left(x; D\right)\right] \\
			%
			 & = \E_D\left[h_{s_1}\left(x; D_{[s_1]}\right)h_{s_2}\left(x; D\right)\right]
			= \Upsilon_{s_1, s_2}\left(x\right)
		\end{aligned}
	\end{equation}

	Thus, we find the following.
	\begin{equation}
		\begin{aligned}
			\zeta_{s_2, s_2}\left(x\right)
			 & \lesssim \left(w_{1}^{*}\right)^2 \Omega_{s_1}
			+ 2 w_{1}^{*}w_{2}^{*} \Upsilon_{s_1, s_2}\left(x\right)
			+ \left(w_{1}^{*}\right)^2 \Omega_{s_2}                                                                       \\
			%
			 & \lesssim \left(w_{1}^{*} + w_{2}^{*}\right)^2 \left(\mu^2(x) + \sigma_{\varepsilon}\right) + o(1)
			= \mu^2(x) + \sigma_{\varepsilon} + o(1).
		\end{aligned}
	\end{equation}
\end{proof}

\newpage

\begin{lem}[Lemma 10 - \citet{demirkaya_optimal_2024}]\label{lem:dem10}
	For the kernel of the TDNN estimator with subsampling scales $s_1$ and $s_2$ satisfying
	\begin{equation}
		0 < \mathfrak{c} \leq s_1 / s_2 \leq 1 - \mathfrak{c} < 1
		\quad \text{and} \quad
		s_2 = o(n),
	\end{equation}
	it holds that
	\begin{equation}
		\zeta_{s_1, s_2}^{1}\left(x\right)
		\sim s_2^{-1}.
	\end{equation}
\end{lem}

\newpage
\subsection{CATE - Kernel Variances \& Covariances}\label{sec:CATE_cov}
\hrule
Next, we will continue by showing analogous properties in the CATE setting.
Similar to before, we will start under the assumption that the functional nuisance parameters are known a priori, to then show that the estimation of said parameters does not impact the asymptotic behavior of the estimator.

\begin{boxD}
    \begin{lem}\label{lem:CATE_omega_s}\mbox{}\\*
	Let $D = \{Z_1, \dotsc, Z_{s}\}$ be a vector of i.i.d.\ random variables generated by the setup shown in Assumption~\ref{asm:CATE_dgp}.
	Furthermore, let
	\begin{equation}
		\Omega_{s}\left(x\right)
		= \E\left[\chi_{s,0}^{2}\left(x; Z_1, \ldots,  Z_{s}\right)\right].
	\end{equation}
	Then,
	\begin{equation}
		\Omega_{s}\left(x\right)
		\lesssim \left(\mu_{0}^{1}\left(x\right) - \mu_{0}^{0}\left(x\right)\right)^2 + \frac{\overline{\sigma}^2_{\varepsilon}}{\mathfrak{p}\left(1 - \mathfrak{p}\right)} + o(1)
	\end{equation}
\end{lem}
\end{boxD}

\begin{proof}[Proof of Lemma~\ref{lem:CATE_omega_s}]\mbox{}\\*
    First, notice that we can decompose the quantity of interest in the following way.
	\begin{equation}
		\begin{aligned}
			\Omega_{s}\left(x\right)
			& = \E\left[\chi_{s,0}^{2}\left(x; Z_1, \ldots,  Z_{s}\right)\right]
            = \E_{D}\left[\left(\sum_{i = 1}^{s} \kappa\left(x; Z_{i}, D\right) m\left(Z_{i}; \eta_{0}\right)\right)^2\right]\\
            %
            & = \E_{D}\left[\sum_{i = 1}^{s}\sum_{j = 1}^{s}\kappa\left(x; Z_{i}, D\right)\kappa\left(x; Z_{j}, D\right)
            m\left(Z_{i}; \eta_{0}\right)m\left(Z_{j}; \eta_{0}\right)\right]
            = \E_{D}\left[s \kappa\left(x; Z_1, D\right)  m^2\left(Z_{1}; \eta_{0}\right)\right] \\
            %
            & = \E_{1}\left[ m^2\left(Z_{1}; \eta_{0}\right) s \E_{2:s}\left[\kappa\left(x; Z_1, D\right)\right]\right]\\
            %
            & = \E_{1}\left[\left(\mu_{0}^{1}\left(X_{1}\right) - \mu_{0}^{0}\left(X_{1}\right) + \beta\left(W_{1}, X_{1}\right)\varepsilon_{1}\right)^2 s \E_{2:s}\left[\kappa(x; Z_1, D)\right]\right]                                                                                             \\
			%
			& = \E_{1}\left[\left(\mu_{0}^{1}\left(X_{i}\right) - \mu_{0}^{0}\left(X_{1}\right)\right)^2 s \E_{2:s}\left[\kappa(x; Z_1, D)\right]\right]
			+ \E_{1}\left[\left(\beta\left(W_{1}, X_{1}\right)\varepsilon_{1}\right)^2 s \E_{2:s}\left[\kappa(x; Z_1, D)\right]\right]\\
			%
		      & =  \E_{1}\left[\left(\mu_{0}^{1}\left(X_{1}\right) - \mu_{0}^{0}\left(X_{1}\right)\right)^2 s \E_{2:s}\left[\kappa(x; Z_1, D)\right]\right]
			+ \E_{1}\left[\left(\frac{W_{1}}{\pi_{0}\left(X_1\right)} - \frac{1 - W_{1}}{1 - \pi_{0}\left(X_1\right)}\right)^2 \varepsilon_{1}^2 s \E_{2:s}\left[\kappa(x; Z_1, D)\right]\right]                                                                                               \\
			%
			& = \underbrace{\E_{1}\left[\left(\mu_{0}^{1}\left(X_{1}\right) - \mu_{0}^{0}\left(X_{1}\right)\right)^2 s \E_{2:s}\left[\kappa(x; Z_1, D)\right]\right]}_{\overset{\text{Lem \ref{lem:dem13}}}{\longrightarrow} \left(\mu_{0}^{1}\left(x\right) - \mu_{0}^{0}\left(x\right)\right)^2 \quad \text{as} \quad s \rightarrow \infty}
			+ \underbrace{\E_{1}\left[\E\left[\left(\frac{W_{1}}{\pi_{0}\left(X_1\right)} - \frac{1 - W_{1}}{1 - \pi_{0}\left(X_1\right)}\right)^2 \varepsilon_{1}^2 \, \middle| \, X_1\right] s \E_{2:s}\left[\kappa(x; Z_1, D)\right]\right]}_{(B)}
		\end{aligned}
	\end{equation}
	Continuing with the second term, marked by $(B)$, we find the following.
	\begin{equation}
		\begin{aligned}
			(B)
			& = \E_{1}\left[\E\left[\left(\frac{W_{1}}{\pi_{0}\left(X_1\right)} - \frac{1 - W_{1}}{1 - \pi_{0}\left(X_1\right)}\right)^2 \varepsilon_{1}^2 \, \middle| \, X_1\right] s \E_{2:s}\left[\kappa(x; Z_1, D)\right]\right] \\
			%
			& = \E_{1}\left[\frac{\sigma_{\varepsilon}^2(X_1) \cdot s \E_{2:s}\left[\kappa(x; Z_1, D)\right]}{\pi_{0}^2\left(X_1\right)\left(1 - \pi_{0}\left(X_1\right)\right)^2} \cdot
			\E\left[\left(W_{1}\left(1 - \pi_{0}\left(X_1\right)\right) - \left(1 - W_{1}\right)\pi_{0}\left(X_1\right)\right)^2 \, \middle| \, X_1\right] \right]                                                                    \\
		\end{aligned}
	\end{equation}
	Observe that $W_1(1-W_1) = 0$, $W_1^2 = W_1$, and $(1-W_1)^2 = 1 - W_1$, which allows us to use the following simplification.
	\begin{equation}
		\begin{aligned}
			(B)
			& = \E_{1}\left[\frac{\sigma_{\varepsilon}^2(X_1) \cdot s \E_{2:s}\left[\kappa(x; Z_1, D)\right]}{\pi_{0}^2\left(X_1\right)\left(1 - \pi_{0}\left(X_1\right)\right)^2} \cdot
			\E\left[W_{1}\left(1 - \pi_{0}\left(X_1\right)\right)^2 + \left(1 - W_{1}\right)\pi_{0}^2\left(X_1\right) \, \middle| \, X_1\right] \right]\\
			%
			& = \E_{1}\left[\frac{\sigma_{\varepsilon}^2(X_1) \cdot s \E_{2:s}\left[\kappa(x; Z_1, D)\right]}{\pi_{0}^2\left(X_1\right)\left(1 - \pi_{0}\left(X_1\right)\right)^2} \cdot \pi_{0}(X_1)\left(1 - \pi_{0}\left(X_1\right)\right) \cdot
			\left(1 - \pi_{0}\left(X_1\right) + \pi_{0}\left(X_1\right)\right)\right]\\
			%
			& = \E_{1}\left[\frac{\sigma_{\varepsilon}^2(X_1) }{\pi_{0}\left(X_1\right)\left(1 - \pi_{0}\left(X_1\right)\right)}\cdot s \E_{2:s}\left[\kappa(x; Z_1, D)\right]\right]
			\overset{\text{(Lem~\ref{lem:dem13})}}{\longrightarrow} \frac{\sigma_{\varepsilon}^2(x)}{\pi_{0}\left(x\right)\left(1 - \pi_{0}\left(x\right)\right)}
			\quad \text{as} \quad s \rightarrow \infty
		\end{aligned}
	\end{equation}
	Recombining the terms of interest, we find the desired limit bound.
	\begin{equation}
		\begin{aligned}
			\E_{1}\left[m^2\left(Z_{i}; \eta_{0}\right) s \E_{2:s}\left[\kappa(x; Z_1, D)\right]\right]
			\overset{\text{(Lem~\ref{lem:dem13})}}{\longrightarrow} \left(\mu_{0}^{1}\left(x\right) - \mu_{0}^{0}\left(x\right)\right)^2 + \frac{\sigma_{\varepsilon}^2(x)}{\pi_{0}\left(x\right)\left(1 - \pi_{0}\left(x\right)\right)}
			\quad \text{as} \quad s \rightarrow \infty
		\end{aligned}
	\end{equation}
    This gives us the desired result.
    \begin{equation}
        \Omega_{s}\left(x\right)
        \lesssim \left(\mu_{0}^{1}\left(x\right) - \mu_{0}^{0}\left(x\right)\right)^2 + \frac{\overline{\sigma}^2_{\varepsilon}}{\mathfrak{p}\left(1 - \mathfrak{p}\right)} + o(1)
    \end{equation}
\end{proof}

\begin{boxD}
    \begin{lem}\label{lem:CATE_omega_sc}\mbox{}\\*
	Let $D = \{Z_1, \dotsc, Z_{s}\}$ be a vector of i.i.d.\ random variables drawn from as described in Setup~\ref{asm:CATE_dgp}.\\
	Let $D^{\prime} = \{Z_1, \dotsc, Z_{c}, Z_{c+1}^{\prime}, \dotsc,  Z_{s}^{\prime}\}$ where $Z_{c+1}^{\prime}, \dotsc,  Z_{s}^{\prime}$ are i.i.d.\ draws from the model that are independent of $D$.
	Furthermore, let
	\begin{equation}
		\Omega_{s}^{c}\left(x\right)
		= \E\left[\chi_{s,0}\left(x; Z_1, \ldots, Z_{c}, Z_{c+1}, \ldots, Z_{s}\right) \cdot
			\chi_{s,0}\left(x; Z_1, \ldots,Z_{c}, Z_{c+1}^{\prime}, \ldots, Z_{s}^{\prime}\right)\right].
	\end{equation}
	Then,
	\begin{equation}
		\Omega_{s}^{c}\left(x\right)
		\lesssim C \left[\left(\mu_{0}^{1}\left(x\right) - \mu_{0}^{0}\left(x\right)\right)^2
		+ \frac{\overline{\sigma}^2_{\varepsilon}}{\mathfrak{p}\left(1 - \mathfrak{p}\right)}\right] + o(1).
	\end{equation}
\end{lem}
\end{boxD}


\begin{proof}[Proof of Lemma~\ref{lem:CATE_omega_sc}]
    First, we decompose the term of interest in a similar fashion to before.
	\begin{equation}
        \begin{aligned}
            \Omega_{s}^{c}\left(x\right)
		    & = \E\left[\chi_{s,0}\left(x; Z_1, \ldots, Z_{c}, Z_{c+1}, \ldots, Z_{s}\right) \cdot
			\chi_{s,0}\left(x; Z_1, \ldots,Z_{c}, Z_{c+1}^{\prime}, \ldots, Z_{s}^{\prime}\right)\right]\\
            %
            & = \E_{D, D^{\prime}}\left[
                \left(\sum_{i = 1}^{s} \kappa\left(x; Z_{i}, D\right) m\left(Z_{i}; \eta_{0}\right)\right) 
                \left(\sum_{j = 1}^{c} \kappa\left(x; Z_{j}, D^{\prime}\right) m\left(Z_{j}; \eta_{0}\right) + \sum_{j = c + 1}^{s} \kappa\left(x; Z_{j}^{\prime}, D^{\prime}\right) m\left(Z_{j}^{\prime}; \eta_{0}\right)\right)
            \right]\\
            %
            & = \underbrace{\E_{D, D^{\prime}}\left[\left(\sum_{i = 1}^{c} \kappa\left(x; Z_{i}, D\right) m\left(Z_{i}; \eta_{0}\right)\right)
            \left(\sum_{j = 1}^{c} \kappa\left(x; Z_{j}, D^{\prime}\right) m\left(Z_{j}; \eta_{0}\right)\right)\right]}_{(A)} \\
            & \quad + 2\underbrace{\E_{D, D^{\prime}}\left[\left(\sum_{i = 1}^{c} \kappa\left(x; Z_{i}, D\right) m\left(Z_{i}; \eta_{0}\right)\right)
            \left(\sum_{j = c + 1}^{s} \kappa\left(x; Z_{j}^{\prime}, D^{\prime}\right) m\left(Z_{j}^{\prime}; \eta_{0}\right)\right)\right]}_{(B)} \\
            % & \quad + \underbrace{\E_{D, D^{\prime}}\left[\left(\sum_{i = c + 1}^{s} \kappa\left(x; Z_{i}, D\right) m\left(Z_{i}; \eta_{0}\right)\right)
            % \left(\sum_{j = 1}^{c} \kappa\left(x; Z_{j}, D^{\prime}\right) m\left(Z_{j}; \eta_{0}\right)\right)\right]}_{(C)} \\
            & \quad + \underbrace{\E_{D, D^{\prime}}\left[\left(\sum_{i = c + 1}^{s} \kappa\left(x; Z_{i}, D\right) m\left(Z_{i}; \eta_{0}\right)\right)
            \left(\sum_{j = c + 1}^{s} \kappa\left(x; Z_{j}^{\prime}, D^{\prime}\right) m\left(Z_{j}^{\prime}; \eta_{0}\right)\right)\right]}_{(C)} 
        \end{aligned}
    \end{equation}
    Considering these terms one by one, we can make the following observations.
    Here, we rely on the same argument structure as in the proof of Lemma \ref{lem:omega_sc} and observations from the proof of Lemma \ref{lem:CATE_omega_s}.
    \begin{equation}
        \begin{aligned}
            (A)
            & = \E_{D, D^{\prime}}\left[\left(\sum_{i = 1}^{c} \kappa\left(x; Z_{i}, D\right) m\left(Z_{i}; \eta_{0}\right)\right)
            \left(\sum_{j = 1}^{c} \kappa\left(x; Z_{j}, D^{\prime}\right) m\left(Z_{j}; \eta_{0}\right)\right)\right] \\
			%
			& = \E_{D, D^{\prime}}\left[
				\sum_{i = 1}^{c} \sum_{j = 1}^{c} \kappa\left(x; Z_{i}, D\right)\kappa\left(x; Z_{j}, D^{\prime}\right) m\left(Z_{i}; \eta_{0}\right)m\left(Z_{j}; \eta_{0}\right)
			\right] \\
            %
			& = \frac{c}{2s-c} \cdot \E_{1}\left[m^{2}\left(Z_{1}; \eta_{0}\right) \cdot (2s-c) \cdot \E_{2:s}\left[\kappa\left(x; Z_{1}, D\right)\kappa\left(x; Z_{1}, D^{\prime}\right)\right]\right]\\
			%
            & \overset{\text{Lem \ref{lem:kernel_prod_dirac_convergence}}}{\lesssim} \frac{c}{2s-c} \cdot \left(\left(\mu_{0}^{1}\left(x\right) - \mu_{0}^{0}\left(x\right)\right)^2 + \frac{\sigma_{\varepsilon}^2(x)}{\pi_{0}\left(x\right)\left(1 - \pi_{0}\left(x\right)\right)}\right) + o(1)\\
            %
            & \leq \frac{c}{2s-c} \cdot \left(\left(\mu_{0}^{1}\left(x\right) - \mu_{0}^{0}\left(x\right)\right)^2 + \frac{\overline{\sigma}^2_{\varepsilon}}{\mathfrak{p}\left(1 - \mathfrak{p}\right)}\right) + o(1)
        \end{aligned}
    \end{equation}
	Similarly, for the second term, we can make the following observation.
	\begin{equation}
		\begin{aligned}
			(B) 
			& = \E_{D, D^{\prime}}\left[\left(\sum_{i = 1}^{c} \kappa\left(x; Z_{i}, D\right) m\left(Z_{i}; \eta_{0}\right)\right)
            \left(\sum_{j = c + 1}^{s} \kappa\left(x; Z_{j}^{\prime}, D^{\prime}\right) m\left(Z_{j}^{\prime}; \eta_{0}\right)\right)\right] \\
			%
			& = \E_{D,D^{\prime}}\left[\sum_{i = 1}^{c} \sum_{j = c + 1}^{s}\kappa\left(x; Z_{i}, D\right)\kappa\left(x; Z_{j}^{\prime}, D^{\prime}\right) 
			m\left(Z_{i}; \eta_{0}\right)m\left(Z_{j}^{\prime}; \eta_{0}\right)\right]\\
			%
			& = \E_{D, D^{\prime}}\left[c (s-c) \kappa\left(x; Z_{1}, D\right)\kappa\left(x; Z_{c+1}^{\prime}, D^{\prime}\right) 
			m\left(Z_{1}; \eta_{0}\right) m\left(Z_{c+1}^{\prime}; \eta_{0}\right)\right]\\
			%
            & \overset{\text{Lem \ref{lem:expec_kernel_prod_bound}}}{\leq} \frac{c(s-c)s}{(2s - c)(2s-c-1)(c+1)} \cdot \E_{1, (c+1)^{\prime}}\left[m\left(Z_{1}; \eta_{0}\right) m\left(Z_{c+1}^{\prime}; \eta_{0}\right)\cdot \frac{\E_{D, D^{\prime}}\left[\kappa\left(x; Z_{1}, D\right)\kappa\left(x; Z_{c+1}^{\prime}, D^{\prime}\right) \; \middle| \; Z_{1}, Z_{c+1}^{\prime}\right]}{\E_{D, D^{\prime}}\left[\kappa\left(x; Z_{1}, D\right)\kappa\left(x; Z_{c+1}^{\prime}, D^{\prime}\right)\right]}\right]\\
            %
			& \overset{\text{Lem \ref{lem:kernel_prod_dirac_convergence}}}{\lesssim} 
            \frac{c(s-c)s}{(2s - c)(2s-c-1)(c+1)} \cdot \left(\mu_{0}^{1}\left(x\right) - \mu_{0}^{0}\left(x\right)\right)^2 + o(1)
		\end{aligned}
	\end{equation}
	% Applying the same principles to the third term we find a similar result.
	% \begin{equation}
	% 	\begin{aligned}
	% 		(C)
	% 		& = \E_{D, D^{\prime}}\left[\left(\sum_{i = c + 1}^{s} \kappa\left(x; Z_{i}, D\right) m\left(Z_{i}; \eta_{0}\right)\right)
 %            \left(\sum_{j = 1}^{c} \kappa\left(x; Z_{j}, D^{\prime}\right) m\left(Z_{j}; \eta_{0}\right)\right)\right]\\
	% 		%
	% 		& \lesssim \frac{c(s-c)}{s^2}\left(\mu_{0}^{1}\left(x\right) - \mu_{0}^{0}\left(x\right)\right)^2  + o(1)
	% 	\end{aligned}
	% \end{equation}
    Finally, for the third term, we can make the following observation.
	\begin{equation}
		\begin{aligned}
			(C)
			& = \E_{D, D^{\prime}}\left[\left(\sum_{i = c + 1}^{s} \kappa\left(x; Z_{i}, D\right) m\left(Z_{i}; \eta_{0}\right)\right)
            \left(\sum_{j = c + 1}^{s} \kappa\left(x; Z_{j}^{\prime}, D^{\prime}\right) m\left(Z_{j}^{\prime}; \eta_{0}\right)\right)\right] \\
			%
			& = \E_{D, D^{\prime}}\left[(s-c)^2 \kappa\left(x; Z_{c+1}, D\right)\kappa\left(x; Z_{c+1}^{\prime}, D^{\prime}\right)
			m\left(Z_{c+1}; \eta_{0}\right)  m\left(Z_{c+1}^{\prime}; \eta_{0}\right)\right] \\
			%
			& \overset{\text{Lem \ref{lem:expec_kernel_prod_bound}}}{\leq}
            \frac{2(s-c)^{3}}{(2s-c)^{2}(2s-c-1)} \cdot \E_{c+1}\left[m\left(Z_{c+1}; \eta_{0}\right)  m\left(Z_{c+1}^{\prime}; \eta_{0}\right)
            \cdot \frac{\E_{D, D^{\prime}}\left[\kappa\left(x; Z_{c+1}, D\right)\kappa\left(x; Z_{c+1}^{\prime}, D^{\prime}\right) \; \middle| \; Z_{c+1}, Z_{c+1}^{\prime}\right]}{\E_{D, D^{\prime}}\left[\kappa\left(x; Z_{c+1}, D\right)\kappa\left(x; Z_{c+1}^{\prime}, D^{\prime}\right)\right]}\right]\\
            %
            & \overset{\text{Lem \ref{lem:kernel_prod_dirac_convergence}}}{\lesssim} 
            \frac{2(s-c)}{(2s-c-1)} \cdot \left(\mu_{0}^{1}\left(x\right) - \mu_{0}^{0}\left(x\right)\right)^2 + o(1)
		\end{aligned}
	\end{equation}
	Thus, we find the desired result.
	\begin{equation}
		\begin{aligned}
			\Omega_{s}^{c}\left(x\right)
			& = (A) + 2\cdot (B) + (C) \\
			%
			& \lesssim 
            \left(\frac{c}{2s-c} + \frac{4(s-c)}{2s-c-1} + \frac{2(s-c)}{2s-c-1}\right) \cdot \left(\mu_{0}^{1}\left(x\right) - \mu_{0}^{0}\left(x\right)\right)^2
            + \frac{c}{2s-c} \cdot \frac{\overline{\sigma}^2_{\varepsilon}}{\mathfrak{p}\left(1 - \mathfrak{p}\right)}
            + o(1) \\
            %
            & \leq \frac{6s-5c}{2s-c-1} \cdot \left(\mu_{0}^{1}\left(x\right) - \mu_{0}^{0}\left(x\right)\right)^2
            + \frac{c}{2s-c} \cdot \frac{\overline{\sigma}^2_{\varepsilon}}{\mathfrak{p}\left(1 - \mathfrak{p}\right)}
            + o(1) 
		\end{aligned}
	\end{equation}
\end{proof}

\newpage 

\begin{boxD}
    \begin{lem}\label{lem:CATE_zeta_s1_ub}\mbox{}\\*
        {\color{red} LOREM IPSUM}
    \end{lem}
\end{boxD}

\begin{proof}[Proof of Lemma \ref{lem:CATE_zeta_s1_ub}]\mbox{}\\*
     \begin{equation}
         \begin{aligned}
            \zeta_{s}^{1}(x)
            & = \Var_{Z}\left(\E_{D}\left[\chi_{s,0}\left(x; \mathbf{D}\right) \, \middle| \, Z_{1} = Z\right]\right)
            = \Var_{Z}\left(\E_{D}\left[
                \sum_{i = 1}^{s}\kappa(x; Z_{i}, \mathbf{D}_{[s]}) 
                m(Z_{i}, \eta_{0})
            \, \middle| \, Z_{1} = Z\right]\right) \\
            %
            & = \Var_{Z}\Bigg(
                \E_{2:s}\left[\kappa(x; Z, \mathbf{D}_{[s]}) \cdot \beta\left(W, X\right) \cdot \varepsilon\right]
                + \E_{D}\left[\sum_{i = 1}^{s}\kappa(x; Z_{i}, \mathbf{D}_{[s]})m(Z_{i}, \eta_{0})
                \, \middle| \, X_{1} = X\right]
            \Bigg) \\
            %
            & = \Var_{Z}\Bigg(
                \E_{2:s}\left[\kappa(x; Z, \mathbf{D}_{[s]}) \cdot \beta\left(W, X\right) \cdot \varepsilon\right]\Bigg) 
                + \Var_{X}\Bigg(\E_{D}\left[\sum_{i = 1}^{s}\kappa(x; Z_{2}, \mathbf{D}_{[s]})m(Z_{i}, \eta_{0})
                \, \middle| \, X_{1} = X\right]
            \Bigg) \\
            %
            & \geq \Var_{Z}\Bigg(\E_{2:s}\left[\kappa(x; Z, \mathbf{D}_{[s]}) \cdot \beta\left(W, X\right) \cdot \varepsilon\right]\Bigg) 
            = \Var_{Z}\Bigg(\E_{2:s}\left[\kappa(x; Z, \mathbf{D}_{[s]})\right] \cdot \beta\left(W, X\right) \cdot \varepsilon \Bigg)\\
            %
            & = \E_{Z}\left[\left(\E_{2:s}\left[\kappa(x; Z, \mathbf{D}_{[s]})\right] \cdot \beta\left(W, X\right) \cdot \varepsilon\right)^2\right]
            - \left(\E_{Z}\left[\E_{2:s}\left[\kappa(x; Z, \mathbf{D}_{[s]})\right] \cdot \beta\left(W, X\right) \cdot \varepsilon\right]\right)^{2} \\
            %
            & = \E_{Z}\left[\left(\E_{2:s}\left[\kappa(x; Z, \mathbf{D}_{[s]})\right] \cdot \beta\left(W, X\right) \cdot \varepsilon\right)^2\right]
            = \E_{Z}\left[\left(\E_{2:s}\left[\kappa(x; Z, \mathbf{D}_{[s]})\right] \cdot \beta\left(W, X\right)\right)^2 \cdot \E\left[\varepsilon^{2} \, \middle| \, X\right]\right]\\
            %
            & = \E_{Z}\left[\E_{2:(2s-1)}\left[\kappa(x; Z, \mathbf{D}_{[2s - 1]})\right] \cdot \beta^{2}\left(W, X\right) \cdot \sigma^{2}(X)\right]\\
            %
            & = \frac{1}{2s} \cdot \E_{Z}\left[\beta^{2}\left(W, X\right) \cdot \sigma^{2}(X) \cdot 2s\E_{2:(2s-1)}\left[\kappa(x; Z, \mathbf{D}_{[2s - 1]})\right]\right]\\
            %
            & \gtrsim \frac{1}{2s} \cdot \left(\pi(x) \beta^{2}\left(1 , x\right) + \left(1-\pi(x)\right) \beta^{2}\left(0 , x\right) \right)\cdot \sigma^{2}(x) - o(s^{-1}) \quad 
            \text{for} \quad s \quad \text{sufficiently large}
         \end{aligned}
     \end{equation}
    {\color{red} Last step needs to be justified still. LOREM IPSUM}
\end{proof}

% \newpage
% \begin{lem}\label{lem:CATE_upsilon_s}\mbox{}\\*
% 	Let $D = \{Z_1, \dotsc, Z_{s_2}\}$ be a vector of i.i.d.\ random variables drawn from $Q$ for $s_2 > s_1$.
% 	Furthermore, let
% 	\begin{equation}
% 		\Upsilon_{s_1, s_2}\left(x\right)
% 		= \E\left[h_{s_1}\left(x; Z_1, \ldots,  Z_{s_1}\right) \cdot
% 			h_{s_2}\left(x; Z_1, \ldots,Z_{s_1}, \ldots, Z_{s_2}\right)\right].
% 	\end{equation}
% 	Then,
% 	\begin{equation}
% 		\Upsilon_{s_1, s_2}\left(x\right)
% 		\lesssim  2\left(\mu_{0}^{1}(x) - \mu_{0}^{0}(x)\right)^2 + \frac{\overline{\sigma}^2_{\varepsilon}}{\mathfrak{p}(1 - \mathfrak{p})} + o(1)
% 		\quad \text{as} \quad s_1, s_2 \rightarrow \infty
% 		\quad \text{with} \quad
% 		0 < \mathfrak{c} \leq s_1 / s_2 \leq 1 - \mathfrak{c} < 1.
% 	\end{equation}
% \end{lem}
% \hrule
% \begin{proof}[Proof of Lemma~\ref{lem:CATE_upsilon_s}]
% 	Consider first the following.
% 	\begin{equation}
% 		\begin{aligned}
% 			\Upsilon_{s_1, s_2}\left(x\right)
% 			& = \E\left[h_{s_1}\left(x; Z_1, \ldots,  Z_{s_1}\right) \cdot
% 			h_{s_2}\left(x; Z_1, \ldots,Z_{s_1}, \ldots, Z_{s_2}\right)\right]\\
% 			%
% 			& = \E_{D}\left[
% 				\left(\sum_{i = 1}^{s_1} \kappa(x; Z_{i}, D_{[s_1]}) m\left(Z_{i}; \eta_{0}\right)\right)
% 				\left(\sum_{j = 1}^{s_1} \kappa(x; Z_{j}, D) m\left(Z_{j}; \eta_{0}\right) 
% 				+ \sum_{j = s_1 + 1}^{s_2} \kappa(x; Z_{j}, D) m\left(Z_{j}; \eta_{0}\right)\right)
% 			\right]     \\
% 			%
% 			& = \underbrace{\E_{D}\left[\sum_{i = 1}^{s_1}\sum_{j = 1}^{s_1}\kappa(x; Z_{i}, D_{[s_1]})\kappa(x; Z_{j}, D) m\left(Z_{i}; \eta_{0}\right)m\left(Z_{j}; \eta_{0}\right)\right]}_{(A)}\\
% 			& \quad + \underbrace{\E_{D}\left[\sum_{i = 1}^{s_1}\sum_{j = s_1 + 1}^{s_2}\kappa(x; Z_{i}, D_{[s_1]})\kappa(x; Z_{j}, D) m\left(Z_{i}; \eta_{0}\right)m\left(Z_{j}; \eta_{0}\right)\right]}_{(B)}
% 		\end{aligned}
% 	\end{equation}
% 	Using this decomposition, we can make the following findings.
% 	\begin{equation}
% 		\begin{aligned}
% 			(A)
% 			& = \E_{D}\left[\sum_{i = 1}^{s_1}\sum_{j = 1}^{s_1}\kappa(x; Z_{i}, D_{[s_1]})\kappa(x; Z_{j}, D) m\left(Z_{i}; \eta_{0}\right)m\left(Z_{j}; \eta_{0}\right)\right] \\
% 			%
% 			& = \E_{D}\left[\sum_{i = 1}^{s_1} \kappa(x; Z_{i}, D_{[s_1]}) m^2\left(Z_{i}; \eta_{0}\right)\right] 
% 			= \E_{D}\left[m^2\left(Z_{1}; \eta_{0}\right) s_1 \kappa(x; Z_{1}, D_{[s_1]}) \right] \\
% 			%
% 			& = \E_{1}\left[m^2\left(Z_{1}; \eta_{0}\right) s_1 \E_{2:s_2}\left[\kappa(x; Z_{1}, D_{[s_1]})\right]\right]
% 			= \E_{1}\left[m^2\left(Z_{1}; \eta_{0}\right) s_1 \E_{2:s_1}\left[\kappa(x; Z_{1}, D_{[s_1]})\right]\right]\\
% 			%
% 			& \overset{\text{(Lem~\ref{lem:limit_res})}}{\lesssim} \left(\mu_{0}^{1}(x) - \mu_{0}^{0}(x)\right)^2 + \frac{\overline{\sigma}^2_{\varepsilon}}{\mathfrak{p}(1 - \mathfrak{p})} + o(1)
% 		\end{aligned}
% 	\end{equation}
% 	\begin{equation}
% 		\begin{aligned}
% 			(B)
% 			& = \E_{D}\left[\sum_{i = 1}^{s_1}\sum_{j = s_1 + 1}^{s_2}\kappa(x; Z_{i}, D_{[s_1]})\kappa(x; Z_{j}, D) m\left(Z_{i}; \eta_{0}\right)m\left(Z_{j}; \eta_{0}\right)\right] \\
% 			%
% 			& = \E_{D}\left[s_1(s_2 - s_1) \kappa(x; Z_{1}, D_{[s_1]})\kappa(x; Z_{s_2}, D) m\left(Z_{1}; \eta_{0}\right) m\left(Z_{s_2}; \eta_{0}\right)\right]\\
% 			%
% 			& \leq \frac{(s_2 - s_1)}{s_2}\E_{D}\left[\left|m\left(Z_{1}; \eta_{0}\right)\right| s_1 \kappa(x; Z_{1}, D_{[s_1]})\right]
% 			\E_{D}\left[\left|m\left(Z_{s_2}; \eta_{0}\right)\right| s_2\kappa(x; Z_{s_2}, D) \right]\\
% 			%
% 			& \lesssim \frac{(s_2 - s_1)}{s_2} \left(\mu_{0}^{1}(x) - \mu_{0}^{0}(x)\right)^2 + o(1)
% 		\end{aligned}
% 	\end{equation}
% 	Thus, we obtain the desired result.
% 	\begin{equation}
% 		\Upsilon_{s_1, s_2}\left(x\right)
% 		= (A) + (B)
% 		\lesssim  2\left(\mu_{0}^{1}(x) - \mu_{0}^{0}(x)\right)^2 + \frac{\overline{\sigma}^2_{\varepsilon}}{\mathfrak{p}(1 - \mathfrak{p})} + o(1)
% 	\end{equation}
% \end{proof}
% \hrule 

% \begin{lem}\label{lem:CATE_upsilon_sc}\mbox{}\\*
% 	Let $D = \{Z_1, \dotsc, Z_{s_2}\}$ be a vector of i.i.d.\ random variables drawn from $Q$ for $s_2 > s_1$.
% 	Let $D^{\prime} = \{Z_1, \dotsc, Z_{c}, Z_{c+1}^{\prime}, \dotsc,  Z_{s_1}^{\prime}\}$ where $Z_{c+1}^{\prime}, \dotsc,  Z_{s_1}^{\prime}$ are i.i.d.\ draws from $P$ that are independent of $D$.
% 	Furthermore, let
% 	\begin{equation}
% 		\Upsilon_{s_1, s_2}^{c}\left(x\right)
% 		= \E\left[h_{s_1}\left(x; Z_1, \ldots, Z_c, Z^{\prime}_{c+1}, \ldots,  Z^{\prime}_{s_1}\right) \cdot
% 			h_{s_2}\left(x; Z_1, \ldots, Z_{s_2}\right)\right].
% 	\end{equation}
% 	Then,
% 	\begin{equation}
% 		\begin{aligned}
% 			 & \Upsilon_{s_1, s_2}^{c}\left(x\right)
% 			\lesssim 4 \left(\mu_{0}^{1}(x) - \mu_{0}^{0}(x)\right)^2 + \frac{\sigma^2_{\varepsilon}(x)}{\mathfrak{p}(1 - \mathfrak{p})} + o(1) \\
% 			%
% 			 & \text{for} \quad s_1, s_2 \quad \text{sufficiently large}
% 			\quad \text{with} \quad
% 			0 < \mathfrak{c} \leq s_1 / s_2 \leq 1 - \mathfrak{c} < 1.
% 		\end{aligned}
% 	\end{equation}
% \end{lem}

% \hrule
% \begin{proof}[Proof of Lemma~\ref{lem:CATE_upsilon_sc}]
% 	\begin{equation}
% 		\begin{aligned}
% 			\Upsilon_{s_1, s_2}^{c}\left(x\right)
% 			& = \E\left[h_{s_1}\left(x; Z_1, \ldots, Z_c, Z^{\prime}_{c+1}, \ldots,  Z^{\prime}_{s_1}\right) \cdot
% 			h_{s_2}\left(x; Z_1, \ldots, Z_{s_2}\right)\right] \\
% 			%
% 			& = \E_{D, D^{\prime}}\left[
% 				\left(\sum_{i = 1}^{c}\kappa(x; Z_{i}, D_{[s_1]}^{\prime}) m\left(Z_{i}; \eta_{0}\right) 
% 				+ \sum_{i = c+1}^{s_1}\kappa(x; Z_{i}^{\prime}, D_{[s_1]}^{\prime}) m\left(Z_{i}^{\prime}; \eta_{0}\right)\right)
% 				\left(\sum_{j = 1}^{s_2} \kappa(x; Z_{j}, D) m\left(Z_{j}; \eta_{0}\right)\right)\right]\\
% 			%
% 			& = \underbrace{\E_{D, D^{\prime}}\left[\sum_{i = 1}^{c}\sum_{j = 1}^{c} \kappa(x; Z_{i}, D_{[s_1]}^{\prime})\kappa(x; Z_{j}, D) m\left(Z_{i}; \eta_{0}\right) m\left(Z_{j}; \eta_{0}\right)\right]}_{(A)} \\
% 			& \quad + \underbrace{\E_{D, D^{\prime}}\left[\sum_{i = 1}^{c}\sum_{j = c+1}^{s_2} \kappa(x; Z_{i}, D_{[s_1]}^{\prime})\kappa(x; Z_{j}, D) m\left(Z_{i}; \eta_{0}\right) m\left(Z_{j}; \eta_{0}\right)\right]}_{(B)} \\
% 			& \quad + \underbrace{\E_{D, D^{\prime}}\left[\sum_{i = c+1}^{s_1}\sum_{j = 1}^{c}\kappa(x; Z_{i}^{\prime}, D_{[s_1]}^{\prime}) \kappa(x; Z_{j}, D) m\left(Z_{i}^{\prime}; \eta_{0}\right) m\left(Z_{j}; \eta_{0}\right)\right]}_{(C)} \\
% 			& \quad + \underbrace{\E_{D, D^{\prime}}\left[\sum_{i = c+1}^{s_1}\sum_{j = c+1}^{s_2}\kappa(x; Z_{i}^{\prime}, D_{[s_1]}^{\prime}) \kappa(x; Z_{j}, D) m\left(Z_{i}^{\prime}; \eta_{0}\right) m\left(Z_{j}; \eta_{0}\right)\right]}_{(D)}
% 		\end{aligned}
% 	\end{equation}
% 	\newpage
% 	Now, considering the terms individually, we find the following.
% 	\begin{equation}
% 		\begin{aligned}
% 			(A)
% 			& = \E_{D, D^{\prime}}\left[c \kappa(x; Z_{1}, D_{[s_1]}^{\prime})\kappa(x; Z_{1}, D) m^2\left(Z_{1}; \eta_{0}\right)\right]
% 			= \frac{c}{s_2} \cdot \E_{1}\left[m^2\left(Z_{1}; \eta_{0}\right) s_2 \E_{2:s_2}\left[\kappa(x; Z_{1}, D_{[s_1]}^{\prime})\kappa(x; Z_{1}, D)\right] \right]\\
% 			%
% 			& \leq  \frac{c}{s_2} \cdot \E_{1}\left[m^2\left(Z_{1}; \eta_{0}\right) s_2 \E_{2:s_2}\left[\kappa(x; Z_{1}, D)\right] \right]
% 			\lesssim \frac{c}{s_2} \left(\left(\mu_{0}^{1}(x) - \mu_{0}^{0}(x)\right)^2 + \frac{\sigma^2_{\varepsilon}(x)}{\mathfrak{p}(1 - \mathfrak{p})}\right) + o(1)\\
% 		\end{aligned}
% 	\end{equation}
% 	Similarly, we find the following.
% 	\begin{equation}
% 		\begin{aligned}
% 			(B)
% 			& = \E_{D, D^{\prime}}\left[c (s_2 - c) \kappa(x; Z_{1}, D_{[s_1]}^{\prime})\kappa(x; Z_{c+1}, D) m\left(Z_{1}; \eta_{0}\right) m\left(Z_{c+1}; \eta_{0}\right)\right]\\
% 			%
% 			& = \frac{c (s_2 - c)}{s_1 s_2} \E_{D, D^{\prime}}\left[m\left(Z_{1}; \eta_{0}\right) m\left(Z_{c+1}; \eta_{0}\right) s_1 s_2 \kappa(x; Z_{1}, D_{[s_1]}^{\prime})\kappa(x; Z_{c+1}, D) \right]\\
% 			%
% 			& \leq \frac{c (s_2 - c)}{s_1 s_2} \E_{D, D^{\prime}}\left[\left|m\left(Z_{1}; \eta_{0}\right)\right|  s_1 \kappa(x; Z_{1}, D_{[s_1]}^{\prime})\right]
% 			\E_{D, D^{\prime}}\left[\left| m\left(Z_{c+1}; \eta_{0}\right) \right| s_2 \kappa(x; Z_{c+1}, D) \right] \\
% 			%
% 			& \lesssim \frac{c (s_2 - c)}{s_1 s_2} \left(\mu_{0}^{1}(x) - \mu_{0}^{0}(x)\right)^2 + o(1)
% 		\end{aligned}
% 	\end{equation}
% 	Applying the same argument to the third term, we find an analogous result.
% 	\begin{equation}
% 		\begin{aligned}
% 			(C)
% 			& = \E_{D, D^{\prime}}\left[(s_1 - c)c\kappa(x; Z_{c+1}^{\prime}, D_{[s_1]}^{\prime}) \kappa(x; Z_{1}, D) m\left(Z_{c+1}^{\prime}; \eta_{0}\right) m\left(Z_{1}; \eta_{0}\right)\right]\\
% 			%
% 			& \lesssim  \frac{c (s_1 - c)}{s_1 s_2} \left(\mu_{0}^{1}(x) - \mu_{0}^{0}(x)\right)^2 + o(1)
% 		\end{aligned}
% 	\end{equation}
% 	Finally, for the fourth term, we can make the following observation.
% 	\begin{equation}
% 		\begin{aligned}
% 			(D)
% 			& = \E_{D, D^{\prime}}\left[(s_1 - c)(s_2 - c)\kappa(x; Z_{c+1}^{\prime}, D_{[s_1]}^{\prime}) \kappa(x; Z_{c+1}, D) m\left(Z_{c+1}^{\prime}; \eta_{0}\right) m\left(Z_{c+1}; \eta_{0}\right)\right]\\
% 			%
% 			& = \frac{(s_1 - c)(s_2 - c)}{s_1 s_2}\E_{D, D^{\prime}}\left[m\left(Z_{c+1}^{\prime}; \eta_{0}\right) m\left(Z_{c+1}; \eta_{0}\right) s_1 s_2 \kappa(x; Z_{c+1}^{\prime}, D_{[s_1]}^{\prime}) \kappa(x; Z_{c+1}, D)\right]\\
% 			%
% 			& \lesssim  \frac{(s_1 - c)(s_2 - s_1)}{s_1 s_2} \left(\mu_{0}^{1}(x) - \mu_{0}^{0}(x)\right)^2 + o(1)
% 		\end{aligned}
% 	\end{equation}
% 	By combining these asymptotic bounds, we find the desired result.
% 	\begin{equation}
% 		\begin{aligned}
% 			\Upsilon_{s_1, s_2}^{c}\left(x\right)
% 			& = (A) + (B) + (C) + (D)
% 			\lesssim 4 \left(\mu_{0}^{1}(x) - \mu_{0}^{0}(x)\right)^2 + \frac{\sigma^2_{\varepsilon}(x)}{\mathfrak{p}(1 - \mathfrak{p})} + o(1)
% 		\end{aligned}
% 	\end{equation}
% \end{proof}

\newpage
\input{Parts/P2c_PWInf_VarEst.tex}

\newpage
\section{Proofs for Results in Section~\ref{sec:unif_inf}}
\hrule

\end{document}