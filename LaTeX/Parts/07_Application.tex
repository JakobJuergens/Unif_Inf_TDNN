\section{Application}\label{sec:application}
\hrule

I apply the method presented in this paper to the estimation of CATE in the well-studied \textit{National Job Training Partnership Act} (JTPA) Study.
Notable papers that have studied the same data set and might serve as useful points of reference include \citet{bloom_benefits_1997}, \citet{heckman_matching_1997}, \citet{abadie_instrumental_2002}, and \citet{kitagawa_who_2018}.
For the analysis presented in the following, I combined data provided by Josh Angrist in his data archive (\citet{abadie_replication_2008}) with data from the original data set provided by the W.E. Upjohn Institute for Employment Research to obtain the type of covariates to suitably illustrate the estimators performance in a real-world setting.
For this purpose, specifically to mimic a non-experimental setting, I restrict the data set to the individuals chosen to be eligible for treatment at random.
In essence, this takes the perspective of a decision maker that wants to evaluate the intent-to-treat CATE given observational data, a common scenario to determine eligibility criteria for policy measures.

{\color{red} LOREM IPSUM}