\section{The (T)DNN Estimator as a Generalized U-Statistic}
\hrule

As the majority of the theoretical results in \citet{demirkaya_optimal_2024} rely on representations as a U-statistic, it is helpful to introduce additional concepts and notation at this stage.
Recalling Equation~\ref{eq:U_stat}, the DNN and TDNN estimators can be expressed in the following U-statistic form and are thus a type of generalized complete U-statistic as introduced by \citet{peng_rates_2022}.
\begin{equation}
	\tilde{\mu}_{s}(x; \mathbf{D}_n)
	= \binom{n}{s}^{-1} \sum_{\ell \in L_{n,s}} h_{s}(x; D_{\ell})
	\quad \text{and} \quad
	\hat{\mu}_{s_1, s_2}(x; \mathbf{D}_n)
	= \binom{n}{s}^{-1} \sum_{\ell \in L_{n,s_2}} h_{s_1, s_2}(x; D_{\ell})
\end{equation}
It is worth pointing out that in contrast to the DNN estimator, the kernel for the TDNN estimator is of order $s_2 > s_1$.
The authors derive an explicit formula for the kernel that shows the connection between the DNN and TDNN estimators.
This connection will prove useful going forward.
\begin{boxD}
	\begin{lem}[Kernel of TDNN Estimator - Adapted from Lemma 8 of \citet{demirkaya_optimal_2024}]\label{lem:dem8}\mbox{}\\*
		The kernel of the TDNN estimator takes the following form.
		\begin{equation}
			\begin{aligned}
				h_{s_1, s_2}\left(x; D\right)
				 & = w_{1}^{*}\left[\binom{s_2}{s_1}^{-1}\sum_{\ell \in L_{s_2, s_1}} h_{s_1}\left(x; D_{\ell}\right)\right] + w_{2}^{*} h_{s_2}\left(x; D\right) \\
				 & = w_{1}^{*} \tilde{\mu}_{s_1}\left(x; D\right) + w_{2}^{*} h_{s_2}\left(x; D\right)                                                            \\
			\end{aligned}
		\end{equation}
	\end{lem}
\end{boxD}
Borrowing the notational conventions from \citet{lee_u-statistics_2019}, additionally, introduce the following notation.
\begin{equation}\label{eq:psi_s_c}
	\psi_{s}^{c}(x; \mathbf{z}_{1}, \dotsc, \mathbf{z}_{c})
	= \E_{D}\left[h_{s}\left(x; D\right) \, | \,  Z_1 = \mathbf{z}_{1}, \dotsc, Z_c = \mathbf{z}_{c}\right]
\end{equation}
\begin{equation}
	h_{s}^{(1)}\left(x; \mathbf{z}_{1}\right)
	= \psi_{s}^{1}(x; \mathbf{z}_{1}) - \mu(x)
\end{equation}
\begin{equation}
	h_{s}^{(c)}\left(x; \mathbf{z}_{1}, \dotsc, \mathbf{z}_{c}\right)
	= \psi_{s}^{c}(x; \mathbf{z}_{1}, \dotsc, \mathbf{z}_{c}) - \sum_{j = 1}^{c-1}\left(\sum_{\ell \in L_{n,j}}h_{s}^{(j)}(x; \mathbf{z}_{\ell})\right) - \mu(x)
	\quad \text{for } c = 2, \dotsc, s
\end{equation}
In contrast to the notational inspiration, the subsampling size $s$ is made explicit.
Since we are dealing with an infinite-order U-statistic, $s$ will be diverging with $n$.
Completely analogous, define the corresponding objects for the TDNN estimator.
For the DNN estimator and any $1 \leq c \leq s$, define
\begin{equation}\label{eq:xi_s_c}
	\xi_{s}^{c}\left(x\right)
	= \Var_{1:c}\left(\psi_{s}^{c}(x; Z_{1}, \dotsc, Z_{c})\right)
\end{equation}
where $Z_{c+1}^{\prime}, \ldots, Z_n^{\prime}$ are i.i.d.\ from $P$ and independent of $Z_1, \ldots, Z_n$ and thus
$\xi_{s}^{s}\left(x\right) = \Var\left(h_s\left(x; Z_1, \ldots, Z_s\right)\right)$.
% Then, we have the following result from the original paper.
Similarly, for the TDNN estimator and any $1 \leq c \leq s_2$, let
\begin{equation}\label{eq:zeta_s1s2_c}
	\zeta_{s_1, s_2}^{c}\left(x\right)
	= \Var_{1:c}\left(\psi_{s_1, s_2}^{c}(x; Z_{1}, \dotsc, Z_{c})\right)
\end{equation}
with an analogous definition of $Z^{\prime}$.

\subsection{Hoeffding-Decomposition}
\hrule
As a byproduct (or main purpose depending on the perspective) these terms can be used to derive the Hoeffding decomposition of the TDNN estimator.
\begin{equation}\label{eq:H_projection}
	\begin{aligned}
		H_{s}^{c}\left(x; \mathbf{D}_n\right)
		= \binom{n}{c}^{-1} \sum_{\ell \in L_{n,c}} h^{(c)}_{s}(x; D_{\ell})
		\quad \text{and} \quad
		H_{s_1, s_2}^{c}\left(x; \mathbf{D}_n\right)
		= \binom{n}{c}^{-1} \sum_{\ell \in L_{n,c}} h^{(c)}_{s_1, s_2}(x; D_{\ell})
	\end{aligned}
\end{equation}
These projection terms can then be used to construct the following Hoeffding decompositions.
\begin{equation}\label{eq:H_Decomp}
	\tilde{\mu}_{s}\left(x; \mathbf{D}_n\right)
	= \mu(x) + \sum_{j = 1}^{s}\binom{s}{j}H_{s}^{j}\left(x; \mathbf{D}_n\right)\\
	\quad \text{and} \quad
	\hat{\mu}_{s_1, s_2}\left(x; \mathbf{D}_n\right)
	= \mu(x) + \sum_{j = 1}^{s_2}\binom{s_2}{j}H_{s_1, s_2}^{j}\left(x; \mathbf{D}_n\right)
\end{equation}

Standard results for U-statistics (see for example \citet{lee_u-statistics_2019}) now give us a number of useful results.
First, an immediate result on the expectations of the Hoeffding-projection kernels.
\begin{align}\label{eq:H_k_expectation}
	\forall c = 1,2,\dotsc, j-1: \quad & \E_{D}\left[h_{s_1, s_2}^{(j)}\left(x; D\right) \, | \, Z_1 = \mathbf{z}_1, \dotsc, Z_c = \mathbf{z}_c\right] = 0
	\quad \text{and} \quad
	\E_{D}\left[h_{s_1, s_2}^{(j)}\left(x; D\right)\right] = 0
\end{align}
Second, we obtain a useful variance decomposition in terms of the Hoeffding-projection variances.
\begin{align}\label{eq:Var_decomp}
	\Var_{D}\left(\hat{\mu}_{s_1, s_2}\left(x; D\right)\right)
	 & = \sum_{j = 1}^{s_2} \binom{s_2}{j}^2 \Var_{D}\left(H_{s_1, s_2}^{j}\left(x; D\right)\right) \\
	%
	\Var_{D}\left(H_{s_1, s_2}^{j}\left(x; D\right)\right)
	 & = \binom{n}{j}^{-1} \Var_{D}\left(h_{s_1, s_2}^{(j)}\left(x; D\right)\right)
	=: \binom{n}{j}^{-1} V_{s_1, s_2}^{j}\left(x\right)
\end{align}
Third, the following equivalent expression for the kernel variance.
\begin{equation}\label{eq:k_var}
	\zeta_{s_1, s_2}^{s_2}\left(x\right)
	= \Var_{D}\left(h_{s_1, s_2}\left(x; D\right)\right)
	= \sum_{j = 1}^{s_2} \binom{s_2}{j}V_{s_1, s_2}^{j}
\end{equation}