\section{Setup}\label{sec:setup}
\hrule
Throughout this paper, we will consider two distinct setups.
The first is a pure nonparametric regression setup closely mirroring the structure of \citet{demirkaya_optimal_2024}.
This setup will be very useful to illustrate the inner workings of the estimator of interest and serve as a leading example for the theoretical results.
\begin{boxD}
	\begin{asm}[Nonparametric Regression DGP]\label{asm:npr_dgp}\mbox{}\\*
		The observed data consists of an i.i.d. sample taking the following form.
		\begin{equation}\label{DGP1}
			\mathbf{D}_n = \{\mathbf{Z}_i = (\mathbf{X}_i, Y_i)\}_{i = 1}^{n}
			\quad \text{from the model} \quad
			Y = \mu(\mathbf{X}) + \varepsilon,
		\end{equation}
		where $Y \in \mathbb{R}$ is the response, $\mathbf{X} \in \mathcal{X} \subset \mathbb{R}^k$ is a feature vector of fixed dimension $k$ distributed according to a density function $f$, and $\mu(\mathbf{x})$ is the unknown mean regression function.
		$\varepsilon$ is the unobservable model error on which we impose the following conditions.
		\begin{equation}
			\E\left[\varepsilon \, \middle| \, \mathbf{X}\right] = 0, \quad
			\Var\left(\varepsilon \, \middle| \, \mathbf{X} = \mathbf{x}\right) = \sigma_{\varepsilon}^2\left(\mathbf{x}\right)
		\end{equation}
		Let the distribution induced by this model be denoted by $P$ and thus $\mathbf{Z}_i = \left(\mathbf{X}_i, Y_i\right) \overset{\text{iid}}{\sim} P$.
	\end{asm}
\end{boxD}
In contrast to this rather statistical setup, I will consider a setting with more immediate econometric relevance: estimation of and inference on heterogeneous treatment effects in the potential outcomes framework.
This serves as a more immediately applicable version of the theoretical setup presented in \citet{ritzwoller_uniform_2024} and brings their results closes to practitioners in the field of economics.
\begin{boxD}
	\begin{asm}[Heterogeneous Treatment Effect DGP]\label{asm:hte_dgp}\mbox{}\\*
		The observed data consists of an i.i.d. sample taking the following form.
		\begin{equation}\label{DGP2}
			\begin{aligned}
				\mathbf{D}_n & = \{\mathbf{Z}_i = (\mathbf{X}_i, W_i, Y_i)\}_{i = 1}^{n}
				\quad \text{from the model} \quad
				Y = \1(W = 0)\mu_{0}(\mathbf{X}) + \1(W = 1)\mu_1(\mathbf{X}) + \varepsilon,	\\
				W_i & \sim \operatorname{Bern}\left(\pi\left(\mathbf{X}_i\right)\right)
			\end{aligned}
		\end{equation}
		where $Y \in \mathbb{R}$ is the response, $W \in \{0,1\}$ is an observed treatment indicator, $\mathbf{X} \in \mathcal{X} \subset \mathbb{R}^k$ is a feature vector of fixed dimension $k$ distributed according to a density function $f$ and $\varepsilon$ is the unobservable model error on which we impose the following conditions.
		\begin{equation}
			\varepsilon \indep W \, | \, \mathbf{X}, \quad
			\E\left[\varepsilon \, | \, \mathbf{X}\right] = 0, \quad
			\Var\left(\varepsilon \, | \, \mathbf{X} = \mathbf{x}\right) = \sigma_{\varepsilon}^2\left(\mathbf{x}\right)
		\end{equation}
		Furthermore, $\mu_0:\mathcal{X} \rightarrow \mathbb{R}$ and $\mu_1:\mathcal{X} \rightarrow \mathbb{R}$ are the two unknown potential outcome functions and $\pi:\mathcal{X} \rightarrow [0,1]$ is a function describing the probability of treatment uptake, effectively corresponding to the propensity score.
		Let the distribution induced by this model be denoted by $Q$ and thus $\mathbf{Z}_i = \left(\mathbf{X}_i, W_i, Y_i\right) \overset{\text{iid}}{\sim} Q$.
	\end{asm}
\end{boxD}
In this second setting, I will use the notation $\mathbf{D}^{(0)}$ and $\mathbf{D}^{(1)}$ to refer to the data subsets containing only observations with $W = 0$ and $W = 1$, respectively.
Clearly, this model can be interpreted in the context of the potential outcomes framework in the usual manner.
To emphasize this interpretation, consider the following two specific contexts in applied research.
\begin{boxD}
	\begin{exmp}[Randomized Experiment Analysis]\mbox{}\\*
		{\color{red} LOREM IPSUM}
		% Consider a government agency evaluating a proposal to extend access to a job-training program.
		% Individuals are eligible to participate in the program depending on their characteristics $\mathbf{X}$.
		% Whether individuals choose to participate given they are eligible, i.e. the function $p$, cannot be influenced by the agency directly and serves as a nuisance parameter in the estimation problem of treatment effects.\\

		% To evaluate the average effect of such changes on individuals with characteristics $\mathbf{x}$, we need to estimate the conditional average treatment effect.
		% Uniformly valid confidence bands for these treatment effects would allow the agency to evaluate the program in an informed way with respect to the whole potentially treated subspace of characteristics.
	\end{exmp}
\end{boxD}

\begin{boxD}
	\begin{exmp}[Natural Experiment Analysis]\mbox{}\\*
		{\color{red} LOREM IPSUM}
	\end{exmp}
\end{boxD}

Throughout this paper, I will additionally rely on a number of assumptions that are more technical in nature.
\begin{boxD}
	\begin{asm}[Technical Assumptions]\label{asm:technical}\mbox{}\\*
		In both settings (Assumption \ref{asm:npr_dgp} and Assumption \ref{asm:hte_dgp}) the following conditions hold:
		\begin{itemize}
			\item The feature space $\mathcal{X} = \operatorname{supp}(X)$ is a bounded, compact subset of $\mathbb{R}^k$
			\item The density $f(\cdot)$ is bounded away from 0 and $\infty$
			\item $f(\cdot)$ are $\mu(\cdot)$ are four times continuously differentiable with bounded second, third, and fourth-order partial derivatives in a neighborhood of $\mathbf{x}$
			\item $\E\left[Y^2\right]<\infty$
		\end{itemize}
		In the Heterogeneous Treatment Effect setting (Assumption \ref{asm:hte_dgp}), the following additional condition holds:
		\begin{itemize}
			\item $\mu_0(\cdot)$ and $\mu_1(\cdot)$ are four times continuously differentiable with bounded second, third, and fourth-order partial derivatives in a neighborhood of $\mathbf{x}$
		\end{itemize}
	\end{asm}
\end{boxD}
\begin{boxD}
	\begin{asm}[Error Distribution Assumptions]\label{asm:sg_errors}\mbox{}\\*
		\begin{itemize}
			\item Uniformly subgaussian
			\item Assumptions on $\sigma^2_{\varepsilon}\left(\mathbf{x}\right)$ and its derivative
		\end{itemize}
		{\color{red} LOREM IPSUM}
	\end{asm}
\end{boxD}
There is potential to relax these assumptions at the cost of requiring both less interpretable conditions and more technically sophisticated proofs.
Furthermore, to assure that there is a sufficient number of treated and untreated observations local to each point of interest asymptotically, we require the following condition on the treatment assignment and uptake mechanism.
\begin{boxD}
	\begin{asm}[Non-Trivial Treatment Overlap]\label{asm:treatment_overlap}\mbox{}\\*
		In the Heterogeneous Treatment Effect Setup (Assumption \ref{asm:hte_dgp}), we assume that there exist a constant $\mathfrak{p} \in (0, 1/2)$ such that
		\begin{equation}
			\forall \mathbf{x} \in \mathcal{X}: \quad 
			0 < \mathfrak{p} \leq \pi\left(\mathbf{x}\right) \leq 1 - \mathfrak{p} < 1.
		\end{equation}
	\end{asm}
\end{boxD}