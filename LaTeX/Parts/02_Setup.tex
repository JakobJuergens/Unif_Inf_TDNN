\section{Setup}\label{sec:setup}
\hrule
Throughout this paper, we will consider two distinct setups.
The first is a pure nonparametric regression setup closely mirroring the structure of \citet{demirkaya_optimal_2024}.
This setup will be very useful to illustrate the inner workings of the estimator of interest and serve as a leading example for the theoretical results.
\begin{boxD}
	\begin{asm}[Nonparametric Regression DGP]\label{asm:npr_dgp}\mbox{}\\*
		The observed data consists of an i.i.d. sample taking the following form.
		\begin{equation}\label{DGP1}
			\mathbf{D}_n = \{Z_{i} = (X_{i}, Y_{i})\}_{i = 1}^{n}
			\quad \text{from the model} \quad
			Y = \mu(X) + \varepsilon,
		\end{equation}
		where $Y \in \mathbb{R}$ is the response, $X \in \mathcal{X} \subset \mathbb{R}^k$ is a feature vector of fixed dimension $k$ distributed according to a density function $f$ with associated probability measure $\varphi$ on $\mathcal{X}$, and $\mu(x)$ is the unknown mean regression function.
		$\varepsilon$ is the unobservable model error on which we impose the following conditions.
		\begin{equation}
			\E\left[\varepsilon \, \middle| \, X\right] = 0, \quad
			\Var\left(\varepsilon \, \middle| \, X = x\right) = \sigma_{\varepsilon}^2\left(x\right)
		\end{equation}
		Let the distribution induced by this model be denoted by $P$ and thus $Z_{i} = \left(X_{i}, Y_{i}\right) \overset{\text{iid}}{\sim} P$.
	\end{asm}
\end{boxD}
In contrast to this rather statistical setup, I will consider a setting with more immediate econometric relevance: estimation of and inference on heterogeneous treatment effects in the potential outcomes framework.
This serves as a more immediately applicable version of the theoretical setup presented in \citet{ritzwoller_uniform_2024} and brings their results closes to practitioners in the field of economics.
\begin{boxD}
	\begin{asm}[Heterogeneous Treatment Effect DGP]\label{asm:hte_dgp}\mbox{}\\*
		The observed data consists of an i.i.d. sample taking the following form.
		\begin{equation}\label{DGP2}
			\begin{aligned}
				\mathbf{D}_n & = \{Z_{i} = (X_{i}, W_{i}, Y_{i})\}_{i = 1}^{n}
				\quad \text{from the model} \quad
				Y = \1(W = 0)\mu_{0}(X) + \1(W = 1)\mu_1(X) + \varepsilon,	\\
				W_{i} & \sim \operatorname{Bern}\left(\pi\left(X_{i}\right)\right)
			\end{aligned}
		\end{equation}
		where $Y \in \mathbb{R}$ is the response and $W \in \{0,1\}$ is an observed treatment indicator.
		$X \in \mathcal{X} \subset \mathbb{R}^k$ is a vector of covariates of fixed dimension $k$ distributed according to a density function $f$ with associated probability measure $\varphi$ on $\mathcal{X}$ and $\varepsilon$ is the unobservable model error on which we impose the following conditions.
		\begin{equation}
			\varepsilon \indep W \, | \, X, \quad
			\E\left[\varepsilon \, | \, X\right] = 0, \quad
			\Var\left(\varepsilon \, | \, X = x\right) = \sigma_{\varepsilon}^2\left(x\right)
		\end{equation}
		Furthermore, $\mu_0:\mathcal{X} \rightarrow \mathbb{R}$ and $\mu_1:\mathcal{X} \rightarrow \mathbb{R}$ are the two unknown potential outcome functions and $\pi:\mathcal{X} \rightarrow [0,1]$ is a function describing the probability of treatment uptake, effectively corresponding to the propensity score.
		Let the distribution induced by this model be denoted by $Q$ and thus $Z_{i} = \left(X_{i}, W_{i}, Y_{i}\right) \overset{\text{iid}}{\sim} Q$.
	\end{asm}
\end{boxD}
In this second setting, I will use the notation $\mathbf{D}^{(0)}$ and $\mathbf{D}^{(1)}$ to refer to the data subsets containing only observations with $W = 0$ and $W = 1$, respectively.
Clearly, this model can be interpreted in the context of the potential outcomes framework in the usual manner.
\begin{boxD}
	\begin{rmk}[Potential Applications]\mbox{}\\*
		From an microeconometric perspective, these two setups cover a wide array of applications.
		While nonparametric regression is itself often advantageous to answer economic questions, the real strengths show when considering the second setup.
		{\color{red} LOREM IPSUM}
	\end{rmk}
\end{boxD}
Throughout this paper, I will additionally rely on a number of assumptions that are more technical in nature.
\begin{boxD}
	\begin{asm}[Technical Assumptions]\label{asm:technical}\mbox{}\\*
		In both settings (Assumption \ref{asm:npr_dgp} and Assumption \ref{asm:hte_dgp}) the following conditions hold:
		\begin{itemize}
			\item The feature space $\mathcal{X} = \operatorname{supp}(X)$ is a bounded, compact subset of $\mathbb{R}^k$
			\item The density $f(\cdot)$ is bounded away from 0 and $\infty$
			\item $f(\cdot)$ are $\mu(\cdot)$ are four times continuously differentiable with bounded second, third, and fourth-order partial derivatives in a neighborhood of $x$
		\end{itemize}
		In the Heterogeneous Treatment Effect setting (Assumption \ref{asm:hte_dgp}), the following additional condition holds:
		\begin{itemize}
			\item $\mu_0(\cdot)$ and $\mu_1(\cdot)$ are four times continuously differentiable with bounded second, third, and fourth-order partial derivatives in a neighborhood of $x$
		\end{itemize}
	\end{asm}
\end{boxD}
There is potential to relax these assumptions at the cost of requiring both less interpretable conditions and more technically sophisticated proofs.
Additionally, we require a rather standard assumption in localized regression approaches, namely that the variance changes continuously.
\begin{boxD}
	\begin{asm}[Error Distribution Assumptions]\label{asm:errors}\mbox{}\\*
		The error terms $\varepsilon$ defined in Setup \ref{asm:npr_dgp} and Setup \ref{asm:hte_dgp}, respectively, have continuously varying variance.
		In other terms,	$\sigma^2_{\varepsilon}: \mathcal{X} \rightarrow \mathbb{R}_{>0}$ is a continuous function.
	\end{asm}
\end{boxD}
As $\mathcal{X}$ is a bounded and compact set, this implies that there exists a $\overline{\sigma}_{\varepsilon}^2 > 0$ such that for any $x \in \mathcal{X}$ we have $\sigma^{2}_{\varepsilon}\left(x\right) \leq \overline{\sigma}_{\varepsilon}^2$.
Additionally, due to the assumptions on the regression functions, this ensures the existence of seconds moments of $Y$ in both scenarios.
Furthermore, to assure that there is a sufficient number of treated and untreated observations local to each point of interest asymptotically, we require the following condition on the treatment assignment and uptake mechanism.
\begin{boxD}
	\begin{asm}[Non-Trivial Treatment Overlap]\label{asm:treatment_overlap}\mbox{}\\*
		In the Heterogeneous Treatment Effect Setup (Assumption \ref{asm:hte_dgp}), we assume that there exist a constant $\mathfrak{p} \in (0, 1/2)$ such that
		\begin{equation}
			\forall x \in \mathcal{X}: \quad 
			0 < \mathfrak{p} \leq \pi\left(x\right) \leq 1 - \mathfrak{p} < 1.
		\end{equation}
	\end{asm}
\end{boxD}
This assumption seems rather strong when considering a full universe of potential treatment recipients.
In reality we can constrain this overlap assumption to neighborhoods of points of interests $x$.
As long as there is sufficient overlap in those neighborhoods the ideas of our identification strategy continue to hold locally.
\begin{boxD}
	\begin{asm}[Stable Unit Treatment Value Assumption (SUTVA)]\label{asm:sutva}\mbox{}\\*
		For any $n$, let $\mathfrak{W}_{n}: \mathcal{X}^{n} \rightarrow \{0,1\}^{n}$ and $\mathfrak{W}_{n}^{\prime}: \mathcal{X}^{n} \rightarrow \{0,1\}^{n}$ be two functions characterizing treatment assignment among a group of $n$ potential observations.
		{\color{red} LOREM IPSUM}
	\end{asm}
\end{boxD}
