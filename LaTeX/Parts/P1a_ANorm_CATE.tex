\subsection{CATE-Estimator - Asymptotic Normality}
\hrule
Recall the decomposition of the DNN-DML2 estimator introduced in equation \ref{eq:DNNDML2_Decomp}.
\begin{equation}
    \begin{aligned}
        \hat{\theta}\left(x; \mathbf{D}\right)
        & = \underbrace{\underbrace{\E_{D}\left[\hat{\theta}_{0}\left(x; \mathbf{D}\right)\right]}_{\text{Centering-Term}}
        + \underbrace{\frac{s}{n}\sum_{i = 1}^{n} \chi_{s,0}^{(1)}\left(x; Z_{i}\right)}_{\text{Oracle-H\'ajek-Projection}}
        + \underbrace{\sum_{j = 2}^{s} \binom{s}{j} \binom{n}{j}^{-1}\sum_{\ell \in L_{n,j}} \chi_{s,0}^{(j)}\left(x; \mathbf{D}_{\ell}\right)}_{\text{Oracle-H\'ajek-Residual}}}_{\text{Oracle-Hoeffding-Projection}}\\
        & \quad \quad + \underbrace{\binom{n}{s}^{-1}\sum_{\ell \in L_{n,s}}\left(\chi_{s}\left(x; \mathbf{D}_{\ell}, \hat{\eta}\right) - \chi_{s,0}\left(x; \mathbf{D}_{\ell}\right)\right)}_{\text{First-Stage Approximation Error}}
    \end{aligned}
\end{equation}
We can furthermore decompose the error component in the following two ways that will be used in further arguments.
\begin{equation}
    \begin{aligned}
        & \binom{n}{s}^{-1} \sum_{\ell \in L_{n,s}}\left(\chi_{s}\left(x; \mathbf{D}_{\ell}, \hat{\eta}\right) - \chi_{s,0}\left(x; \mathbf{D}_{\ell}\right)\right)
        = \underbrace{\sum_{i = 1}^{n - s + 1} \frac{\binom{n-i}{s-1}}{\binom{n}{s}} \left(m(Z_{(i)}, \hat{\eta}_{k_{(i)}}) - m(Z_{(i)}, \eta_{0})\right)}_{\text{Weighted-NN Error Representation}} \\
        %
        & \quad = \underbrace{\frac{s}{K} \sum_{k = 1}^{K} \frac{1}{m} \sum_{i \in \mathcal{I}_{k}}\left(\underbrace{\chi_{s}^{(1)}\left(x; Z_{i}, \hat{\eta}_{k}\right) - \chi_{s,0}^{(1)}\left(x; Z_{i}\right)}_{R_{1,k}\left(x; Z_{i}\right)}\right)}_{\text{Oracle-H\'ajek-Projection Error}}
         + \underbrace{\sum_{j = 2}^{s} \binom{s}{j} \binom{n}{j}^{-1}\sum_{\ell \in L_{n,j}} R_{j}\left(x; \mathbf{D}_{\ell}\right)}_{\text{Higher-Order Error Terms}}
    \end{aligned}
\end{equation}
where we have the following definition from Equation \ref{eq:DNNDML2_ResidDecomp}.
\begin{equation}
    \begin{aligned}
        \chi_{s}^{(c)}\left(x; \mathbf{D}_{\ell}, \hat{\eta}\right)
        & = \chi_{s,0}^{(c)}\left(x; \mathbf{D}_{\ell}\right) + \underbrace{\chi_{s}^{(c)}\left(x; \mathbf{D}_{\ell}, \hat{\eta}\right) - \chi_{s,0}^{(c)}\left(x; \mathbf{D}_{\ell}\right)}_{R_{c}\left(x; \mathbf{D}_{\ell}\right)}
    \end{aligned}
\end{equation}
Recall that we are ultimately even more interested in the approximation errors of the following form.
\begin{equation}
    \tilde{R}_{c}\left(x; \mathbf{D}_{\ell}\right)
     = R_{c}\left(x; \mathbf{D}_{\ell}\right) - (-1)^{c} \cdot \left(\E_{D}\left[\chi_{s}(x; \mathbf{D}_{[s]}, \hat{\eta}) - \chi_{s,0}\left(x; \mathbf{D}_{[s]}\right)\right]\right)
\end{equation}
This follows from the following combinatorial fact describing the elimination of centering terms in successive Hoeffding projection terms.
\begin{equation}
    \sum_{j = 1}^{s-1}(-1)^{j+1} \binom{s}{j} = 1 + (-1)^{s}
\end{equation}
Additionally, define the following empirical process notation, where $f$ is any $Q$-integrable function on $\mathcal{Z}$.
\begin{equation}
    \begin{aligned}
        \mathbb{G}_{m,k}\left[f(Z)\right]
        & = \sqrt{\frac{1}{m}} \sum_{i \in I_{k}} \Big(f(Z_i) - \E_{Z}\left[f(Z)\right]\Big)
    \end{aligned}
\end{equation}
    
\newpage

\begin{boxD}
    \begin{lem}[Behavior of First-Stage Approximation Error]\label{lem:fs_approx_error}\mbox{}\\*
    Concerning the first-stage approximation error, the following holds.
        \begin{equation}
         \binom{n}{s}^{-1}\sum_{\ell \in L_{n,s}}\left(\chi_{s}\left(x; \mathbf{D}_{\ell}, \hat{\eta}\right) - \chi_{s,0}\left(x; \mathbf{D}_{\ell}\right)\right)
         = O_{p}\left(s \cdot \left(n^{-1/2} r_{n}^{\prime} + \lambda_{n}^{\prime}\right)\right)
    \end{equation}
    Furthermore, if $s = o\left(\min\left\{1/r_{n}^{\prime}, \left(\sqrt{n} \cdot \lambda_{n}^{\prime}\right)^{-1}\right\}\right)$, this implies that 
    \begin{equation}
        \binom{n}{s}^{-1}\sum_{\ell \in L_{n,s}}\left(\chi_{s}\left(x; \mathbf{D}_{\ell}, \hat{\eta}\right) - \chi_{s,0}\left(x; \mathbf{D}_{\ell}\right)\right)
        \longrightarrow_{p} 0.
    \end{equation}
\end{lem}
\end{boxD}

\begin{proof}[Proof of Lemma \ref{lem:fs_approx_error}]\mbox{}\\*
    Recall first the weighted nearest neighbor representation of the first-stage approximation error.
    \begin{equation}
        \begin{aligned}
            \mathbf{R}\left(x; \mathbf{D}_{\ell}, \hat{\eta}\right)
            := \binom{n}{s}^{-1}\sum_{\ell \in L_{n,s}}\left(\chi_{s}\left(x; \mathbf{D}_{\ell}, \hat{\eta}\right) - \chi_{s,0}\left(x; \mathbf{D}_{\ell}\right)\right) 
            = \sum_{i = 1}^{n - s + 1} \frac{\binom{n-i}{s-1}}{\binom{n}{s}} \left(m(Z_{(i)}, \hat{\eta}_{k_{(i)}}) - m(Z_{(i)}, \eta_{0})\right)
        \end{aligned}
    \end{equation}
    Now, fix an arbitrary fold $k \in [K]$ and consider the corresponding terms of the sum shown above.
    \begin{equation}
        \begin{aligned}
            \left| \mathbf{R}_{k}\left(x; \mathbf{D}_{\ell}, \hat{\eta}\right) \right|
            & = \left| \sum_{i \in I_{k}} \frac{\binom{n - \rk(x; X_{i}, \mathbf{D})}{s-1}}{\binom{n}{s}} \left(m(Z_{i}, \hat{\eta}_{k}) - m(Z_{i}, \eta_{0})\right) \right|
            \leq \sum_{i \in I_{k}} \frac{\binom{n - \rk(x; X_{i}, \mathbf{D})}{s-1}}{\binom{n}{s}} \left|m(Z_{i}, \hat{\eta}_{k}) - m(Z_{i}, \eta_{0}) \right|
        \end{aligned}
    \end{equation}
    Observe, that given $\mathbf{D}_{I_{k}^{C}}$ this is a sum of identically distributed but dependent random variables where the dependence arises from the interdependence of the weights assigned to each observation.
    We can provide an upper bound to this sum by removing the interdependence and assigning the highest possible weight to each observation.
    \begin{equation}\label{eq:equal_weight_ineq}
        \begin{aligned}
            \left| \mathbf{R}_{k}\left(x; \mathbf{D}_{\ell}, \hat{\eta}\right) \right| 
            & \leq \sum_{i \in I_{k}} \frac{\binom{n - 1}{s-1}}{\binom{n}{s}} \left|m(Z_{i}, \hat{\eta}_{k}) - m(Z_{i}, \eta_{0}) \right|
            = \frac{s}{n} \sum_{i \in I_{k}} \left|m(Z_{i}, \hat{\eta}_{k}) - m(Z_{i}, \eta_{0}) \right| \\
            %
            & = \frac{s}{k} \left(\frac{1}{m} \sum_{i \in I_{k}} \left|m(Z_{i}, \hat{\eta}_{k}) - m(Z_{i}, \eta_{0}) \right|\right)
        \end{aligned}
    \end{equation}
    Clearly, this is a very lax upper bound as the weights decay at a considerable rate and a more nuanced approach to bounding this term has the potential to improve the conditions obtained as part of this proof.
    Now we can take the ideas from \citet{chernozhukov_doubledebiased_2018} and apply part of the proof strategy of Theorem 3.1 to this sum.
    Thus observe the following.
    \begin{equation}
        \begin{aligned}
            \left| \mathbf{R}_{k}\left(x; \mathbf{D}_{\ell}, \hat{\eta}\right) \right| 
            & \leq \frac{s}{k} \cdot \frac{\mathcal{I}_{3,k} + \mathcal{I}_{4,k}}{\sqrt{m}}
        \end{aligned}
    \end{equation}
    Here, we use the following definitions.
    \begin{align}
        \mathcal{I}_{3,k} 
        & = \Big|\mathbb{G}_{m,k}\left[m(Z, \hat{\eta}_{k})\right] 
        - \mathbb{G}_{m,k}\left[m(Z, \eta_{0})\right]\Big|\\
        %
        \mathcal{I}_{4,k} 
        & := \sqrt{m} \cdot \left|
        \E_{Z}\left[m(Z, \hat{\eta}_{k}) \, \middle| \, \mathbf{D}_{I_{k}^{C}}\right]
        - \E_{Z}\left[m(Z, \eta_{0}) \right]\right|
    \end{align}
    Continuing along the lines of the original proof, we can then find the following conditional on the event $\mathcal{E}_{n}$.
    \begin{equation}
        \begin{aligned}
            \E_{D}\left[\mathcal{I}_{3,k}^{2} \, \middle| \, \mathbf{D}_{I_{k}^{C}}\right]
            & = \E_{D}\left[ \left(\mathbb{G}_{m,k}\left[m\left(Z, \hat{\eta}_{k}\right)\right] 
            - \mathbb{G}_{m,k}\left[m\left(Z, \eta_{0}\right)\right]\right)^{2} \, \middle| \, \mathbf{D}_{I_{k}^{C}}\right]
            = \E_{D}\left[\left|m\left(Z, \hat{\eta}_{k}\right) - m\left(Z, \eta_{0}\right)\right|^{2}
            \, \middle| \, \mathbf{D}_{I_{k}^{C}}\right]\\
            %
            & \overset{\mathcal{E}_{n}}{\leq} \sup_{\eta \in \mathcal{T}_{n}} 
             \E_{D}\left[\left|m\left(Z, \eta\right) - m\left(Z, \eta_{0}\right)\right|^{2}
            \, \middle| \, \mathbf{D}_{I_{k}^{C}}\right]\\
            %
            & \leq \sup_{\eta \in \mathcal{T}_{n}} 
            \E_{D}\left[\left|m\left(Z, \eta\right) - m\left(Z, \eta_{0}\right)\right|^{2}\right]
            \overset{\text{Asm \ref{asm:DDML_Rate_Cond}}}{=} (r_{n}^{\prime})^{2}
        \end{aligned}
    \end{equation}
    Using Lemma 6.1 from \citet{chernozhukov_doubledebiased_2018}, we can thus find the following.
    \begin{equation}
        \mathcal{I}_{3,k} = O_{p}(r_{n}^{\prime})
    \end{equation}
    Continuing with $\mathcal{I}_{4,k}$, introduce the following function.
    \begin{equation}
        \begin{aligned}
            f_{k}(r) 
            & := \E_{Z}\left[m\left(Z, \eta_{0} + r \left(\hat{\eta}_{k} - \eta_{0}\right)\right) \, \middle| \, \mathbf{D}_{I_{k}^{C}}\right] - \E_{Z}\left[m\left(Z, \eta_{0}\right) \right] 
            \quad \text{for} \quad r \in [0,1]
        \end{aligned}
    \end{equation}
    Then, using a Taylor expansion, we can rewrite as follows.
    \begin{equation}
        f_{k}(r) = f_{k}(0) + f_{k}^{\prime}(0) + \frac{1}{2} f_{k}^{\prime\prime}(\tilde{r})
        \quad \text{for some} \quad \tilde{r} \in (0,1)
    \end{equation}
    Observe now that $f_{k}(0) = 0$, and furthermore by Neyman-orthogonality $f_{k}^{\prime}(0) = 0$.
    Thus, $f_{k}(r) = \frac{1}{2} f_{k}^{\prime\prime}(\tilde{r})$ for some $\tilde{r} \in (0,1)$.
    On the even $\mathcal{E}_{n}$, we can furthermore observe the following.
    \begin{equation}
        \left| f_{k}^{\prime\prime}(\tilde{r}) \right| 
        \leq \sup_{r \in (0,1)} \left| f_{k}^{\prime\prime}(r) \right|
        \overset{\text{Asm \ref{asm:DDML_Rate_Cond}}}{\leq} \lambda_{n}^{\prime}
    \end{equation}
    Combining these findings, we obtain the following result.
    \begin{equation}
         \mathbf{R}_{k}\left(x; \mathbf{D}_{\ell}, \hat{\eta}\right)
         = O_{p}\left(s \cdot \left(n^{-1/2} r_{n}^{\prime} + \lambda_{n}^{\prime}\right)\right)
    \end{equation}
    Since $k$ is fixed, we thus obtain the desired result.
    \begin{equation}
        \sqrt{n} \cdot \mathbf{R}\left(x; \mathbf{D}_{\ell}, \hat{\eta}\right)
         = O_{p}\left(s \cdot \left(r_{n}^{\prime} + \sqrt{n} \cdot \lambda_{n}^{\prime}\right)\right)
    \end{equation}
    If the subsampling scale $s$ is of order $o\left(\min\left\{1/r_{n}^{\prime}, \left(\sqrt{n} \cdot \lambda_{n}^{\prime}\right)^{-1}\right\}\right)$ this term is thus asymptotically negligible.
\end{proof}

\newpage 

\begin{boxD}
    \begin{lem}[Behavior of Oracle-H\'ajek-Projection Error]\label{lem:ps_hajek_error}\mbox{}\\*
    {\color{red} LOREM IPSUM}    
\end{lem}
\end{boxD}

\begin{proof}[Proof of Lemma \ref{lem:ps_hajek_error}]\mbox{}\\*
    Consider first the average Oracle-error within a given fold $k$ and observe the following.
    \begin{equation}
        \begin{aligned}
            \bar{R}_{1, k}\left(x\right)
            & = \frac{1}{m}\sum_{l \in I_{k}}\tilde{R}_{1, k}\left(x, Z_{l}\right)
            % = \frac{1}{m}\sum_{l \in I_{k}}
            % \left(\chi_{s}^{(1)}\left(x; Z_l, \hat{\eta}_{k}\right) - \chi_{s,0}^{(1)}\left(x; Z_l\right)\right) \\
            % %
            = \frac{1}{m}\sum_{l \in I_{k}}
            \left(\vartheta_{s}^{1}\left(x; Z_l, \hat{\eta}_{k}\right)
            - \vartheta_{s,0}^{1}\left(x; Z_l\right) 
            \right)
            % %
            % & = \frac{1}{m}\sum_{l \in I_{k}}
            % \E_{D}\left[\chi_{s}(x; \mathbf{D}_{[s]}, \hat{\eta}_{k}) - \chi_{s,0}(x; \mathbf{D}_{[s]})\, \middle| \, Z_1 = Z_l \right]
            % %
            % & = \frac{1}{m s!}\sum_{l \in I_{k}}
            % \E_{D}\left[\sum_{i = 1}^{n}\kappa(x; Z_{i}, \mathbf{D}_{[s]})
            % \left(m\left(Z_{i}, \hat{\eta}_{k}\right) - m\left(Z_{i}, \eta_{0}\right)\right) \, \middle| \, Z_1 = Z_l \right]
            % %
            % & = \frac{1}{m s!}\sum_{l \in I_{k}}
            % \Bigg\{
            %     \underbrace{\E_{D}\left[\kappa(x; Z_{1}, \mathbf{D}_{[s]})m\left(Z_{1}, \hat{\eta}_{k}\right)\, \middle| \, Z_1 = Z_l \right]}_{(A_{l})}
            %     - \underbrace{m\left(Z_{l}, \eta_{0}\right)\E_{D}\left[\kappa(x; Z_{1}, \mathbf{D}_{[s]}) \, \middle| \, Z_1 = Z_l \right]}_{(B_{l})}  \\
            % & \quad \quad  + (s-1) \left( \underbrace{\E_{D}\left[\kappa(x; Z_{2}, \mathbf{D}_{[s]})m\left(Z_{2}, \hat{\eta}_{k}\right)\, \middle| \, Z_1 = Z_l \right]}_{(C_{l})}
            % - \underbrace{\E_{D}\left[\kappa(x; Z_{2}, \mathbf{D}_{[s]})m\left(Z_{2}, \eta_{0}\right) \, \middle| \, Z_1 = Z_l \right]}_{(D_{l})} \right)
            % \Bigg\}
        \end{aligned}
    \end{equation}
    Here, in analogy to step 3 in the proof of Theorem 3.1 in \citet{chernozhukov_doubledebiased_2018}, we can now observe the following which follows from the triangle inequality.
    \begin{equation}
        \begin{aligned}
            \left|\bar{R}_{1, k}\left(x\right)\right|
            & \leq \sqrt{\frac{1}{m}} \cdot \left(\mathcal{I}_{3,k}^{(1)} + \mathcal{I}_{4,k}^{(1)}\right)
        \end{aligned}
    \end{equation}
    where we use the following two definitions.
    \begin{align}
        \mathcal{I}_{3,k}^{(1)} 
        & := \left|\mathbb{G}_{m,k}\left[\vartheta_{s}^{1}\left(x; Z, \hat{\eta}_{k}\right)\right] 
        - \mathbb{G}_{m,k}\left[\vartheta_{s,0}^{1}\left(x; Z\right) \right]\right|\\
        %
        \mathcal{I}_{4,k}^{(1)} 
        & := \sqrt{m} \cdot \left|
        \E_{Z}\left[\vartheta_{s}^{1}\left(x; Z, \hat{\eta}_{k}\right) \, \middle| \, \mathbf{D}_{I_{k}^{C}}\right]
        - \E_{Z}\left[\vartheta_{s,0}^{1}\left(x; Z\right) \right]\right|
    \end{align}
    Notice that conditional on $\mathbf{D}_{I_{k}^{C}}$ the first-stage estimate $\hat{\eta}_{k}$ is non-stochastic.
    This allows us to make the following observation given the event $\mathcal{E}_{n}$ obtains.
    \begin{equation}
        \begin{aligned}
            \E_{D}\left[\left(\mathcal{I}_{3,k}^{(1)}\right)^{2} \, \middle| \, \mathbf{D}_{I_{k}^{C}}\right] 
            & = \E_{D}\left[ \left(\mathbb{G}_{m,k}\left[\vartheta_{s}^{1}\left(x; Z, \hat{\eta}_{k}\right)\right] 
            - \mathbb{G}_{m,k}\left[\vartheta_{s,0}^{1}\left(x; Z\right)\right]\right)^{2} \, \middle| \, \mathbf{D}_{I_{k}^{C}}\right]
            = \E_{D}\left[\left|\vartheta_{s}^{1}\left(x; Z, \hat{\eta}_{k}\right) - \vartheta_{s,0}^{1}\left(x; Z\right)\right|^{2}
            \, \middle| \, \mathbf{D}_{I_{k}^{C}}\right]\\
            %
            & \overset{\mathcal{E}_{n}}{\leq} \sup_{\eta \in \mathcal{T}_{n}} 
             \E_{D}\left[\left|\vartheta_{s}^{1}\left(x; Z, \eta\right) - \vartheta_{s,0}^{1}\left(x; Z\right)\right|^{2}
            \, \middle| \, \mathbf{D}_{I_{k}^{C}}\right]\\
            %
            & \leq \sup_{\eta \in \mathcal{T}_{n}} 
             \E_{D}\left[\left|\vartheta_{s}^{1}\left(x; Z, \eta\right) - \vartheta_{s,0}^{1}\left(x; Z\right)\right|^{2}\right]
             \overset{\text{Lem \ref{lem:rate_cond_errors1}}}{=} \left(\frac{r_{n}^{\prime}}{(s-1)!}\right)^{2}
        \end{aligned}
    \end{equation}
    By \citet{chernozhukov_doubledebiased_2018} Lemma 6.1, this allows us to draw the following conclusion.
    \begin{equation}
        \begin{aligned}
            \mathcal{I}_{3,k}^{(1)} = O_{p}\left(\frac{r_{n}^{\prime}}{(s-1)!}\right)
        \end{aligned}
    \end{equation}  
    Next, to bound $\mathcal{I}_{4,k}^{(1)}$, we introduce a function analogous to the proof presented in \citet{chernozhukov_doubledebiased_2018}.
    \begin{equation}
        \begin{aligned}
            f_{k}^{(1)}(r) 
            & := \E_{Z}\left[\vartheta_{s}^{1}\left(x; Z, \eta_{0} + r \left(\hat{\eta}_{k} - \eta_{0}\right)\right) \, \middle| \, \mathbf{D}_{I_{k}^{C}}\right] - \E_{Z}\left[\vartheta_{s,0}^{1}\left(x; Z\right) \right] 
            \quad \text{for} \quad r \in [0,1]
        \end{aligned}
    \end{equation}
    Using a Taylor expansion, we obtain the following for some $\tilde{r} \in (0,1)$.
    \begin{equation}
        f_{k}^{(1)}(1)
        = f_{k}^{(1)}(0) + \left[\frac{\partial}{\partial r}f_{k}^{(1)}(r)\right]_{r = 0} + \frac{1}{2}\left[\frac{\partial^{2}}{\partial r^{2}}f_{k}^{(1)}(r)\right]_{r = \tilde{r}}
    \end{equation}
    It can be easily recognized that $f_{k}^{(1)}(0) = 0$.
    Additionally, by Neyman orthogonality, we find that $\left[\frac{\partial}{\partial r}f_{k}^{(1)}(r)\right]_{r = 0} = 0$.
    Thus, on the event $\mathcal{E}_{n}$ we find the following.
    \begin{equation}
        \begin{aligned}
            \mathcal{I}_{4,k}^{(1)}
            & = \sqrt{n} \cdot f_{k}^{(1)}(1)
            = \sqrt{n} \cdot \frac{1}{2}\left[\frac{\partial^{2}}{\partial r^{2}}f_{k}^{(1)}(r)\right]_{r = \tilde{r}}
            \leq \sqrt{n} \cdot \sup_{x \in (0,1)} \frac{1}{2}\left[\frac{\partial^{2}}{\partial r^{2}}f_{k}^{(1)}(r)\right]_{r = x}
            \overset{\text{Lem \ref{lem:rate_cond_errors2}}}{\leq} {\color{red} LOREM IPSUM}
        \end{aligned}
    \end{equation}
    Combining these bounds, we obtain the following intermediate result.
    \begin{equation}
        \begin{aligned}
            \bar{R}_{1, k}\left(x\right)
            & = O_{p}\left(\sqrt{\frac{1}{n}} \cdot \frac{r_{n}^{\prime}}{(s-1)!} + {\color{red} LOREM IPSUM}\right)
        \end{aligned}
    \end{equation}
    Now, recognizing that since $k$ is fixed, we can apply this logic to each individual fold, we obtain the desired result.
    \begin{equation}
        \begin{aligned}
            \frac{s}{K} \sum_{k = 1}^{K} \bar{R}_{1, k}\left(x\right) 
            = O_{p}\left(\sqrt{\frac{1}{n}} \cdot \frac{s \cdot r_{n}^{\prime}}{(s-1)!} + {\color{red} LOREM IPSUM}\right)
        \end{aligned}
    \end{equation}
{\color{red} LOREM IPSUM}
\end{proof}

\newpage

To obtain results on the asymptotic distribution of the proposed estimator it remains to analyze the behavior of the higher-order error terms due to the first-stage estimation error.
Thus, similar to the analysis presented in Lemma \ref{lem:ps_hajek_error}, we obtain the following result.

\begin{boxD}
    \begin{lem}[Behavior of Higher-Order Error Terms]\label{lem:HO_error}\mbox{}\\*
        {\color{red} LOREM IPSUM}
    \end{lem}    
\end{boxD}


\begin{proof}[Proof of Lemma \ref{lem:HO_error}]\mbox{}\\*
    Consider a data set $\mathbf{D}_{[n]}$ and an arbitrary subset $\ell \subseteq [n]$ such that $1 < \left|\ell\right| \leq s$.
    Recognize that for each such subset there is exactly one observation $Z^{*}\left(\mathbf{D}_{\ell}\right)$ in $\mathbf{D}_{\ell}$ that is closest to the point of interest $x$.
    Let $k^{*}\left(\mathbf{D}_{\ell}\right) \in [K]$ denote the fold to which this closest observation belongs.
    Now, define the following random variables for $j = 2, \dotsc, s$ and $k \in [K]$.
    \begin{equation}
        m_{k}^{(j)}
        = \sum_{\ell \in L_{n,j}} \1\left(k^{*}\left(\mathbf{D}_{\ell}\right) = k\right) 
        \quad \text{and} \quad
        \mathbf{M}_{k}^{(j)}
        = \left\{\ell \in L_{n,j} \, \middle| \,  \1\left(k^{*}\left(\mathbf{D}_{\ell}\right) = k\right) \right\}
    \end{equation}
    Choose now an arbitrary $k \in [K]$ and consider the following argument.
    \begin{equation}
        \begin{aligned}
            \bar{R}_{j, k}\left(x\right) 
            & := \binom{s}{j}\binom{n}{j}^{-1} \sum_{\ell \in \mathbf{M}_{k}^{(j)}} \left(\vartheta_{s}^{j}(x; \mathbf{D}_{\ell}, \hat{\eta}_{k}) - \vartheta_{s,0}^{j}(x; \mathbf{D}_{\ell})\right) \\
            %
            & = \binom{s}{j}\binom{n}{j}^{-1} \sum_{\ell \in \mathbf{M}_{k}^{(j)}} \E_{D}\left[\chi_{s}(x; \mathbf{D}_{[s]}, \hat{\eta}_{k}) - \chi_{s,0}(x; \mathbf{D}_{[s]}) \, \middle| \, \mathbf{D}_{[j]} = \mathbf{D}_{\ell}\right] \\
            %
            & = \binom{s}{j}\binom{n}{j}^{-1} \sum_{\ell \in \mathbf{M}_{k}^{(j)}} \E_{D}\left[
                \sum_{i = 1}^{s}\frac{\1\left(\kappa(x; Z_{i}, \mathbf{D}_{[s]}) = 1\right)}{s!} \left(m(Z_{i}, \hat{\eta}_{k_i}) - m(Z_{i}, \eta_{0})\right)
            \, \middle| \, \mathbf{D}_{[j]} = \mathbf{D}_{\ell}\right] \\
            %
            & = \binom{s}{j}\binom{n}{j}^{-1}  \sum_{\ell \in \mathbf{M}_{k}^{(j)}} \Bigg\{\E_{D}\left[
                \sum_{i = 1}^{j}\frac{\1\left(\kappa(x; Z_{i}, \mathbf{D}_{[s]}) = 1\right)}{s!} \left(m(Z_{i}, \hat{\eta}_{k}) - m(Z_{i}, \eta_{0})\right)
                \, \middle| \, \mathbf{D}_{[j]} = \mathbf{D}_{\ell}\right]\\
                & \quad \quad + \E_{D}\left[\sum_{i = j + 1}^{s}\frac{\1\left(\kappa(x; Z_{i}, \mathbf{D}_{[s]}) = 1\right)}{s!} \left(m(Z_{i}, \hat{\eta}_{k_i}) - m(Z_{i}, \eta_{0})\right)
            \, \middle| \, \mathbf{D}_{[j]} = \mathbf{D}_{\ell}\right] \Bigg\}\\
            %
            & = \binom{s}{j}\binom{n}{j}^{-1} \sum_{\ell \in \mathbf{M}_{k}^{(j)}} \Bigg\{\frac{1}{s!} \E_{D}\left[
                m(Z^{*}\left(\mathbf{D}_{\ell}\right), \hat{\eta}_{k}) - m(Z^{*}\left(\mathbf{D}_{\ell}\right), \eta_{0})\right]\\
                & \quad \quad + \frac{s-j}{s!}\E_{D}\left[\1\left(\kappa(x; Z_{s}, \mathbf{D}_{[s]}) = 1\right) \left(m(Z_{s}, \hat{\eta}_{k_s}) - m(Z_{s}, \eta_{0})\right)
            \, \middle| \, \mathbf{D}_{[j]} = \mathbf{D}_{\ell}\right] \Bigg\}\\
        \end{aligned}
    \end{equation}
\end{proof}