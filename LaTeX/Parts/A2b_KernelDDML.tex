\subsection{Consequences of DDML Rate-Assumptions}
\hrule

Recall first the conditions imposed in Assumption \ref{asm:DDML_Rate_Cond}.
\begin{align}
    m_{n} 
    & :=\sup_{\eta \in \mathcal{T}_{n}}\left(\E_{Z}\left[\left|m\left(Z; \theta_0, \eta\right)\right|^q\right]\right)^{1 / q}
    \leq c_1\\
    %
    r_{n}^{\prime}
    & :=\sup_{\eta \in \mathcal{T}_{n}}\left(\E_{Z}\left[\left|m\left(Z; \theta_0, \eta\right)-m\left(Z; \theta_0, \eta_0\right)\right|^2\right]\right)^{1 / 2} 
    \leq \delta_{n}\\
    %
    \lambda_{n}^{\prime}
    & :=\sup_{r \in(0,1), \eta \in \mathcal{T}_{n}}\left|\partial_r^2 \E_{Z}\left[m\left(Z; \theta_0, \eta_0+r\left(\eta-\eta_0\right)\right)\right]\right| 
    \leq \delta_{n} / \sqrt{n}
\end{align}

We are interested in how these conditions translate into statements on the Oracle-Hoeffding decomposition errors.
Thus, recall the form of the error terms as introduced in Equation \ref{eq:DNNDML2_ResidDecomp}.
\begin{equation}
    R_{c}\left(x; \mathbf{D}_{\ell}\right) 
    = \chi_{s}^{(c)}\left(x; \mathbf{D}_{\ell}, \hat{\eta}\right) - \chi_{s,0}^{(c)}\left(x; \mathbf{D}_{\ell}\right)
\end{equation}

\begin{lem}[]\label{lem:rate_cond_errors}\mbox{}\\*
    {\color{red} LOREM IPSUM}
\end{lem}

\begin{proof}[Proof of Lemma \ref{lem:rate_cond_errors}]\mbox{}\\*
    Consider the following argument.
    \begin{equation}
        \begin{aligned}
            & \sup_{\eta \in \mathcal{T}_{n}}\left(
                \E_{Z}\left[
                    \left|\chi_{s}^{(1)}\left(x; Z, \eta\right) 
                    - \chi_{s,0}^{(1)}\left(x; Z\right)\right|^2
                \right]
            \right)^{1 / 2} \\
            %
            & \quad = \sup_{\eta \in \mathcal{T}_{n}}\left(
                \E_{Z}\left[\left|\vartheta_{s}^{1}\left(x; Z, \eta\right)
            - \E_{D}\left[\chi_{s}(x; \mathbf{D}_{[s]}, \eta)\right]
            - \vartheta_{s,0}^{1}\left(x; Z\right) 
            + \E_{D}\left[\chi_{s,0}\left(x; \mathbf{D}_{[s]}\right)\right]\right|^{2}\right]
            \right)^{1 / 2}
        \end{aligned}
    \end{equation}
    {\color{red} LOREM IPSUM}
\end{proof}