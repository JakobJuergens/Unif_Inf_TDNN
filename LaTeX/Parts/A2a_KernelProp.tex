\section{Useful Results}
\hrule

\subsection{Properties of the $\kappa$ Function}
\hrule

\begin{lem}[\citet{demirkaya_optimal_2024} - Lemma 12]\label{lem:dem12}\mbox{}\\*
	Let $D = \{Z_1, \dotsc, Z_s\}$ an i.i.d.\ sample drawn from $P$.
	The indicator functions $\kappa\left(x; Z_{i}, D\right)$ satisfy the following properties.
	\begin{enumerate}
		\item For any $i \neq j$, we have $\kappa\left(x; Z_{i}, D\right) \kappa\left(x;
			      Z_{j}, D\right)=0$ with probability one;
		\item $\sum_{i=1}^{s} \kappa\left(x; Z_{i}, D\right)=1$;
		\item $\forall i \in [s]: \quad \E_{1:s}\left[\kappa\left(x; Z_{i}, D\right)\right]=s^{-1}$
		\item $\E_{2: s}\left[\kappa\left(x; Z_1, D\right)\right]
			      = \left\{1-\varphi\left(B\left(x,\left\|X_1-x\right\|\right)\right)\right\}^{s-1}$
	\end{enumerate}
	Here $\E_{i: s}$ denotes the expectation with respect to $\left\{Z_{i}, Z_{i+1}, \dotsc, Z_s\right\}$.
	Furthermore, $\varphi$ denotes the probability measure on $\mathbb{R}^{d}$ induced by the random vector $X$.
\end{lem}

\hrule

\begin{lem}[\citet{demirkaya_optimal_2024} - Lemma 13]\label{lem:dem13}\mbox{}\\*
	For any $L^1$ function $f$ that is continuous at $x$, it holds that
	\begin{equation}
		\lim _{s \rightarrow \infty} \E_{1}\left[f\left(X_1\right) s \E_{2:s}\left[\kappa(x; Z_1, D)\right]\right]
		= f(x).
	\end{equation}
\end{lem}

\hrule

As an additional tool, we will make use of the following analogous results concerning products of two kernel functions with nonzero expectation.
These results will then be used to construct an analogon of Lemma \ref{lem:dem13} for the corresponding cases.
This will serve very similar purposes in the analysis of (conditional) covariance terms as the previous results serve for (conditional) expectations.
\vspace{0.5cm}
\hrule

\begin{lem}[]\label{lem:double_cond1}\mbox{}\\*
    Fix sample size $n$, subsampling scale $s$, and $c$ such that $0 < c \leq s \leq n$.
	Let $D = \left\{Z_1, Z_2, \dotsc, Z_c, Z_{c+1}, \dotsc Z_s \right\}$ be an i.i.d.\ data set drawn from $P$ as described in Setup~\ref{asm:npr_dgp}.
	Let $D^{\prime} = \left\{Z_1, Z_2, \dotsc, Z_c, Z_{c+1}^{\prime}, \dotsc Z_s^{\prime} \right\}$ be a second data set that shares the first $c$ observations with $D$.
	The remaining $s - c$ observations of $D^{\prime}$, i.e.\ $\left\{Z_{c+1}^{\prime}, \dotsc Z_s^{\prime} \right\}$, are i.i.d.\ draws from $P$ that are independent of $D$.

    Then the following three equalities hold.
    \begin{equation}
        \forall i \in [c]: \quad
        \E_{D, D^{\prime}}\left[\kappa\left(x; Z_{i}, D\right)\kappa\left(x; Z_{i}, D^{\prime}\right)\right]
        = (2s - c)^{-1}
    \end{equation}
    \begin{equation}
        \forall i \in [c] \; \forall j \in \{c+1, \dotsc, s\}: \quad
        \E_{D, D^{\prime}}\left[\kappa\left(x; Z_{i}, D\right)\kappa\left(x; Z_{j}^{\prime}, D^{\prime}\right)\right]
        = \frac{(2s - c - 2)!}{(2s - c)!}\sum_{i = 0}^{s - c - 1}\binom{s - c - 1}{i}\binom{2s - c - 2}{i}^{-1}
    \end{equation}
    \begin{equation}
        \forall i,j \in \{c+1, \dotsc, s\}: \quad
        \E_{D, D^{\prime}}\left[\kappa\left(x; Z_{i}, D\right)\kappa\left(x; Z_{j}^{\prime}, D^{\prime}\right)\right]
        = \frac{2(2s - c - 1)!}{(2s-c)!} \sum_{i = 0}^{s-c-1} \binom{s - c - 1}{i}\binom{2s - c - 1}{s-1+i}^{-1}
    \end{equation}
\end{lem}

\begin{proof}[Proof of Lemma \ref{lem:double_cond1}]\mbox{}\\*
    Without loss of generality, we will consider the cases of $i = 1$ and $j = c + 1$ for the first two equations.
    \begin{equation}
        \begin{aligned}
            \E_{D, D^{\prime}}\left[\kappa\left(x; Z_{1}, D\right)\kappa\left(x; Z_{1}, D^{\prime}\right)\right]
            & = \E_{D, D^{\prime}}\left[\kappa\left(x; Z_{1}, D_{1:c}\right)\kappa\left(x; Z_{1}, D_{(c+1):s}\right)\kappa\left(x; Z_{1}, D^{\prime}_{(c+1):s}\right)\right]\\
            %
            & = \E\left[\kappa\left(x; Z_{1}, D_{[2s - c]}\right)\right]
            = (2s - c)^{-1}
        \end{aligned}
    \end{equation}
    Considering the second case, we find the following.
    \begin{equation}
        \begin{aligned}
            & \E_{D, D^{\prime}}\left[\kappa\left(x; Z_{1}, D\right)\kappa\left(x; Z_{c+1}^{\prime}, D^{\prime}\right)\right]
            = \frac{1}{(2s - c)!} \sum_{i = 0}^{s - c - 1}\binom{s - c - 1}{i} i! \left((s - 1) + (s - c - 1 - i)\right)! \\
            %
            & \quad = \frac{1}{(2s - c)!}\sum_{i = 0}^{s - c - 1}\binom{s - c - 1}{i} i! \left(2s - c - 2 - i\right)! 
            = \frac{(2s - c - 2)!}{(2s - c)!}\sum_{i = 0}^{s - c - 1}\binom{s - c - 1}{i}\binom{2s - c - 2}{i}^{-1}
        \end{aligned}
    \end{equation}
    While unintuitive at first, the terms in this expression have intuitive meaning when we consider this as a combinatoric problem.
    Consider lining up the observations in order of their distance to the point of interest and counting the cases for which the expression in the expectation is equal to one.
    First, there are $(2s-c)!$ possible orderings of the observations with probability one, leading to the denominator.
    Next, notice that only those orderings where $\|X_{c+1}^{\prime} - x\| \leq \|X_{1}-x\|$ and $\|X_{1} - x\| \leq \|X_{i} - x\|$ for any $i = 2, \dotsc, c$ can possibly lead to a non-zero realization of the kernel term.
    Furthermore, out of the $(s-c-1)$ observations in $D^{\prime}_{(c+2):s}$, it is possible for $i = 0, \dotsc, s-c-1$ observations to lie at a distance to the point of interest that is smaller than $\|X_{1}-x\|$ but larger than $\|X_{c+1}^{\prime} - x\|$ in any permutation.
    The sum adjusts for those possible configurations.
    
    Considering the third case, without loss of generality, we consider the case of $i = j = c+1$.
    We find the following.
    \begin{equation}
        \begin{aligned}
            & \E_{D, D^{\prime}}\left[\kappa\left(x; Z_{c+1}, D\right)\kappa\left(x; Z_{c+1}^{\prime}, D^{\prime}\right)\right]\\
            %
            & \quad  = \E_{D, D^{\prime}}\left[
                \kappa\left(x; Z_{c+1}, D_{1:c}\right)\kappa\left(x; Z_{c+1}^{\prime}, D_{1:c}\right)
                \kappa\left(x; Z_{c+1}, D_{(c+1):s}\right)\kappa\left(x; Z_{c+1}^{\prime}, D_{(c+1):s}^{\prime}\right)
            \right]\\
            %
            & \quad = \frac{2}{(2s - c)!} \sum_{i = 0}^{s-c-1} \binom{s - c - 1}{i}(s - 1 + i)!(s-c-i)! \\
            %
            & \quad = \frac{2(2s - c - 1)!}{(2s-c)!} \sum_{i = 0}^{s-c-1} \binom{s - c - 1}{i}\binom{2s - c - 1}{s-1+i}^{-1}
        \end{aligned}
    \end{equation}
    The third case follows from a similar combinatorial logic as the second.
    We consider without loss of generality the case that $\|X_{c+1}^{\prime} - x\| \leq \|X_{c+1}-x\|$ and adjust for this fact by multiplying the whole expression by two.
    Notice now that any number $i = 0, \dotsc, s-c-1$ of observations in $D^{\prime}_{(c+2):s}$ can be farther away from $x$ than $X_{c+1}$ or at a distance that is between $\|X_{c+1}^{\prime} - x\|$ and $\|X_{c+1}-x\|$.
    The summation adjusts for all possible permutations that fulfill this criterion.
\end{proof}

\hrule

\newpage
\begin{lem}[]\label{lem:double_cond2}\mbox{}\\*
    Fix sample size $n$, subsampling scale $s$, and $c$ such that $0 < c \leq s \leq n$.
	Let $D = \left\{Z_1, Z_2, \dotsc, Z_c, Z_{c+1}, \dotsc Z_s \right\}$ be an i.i.d.\ data set drawn from $P$ as described in Setup~\ref{asm:npr_dgp}.
	Let $D^{\prime} = \left\{Z_1, Z_2, \dotsc, Z_c, Z_{c+1}^{\prime}, \dotsc Z_s^{\prime} \right\}$ be a second data set that shares the first $c$ observations with $D$.
	The remaining $s - c$ observations of $D^{\prime}$, i.e.\ $\left\{Z_{c+1}^{\prime}, \dotsc Z_s^{\prime} \right\}$, are i.i.d.\ draws from $P$ that are independent of $D$.

    Then, the following statements hold.
    \begin{equation}
        \begin{aligned}
            & \forall i \in [c] \; \forall j \in \{c+1, \dotsc, s\}: \quad
            \E\left[\kappa\left(x; Z_{i}, D\right)\kappa\left(x; Z_{j}^{\prime}, D^{\prime}\right) \; \middle| \; X_i, X_{j}^{\prime}\right]\\
            %
            & \quad = \1\left(\|X_{j}^{\prime} - x\| \leq \|X_{i} - x\|\right)
            \cdot \left\{1-\varphi\left(B\left(x,\left\|X_{i}-x\right\|\right)\right)\right\}^{s-1} 
            \cdot \left\{1-\varphi\left(B\left(x,\left\|X_{j}^{\prime}-x\right\|\right)\right)\right\}^{s-c-1}
        \end{aligned}
    \end{equation}
    \begin{equation}
        \begin{aligned}
            & \forall i,j \in \{c+1, \dotsc, s\}: \quad
            \E\left[\kappa\left(x; Z_{i}, D\right)\kappa\left(x; Z_{j}^{\prime}, D^{\prime}\right) \; \middle| \; X_{i}, X_{j}^{\prime}\right] \\ 
            %
            & \quad = \left\{1-\varphi\left(B\left(x, \min\left(\left\|X_{i} - x\right\|, \left\|X_{j}^{\prime}-x\right\|\right)\right)\right)\right\}^{s-c-1}
            \cdot \left\{1-\varphi\left(B\left(x,\max\left(\left\|X_{i} - x\right\|, \left\|X_{j}^{\prime}-x\right\|\right)\right)\right)\right\}^{s-1}
        \end{aligned}
    \end{equation}
\end{lem}

\begin{proof}[Proof of Lemma \ref{lem:double_cond2}]\mbox{}\\*
    Without loss of generality, we will consider the cases of $i = 1$ and $j = c + 1$ for the first equation.
    \begin{equation}
    \begin{aligned}
        & \E\left[\kappa\left(x; Z_{1}, D\right)\kappa\left(x; Z_{c+1}^{\prime}, D^{\prime}\right) \; \middle| \; X_1, X_{c+1}^{\prime}\right]\\
        %
        & = \E\left[
                \E\left[
                    \kappa\left(x; Z_{1}, D_{1:c}\right)
                    \kappa\left(x; Z_{1}, D_{(c+1):s}\right)
                    \kappa\left(x; Z_{c+1}^{\prime}, D_{1:c}\right) 
                    \kappa\left(x; Z_{c+1}^{\prime}, D^{\prime}_{(c+1):s}\right) 
                \; \middle| \; X_1, \dotsc, X_{c}, X_{c+1}^{\prime}\right] 
            \; \middle| \; X_1, X_{c+1}^{\prime}\right] \\
		%
        & = \E\left[
                \E\left[
                    \kappa\left(x; Z_{1}, D_{(c+1):s}\right)
                    \cdot \kappa\left(x; Z_{c+1}^{\prime}, D^{\prime}_{(c+1):s}\right)
                \; \middle| \; X_1, \dotsc, X_{c}, X_{c+1}^{\prime}\right] 
                \cdot \kappa\left(x; Z_{1}, D_{1:c}\right)
                \cdot \kappa\left(x; Z_{c+1}^{\prime}, D_{1:c}\right)
            \; \middle| \; X_1, X_{c+1}^{\prime}\right] \\
        %
        & = \E\left[
                \E\left[
                    \kappa\left(x; Z_{1}, D_{(c+1):s}\right)
                    \cdot \kappa\left(x; Z_{c+1}^{\prime}, D^{\prime}_{(c+1):s}\right)
                \; \middle| \; X_1, X_{c+1}^{\prime}\right] 
                \cdot \kappa\left(x; Z_{1}, D_{1:c}\right)
                \cdot \kappa\left(x; Z_{c+1}^{\prime}, D_{1:c}\right)
            \; \middle| \; X_1, X_{c+1}^{\prime}\right] \\
        %
        & = \E\left[
                    \kappa\left(x; Z_{1}, D_{(c+1):s}\right)
                    \cdot \kappa\left(x; Z_{c+1}^{\prime}, D^{\prime}_{(c+1):s}\right)
            \; \middle| \; X_1, X_{c+1}^{\prime}\right] 
            \cdot \E\left[
                \kappa\left(x; Z_{1}, D_{1:c}\right)
                \cdot \kappa\left(x; Z_{c+1}^{\prime}, D_{1:c}\right)
            \; \middle| \; X_1, X_{c+1}^{\prime}\right] \\
        %
        & = \E\left[\kappa\left(x; Z_{1}, D_{(c+1):s}\right)\; \middle| \; X_1\right] 
            \cdot \E\left[\kappa\left(x; Z_{c+1}^{\prime}, D^{\prime}_{(c+1):s}\right)
            \; \middle| \; X_{c+1}^{\prime}\right] 
            \cdot \1\left(\|X_{c+1}^{\prime} - x\| \leq \|X_{1} - x\|\right)
            \cdot \E\left[
                \kappa\left(x; Z_{1}, D_{1:c}\right)
            \; \middle| \; X_1\right] \\
        %
        & = \1\left(\|X_{c+1}^{\prime} - x\| \leq \|X_{1} - x\|\right)
        \cdot \E\left[\kappa\left(x; Z_{1}, D\right)\; \middle| \; X_1\right] 
        \cdot \E\left[\kappa\left(x; Z_{c+1}^{\prime}, D^{\prime}_{(c+1):s}\right)
            \; \middle| \; X_{c+1}^{\prime}\right] \\
        %
        & = \1\left(\|X_{c+1}^{\prime} - x\| \leq \|X_{1} - x\|\right)
            \cdot \left\{1-\varphi\left(B\left(x,\left\|X_1-x\right\|\right)\right)\right\}^{s-1}
            \cdot \left\{1-\varphi\left(B\left(x,\left\|X_{c+1}^{\prime}-x\right\|\right)\right)\right\}^{s-c-1}
    \end{aligned}
    \end{equation}
    For the second case, without loss of generality, we consider the case of $i = j = c+1$.
    \begin{equation}
		\begin{aligned}
			& \E\left[
            \kappa\left(x; Z_{c+1}, D\right)
            \kappa\left(x; Z_{c+1}^{\prime}, D^{\prime}\right) \; \middle| \; X_{c+1}, X_{c+1}^{\prime}\right] \\
            % 
			& \quad = \E\left[
                \E\left[
                    \kappa\left(x; Z_{c+1}, D\right)
                    \kappa\left(x; Z_{c+1}^{\prime}, D^{\prime}\right) 
                \; \middle| \; X_{1}, \dotsc, X_{c}, X_{c+1}, X_{c+1}^{\prime}\right]
                \; \middle| \; X_{c+1}, X_{c+1}^{\prime} \right]\\
            % 
			& \quad = \E\left[
            \E\left[
                \kappa\left(x; Z_{c+1}, D_{1:(c+1)}\right)
                \kappa\left(x; Z_{c+1}^{\prime}, D^{\prime}_{1:(c+1)}\right) \right. \right. \\
                & \quad \quad \left. \left.
                \kappa\left(x; Z_{c+1}, D_{(c+1):s}\right)
                \kappa\left(x; Z_{c+1}^{\prime}, D^{\prime}_{(c+1):s}\right)
                \; \middle| \; X_{1}, \dotsc, X_{c}, X_{c+1}, X_{c+1}^{\prime}\right]
                \; \middle| \; X_{c+1}, X_{c+1}^{\prime} \right]\\
            % 
            & \quad = \E\left[
            \E\left[
                \kappa\left(x; Z_{c+1}, D_{1:(c+1)}\right)
                \kappa\left(x; Z_{c+1}^{\prime}, D^{\prime}_{1:(c+1)}\right) \; \middle| \; X_{1}, \dotsc, X_{c}, X_{c+1}, X_{c+1}^{\prime}\right] \right. \\
                & \quad \quad  \left.
                \kappa\left(x; Z_{c+1}, D_{(c+1):s}\right)
                \kappa\left(x; Z_{c+1}^{\prime}, D^{\prime}_{(c+1):s}\right)
                \; \middle| \; X_{c+1}, X_{c+1}^{\prime} \right]\\
            % 
			& \quad = \E\left[
                \kappa\left(x; Z_{c+1}, D_{1:(c+1)}\right)
                \kappa\left(x; Z_{c+1}^{\prime}, D^{\prime}_{1:(c+1)}\right)
                \; \middle| \; X_{c+1}, X_{c+1}^{\prime}\right] \\
                & \quad \quad  
                \cdot \E\left[\kappa\left(x; Z_{c+1}, D_{(c+1):s}\right)\; \middle| \; X_{c+1}\right]
                \cdot \E\left[\kappa\left(x; Z_{c+1}^{\prime}, D^{\prime}_{(c+1):s}\right)\; \middle| \; X_{c+1}^{\prime} \right]
		\end{aligned}
	\end{equation}
    Without loss of generality, consider the case that $\|X_{c+1} - x\| \leq \|X_{c+1}^{\prime} - x\|$.
    \begin{equation}
        \E\left[
            \kappa\left(x; Z_{c+1}, D_{1:(c+1)}\right)
            \kappa\left(x; Z_{c+1}^{\prime}, D^{\prime}_{1:(c+1)}\right)
            \; \middle| \; X_{c+1}, X_{c+1}^{\prime}\right] 
        = \E\left[
            \kappa\left(x; Z_{c+1}^{\prime}, D^{\prime}_{1:(c+1)}\right)
            \; \middle| \; X_{c+1}^{\prime}\right]
    \end{equation}
    Furthermore, observe the following.
    \begin{equation}
        \E\left[\kappa\left(x; Z_{c+1}^{\prime}, D^{\prime}_{1:(c+1)}\right) \; \middle| \; X_{c+1}^{\prime}\right]
        \cdot \E\left[\kappa\left(x; Z_{c+1}^{\prime}, D^{\prime}_{(c+1):s}\right)\; \middle| \; X_{c+1}^{\prime} \right]
        = \E\left[\kappa\left(x; Z_{c+1}^{\prime}, D^{\prime}\right)\; \middle| \; X_{c+1}^{\prime} \right]
    \end{equation}
    Thus, we can find the following.
    \begin{equation}
        \begin{aligned}
            & \E\left[\kappa\left(x; Z_{1}, D\right)\kappa\left(x; Z_{c+1}^{\prime}, D^{\prime}\right) \; \middle| \; X_1, X_{c+1}^{\prime}\right]\\
            & \quad = \1\left(\|X_{c+1}^{\prime} - x\| \leq \|X_{c+1} - x\|\right)
            \cdot \left\{1-\varphi\left(B\left(x,\left\|X_{c+1} - x\right\|\right)\right)\right\}^{s-1}
            \cdot \left\{1-\varphi\left(B\left(x,\left\|X_{c+1}^{\prime}-x\right\|\right)\right)\right\}^{s-c-1}\\
            & \quad \quad + \1\left(\|X_{c+1}^{\prime} - x\| > \|X_{c+1} - x\|\right)
            \cdot \left\{1-\varphi\left(B\left(x,\left\|X_{c+1} - x\right\|\right)\right)\right\}^{s-c-1}
            \cdot \left\{1-\varphi\left(B\left(x,\left\|X_{c+1}^{\prime}-x\right\|\right)\right)\right\}^{s-1}\\
            %
            & \quad = \left\{1-\varphi\left(B\left(x, \min\left(\left\|X_{c+1} - x\right\|, \left\|X_{c+1}^{\prime}-x\right\|\right)\right)\right)\right\}^{s-c-1}
            \cdot \left\{1-\varphi\left(B\left(x,\max\left(\left\|X_{c+1} - x\right\|, \left\|X_{c+1}^{\prime}-x\right\|\right)\right)\right)\right\}^{s-1}
        \end{aligned}
    \end{equation}
\end{proof}

\hrule

\begin{lem}[]\label{lem:double_cond3}\mbox{}\\*
    Fix sample size $n$, subsampling scale $s$, and $c$ such that $0 < c \leq s \leq n$.
	Let $D = \left\{Z_1, Z_2, \dotsc, Z_c, Z_{c+1}, \dotsc Z_s \right\}$ be an i.i.d.\ data set drawn from $P$ as described in Setup~\ref{asm:npr_dgp}.
	Let $D^{\prime} = \left\{Z_1, Z_2, \dotsc, Z_c, Z_{c+1}^{\prime}, \dotsc Z_s^{\prime} \right\}$ be a second data set that shares the first $c$ observations with $D$.
	The remaining $s - c$ observations of $D^{\prime}$, i.e.\ $\left\{Z_{c+1}^{\prime}, \dotsc Z_s^{\prime} \right\}$, are i.i.d.\ draws from $P$ that are independent of $D$.

    For any $L^1$ function $f$ that is continuous at $x$, it holds that
        \begin{equation}
        \lim_{s \rightarrow \infty} \underbrace{\E_{1}\left[f^{2}(X_{1}) (2s - c) 
            \E_{2:s}\left[\kappa\left(x; Z_{1}, D\right)\kappa\left(x; Z_{1}, D^{\prime}\right)\right]
        \right]}_{(A)}
        = f^{2}(x)
    \end{equation}
    \begin{equation}
        \lim_{s \rightarrow \infty} \underbrace{\E_{1, (c+1)^{\prime}}\left[
            f(X_{1}) f(X_{c+1}^{\prime})
            \cdot \frac{\E_{D, D^{\prime}}\left[\kappa\left(x; Z_{1}, D\right)\kappa\left(x; Z_{c+1}^{\prime}, D^{\prime}\right) \; \middle| \; Z_{1}, Z_{c+1}^{\prime}\right]}{\E_{D, D^{\prime}}\left[\kappa\left(x; Z_{1}, D\right)\kappa\left(x; Z_{c+1}^{\prime}, D^{\prime}\right)\right]}
        \right]}_{(B)}
        = f^{2}(x)
    \end{equation}
    \begin{equation}
        \lim_{s \rightarrow \infty} \underbrace{\E_{c+1}\left[
            f(X_{c+1}) f(X_{c+1}^{\prime})
            \cdot \frac{\E_{D, D^{\prime}}\left[\kappa\left(x; Z_{c+1}, D\right)\kappa\left(x; Z_{c+1}^{\prime}, D^{\prime}\right) \; \middle| \; Z_{c+1}, Z_{c+1}^{\prime}\right]}{\E_{D, D^{\prime}}\left[\kappa\left(x; Z_{c+1}, D\right)\kappa\left(x; Z_{c+1}^{\prime}, D^{\prime}\right)\right]} 
        \right]}_{(C)}
        = f^{2}(x)
    \end{equation}
\end{lem}

\newpage
\begin{proof}[Proof of Lemma \ref{lem:double_cond3}]\mbox{}\\*
    We will largely argue along the same lines as the original proof in \citet{demirkaya_optimal_2024}.
    Thus, consider first the following inequalities.
    \begin{equation}
        \begin{aligned}
            \left|(A) - f^{2}(x)\right| 
            & = \left|\E_{1}\left[f^{2}(X_{1}) (2s - c) 
               \E_{2:s}\left[\kappa\left(x; Z_{1}, D\right)\kappa\left(x; Z_{1}, D^{\prime}\right)\right]
            \right] - f^{2}(x)\right| \\
            & \leq \E_{1}\left[\left|f^{2}(X_{1}) - f^{2}(x) \right| (2s - c) 
                \E_{2:s}\left[\kappa\left(x; Z_{1}, D\right)\kappa\left(x; Z_{1}, D^{\prime}\right)\right]
            \right]
        \end{aligned}
    \end{equation}
    \begin{equation}
        \begin{aligned}
            \left|(B) - f^{2}(x)\right| 
        & = \left|\E_{1, (c+1)^{\prime}}\left[
            f(X_{1}) f(X_{c+1}^{\prime})
            \cdot \frac{\E_{D, D^{\prime}}\left[\kappa\left(x; Z_{1}, D\right)\kappa\left(x; Z_{c+1}^{\prime}, D^{\prime}\right) \; \middle| \; Z_{1}, Z_{c+1}^{\prime}\right]}{\E_{D, D^{\prime}}\left[\kappa\left(x; Z_{1}, D\right)\kappa\left(x; Z_{c+1}^{\prime}, D^{\prime}\right)\right]}
        \right] - f^{2}(x)\right|\\
        & \leq \E_{1, (c+1)^{\prime}}\left[
            \left| f(X_{1}) f(X_{c+1}^{\prime}) - f^{2}(x)\right|
            \cdot \frac{\E_{D, D^{\prime}}\left[\kappa\left(x; Z_{1}, D\right)\kappa\left(x; Z_{c+1}^{\prime}, D^{\prime}\right) \; \middle| \; Z_{1}, Z_{c+1}^{\prime}\right]}{\E_{D, D^{\prime}}\left[\kappa\left(x; Z_{1}, D\right)\kappa\left(x; Z_{c+1}^{\prime}, D^{\prime}\right)\right]}
        \right]
        \end{aligned}
    \end{equation}
    \begin{equation}
        \begin{aligned}
            \left|(C) - f^{2}(x)\right| 
        & = \left|\E_{c+1}\left[
            f(X_{c+1}) f(X_{c+1}^{\prime})
            \cdot \frac{\E_{D, D^{\prime}}\left[\kappa\left(x; Z_{c+1}, D\right)\kappa\left(x; Z_{c+1}^{\prime}, D^{\prime}\right) \; \middle| \; Z_{c+1}, Z_{c+1}^{\prime}\right]}{\E_{D, D^{\prime}}\left[\kappa\left(x; Z_{c+1}, D\right)\kappa\left(x; Z_{c+1}^{\prime}, D^{\prime}\right)\right]} 
        \right] - f^{2}(x)\right| \\
        & \leq \E_{c+1}\left[
            \left|f(X_{c+1}) f(X_{c+1}^{\prime}) - f^{2}(x)\right|
            \cdot \frac{\E_{D, D^{\prime}}\left[\kappa\left(x; Z_{c+1}, D\right)\kappa\left(x; Z_{c+1}^{\prime}, D^{\prime}\right) \; \middle| \; Z_{c+1}, Z_{c+1}^{\prime}\right]}{\E_{D, D^{\prime}}\left[\kappa\left(x; Z_{c+1}, D\right)\kappa\left(x; Z_{c+1}^{\prime}, D^{\prime}\right)\right]} 
        \right]
        \end{aligned}
    \end{equation}
    Now, fix an arbitrary $\epsilon > 0$.
    By continuity of $f$ at $x$, there exists a $\delta > 0$, such that the following holds.
    \begin{equation}
        \forall X, X^{\prime} \in B(x, \delta): \quad 
        \left| f(X) \cdot f(X^{\prime}) - f^{2}(x) \right| < \epsilon
    \end{equation}
    We can consider decompositions of these terms in analogy to \citet{demirkaya_optimal_2024}, i.e. by considering cases with observations lying within this sphere or outside of it, and observe the following.
    \begin{equation}
        \begin{aligned}
            & \E_{1}\left[\left|f^{2}(X_{1}) - f^{2}(x) \right| (2s - c) 
                \E_{2:s}\left[\kappa\left(x; Z_{1}, D\right)\kappa\left(x; Z_{1}, D^{\prime}\right) 
                \1\left(X_1 \in B(x, \delta)\right)
                \right]
            \right]\\
            %
            & \quad \leq \epsilon \E_{1}\left[(2s - c) 
                \E_{2:s}\left[\kappa\left(x; Z_{1}, D\right)\kappa\left(x; Z_{1}, D^{\prime}\right) 
                \1\left(X_1 \in B(x, \delta)\right)
                \right]
            \right]\\
            %
            & \quad \leq \epsilon \E_{1}\left[(2s - c) 
                \E_{2:s}\left[\kappa\left(x; Z_{1}, D\right)\kappa\left(x; Z_{1}, D^{\prime}\right)
                \right]
            \right]
            = \epsilon
        \end{aligned}
    \end{equation}
    \begin{equation}
        \begin{aligned}
            & \E_{1, (c+1)^{\prime}}\left[
                \left| f(X_{1}) f(X_{c+1}^{\prime}) - f^{2}(x)\right|
                \cdot \frac{\E_{D, D^{\prime}}\left[\kappa\left(x; Z_{1}, D\right)\kappa\left(x; Z_{c+1}^{\prime}, D^{\prime}\right) \; \middle| \; Z_{1}, Z_{c+1}^{\prime}\right]}{\E_{D, D^{\prime}}\left[\kappa\left(x; Z_{1}, D\right)\kappa\left(x; Z_{c+1}^{\prime}, D^{\prime}\right)\right]}
                \1\left(X_1, X_{c+1}^{\prime} \in B(x, \delta)\right)
            \right]\\
            %
            & \quad \leq \epsilon \E_{1, (c+1)^{\prime}}\left[
                \frac{\E_{D, D^{\prime}}\left[\kappa\left(x; Z_{1}, D\right)\kappa\left(x; Z_{c+1}^{\prime}, D^{\prime}\right) \; \middle| \; Z_{1}, Z_{c+1}^{\prime}\right]}{\E_{D, D^{\prime}}\left[\kappa\left(x; Z_{1}, D\right)\kappa\left(x; Z_{c+1}^{\prime}, D^{\prime}\right)\right]}
                \1\left(X_1, X_{c+1}^{\prime} \in B(x, \delta)\right)
            \right] \\
            %
            & \quad \leq \epsilon \E_{1, (c+1)^{\prime}}\left[
                \frac{\E_{D, D^{\prime}}\left[\kappa\left(x; Z_{1}, D\right)\kappa\left(x; Z_{c+1}^{\prime}, D^{\prime}\right) \; \middle| \; Z_{1}, Z_{c+1}^{\prime}\right]}{\E_{D, D^{\prime}}\left[\kappa\left(x; Z_{1}, D\right)\kappa\left(x; Z_{c+1}^{\prime}, D^{\prime}\right)\right]}
            \right] 
            = \epsilon
        \end{aligned}
    \end{equation}
    \begin{equation}
        \begin{aligned}
            & \E_{c+1}\left[
                \left|f(X_{c+1}) f(X_{c+1}^{\prime}) - f^{2}(x)\right|
                \cdot \frac{\E_{D, D^{\prime}}\left[\kappa\left(x; Z_{c+1}, D\right)\kappa\left(x; Z_{c+1}^{\prime}, D^{\prime}\right) \; \middle| \; Z_{c+1}, Z_{c+1}^{\prime}\right]}{\E_{D, D^{\prime}}\left[\kappa\left(x; Z_{c+1}, D\right)\kappa\left(x; Z_{c+1}^{\prime}, D^{\prime}\right)\right]} 
                \1\left(X_{c+1}, X_{c+1}^{\prime} \in B(x, \delta)\right)
            \right] \\
            %
            & \quad \leq \epsilon \E_{c+1}\left[
                \frac{\E_{D, D^{\prime}}\left[\kappa\left(x; Z_{c+1}, D\right)\kappa\left(x; Z_{c+1}^{\prime}, D^{\prime}\right) \; \middle| \; Z_{c+1}, Z_{c+1}^{\prime}\right]}{\E_{D, D^{\prime}}\left[\kappa\left(x; Z_{c+1}, D\right)\kappa\left(x; Z_{c+1}^{\prime}, D^{\prime}\right)\right]} 
                \1\left(X_{c+1}, X_{c+1}^{\prime} \in B(x, \delta)\right)
            \right]\\
            %
            & \quad \leq \epsilon \E_{c+1}\left[
                \frac{\E_{D, D^{\prime}}\left[\kappa\left(x; Z_{c+1}, D\right)\kappa\left(x; Z_{c+1}^{\prime}, D^{\prime}\right) \; \middle| \; Z_{c+1}, Z_{c+1}^{\prime}\right]}{\E_{D, D^{\prime}}\left[\kappa\left(x; Z_{c+1}, D\right)\kappa\left(x; Z_{c+1}^{\prime}, D^{\prime}\right)\right]}
            \right]
            = \epsilon
        \end{aligned}
    \end{equation}
    Considering next the parts of the expectation that are not covered by the previous cases, we can find the following.
    As in the original proof, we use the fact that if $X$ or $X^{\prime}$ do not lie within $B(x, \delta)$, then the following holds 
    \begin{equation}
        B(x, \delta) \subseteq B(x, \max\left(\|X - x\|, \|X^{\prime} - x\|\right)).
    \end{equation}
    {\color{red} LOREM IPSUM}
\end{proof}

\hrule
