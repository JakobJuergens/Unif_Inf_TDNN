\section{Useful Results}
\hrule

\begin{lem}[\citet{demirkaya_optimal_2024} - Lemma 12]\label{lem:dem12}\mbox{}\\*
	Let $D = \{Z_1, \dotsc, Z_s\}$ an i.i.d.\ sample drawn from $P$.
	The indicator functions $\kappa\left(x; Z_{i}, D\right)$ satisfy the following properties.
	\begin{enumerate}
		\item For any $i \neq j$, we have $\kappa\left(x; Z_{i}, D\right) \kappa\left(x;
			      Z_{j}, D\right)=0$ with probability one;
		\item $\sum_{i=1}^{s} \kappa\left(x; Z_{i}, D\right)=1$;
		\item $\forall i \in [s]: \quad \E_{1:s}\left[\kappa\left(x; Z_{i}, D\right)\right]=s^{-1}$
		\item $\E_{2: s}\left[\kappa\left(x; Z_1, D\right)\right]
			      = \left\{1-\varphi\left(B\left(x,\left\|X_1-x\right\|\right)\right)\right\}^{s-1}$
	\end{enumerate}
	Here $\E_{i: s}$ denotes the expectation with respect to $\left\{Z_{i}, Z_{i+1}, \dotsc, Z_s\right\}$.
	Furthermore, $\varphi$ denotes the probability measure on $\mathbb{R}^{d}$ induced by the random vector $X$.
\end{lem}

\hrule

\begin{lem}[\citet{demirkaya_optimal_2024} - Lemma 13]\label{lem:dem13}\mbox{}\\*
	For any $L^1$ function $f$ that is continuous at $x$, it holds that
	\begin{equation}
		\lim _{s \rightarrow \infty} \E_{1}\left[f\left(X_1\right) s \E_{2:s}\left[\kappa(x; Z_1, D)\right]\right]
		= f(x).
	\end{equation}
\end{lem}

\hrule

\begin{lem}\label{lem:limit_res}\mbox{}\\*
	As a consequence of Lemma~\ref{lem:dem13}, we find the following limit results in the nonparametric regression setup.
	\begin{equation}
		\begin{aligned}
			\lim_{s \rightarrow \infty} \E_{1}\left[Y_1 s \E_{2:s}\left[\kappa(x; Z_1, D)\right]\right]
			 & = \mu\left(x\right)
		\end{aligned}
	\end{equation}
	\begin{equation}
		\begin{aligned}
			\lim_{s \rightarrow \infty} \E_{1}\left[Y_1^2 s \E_{2:s}\left[\kappa(x; Z_1, D)\right]\right]
			 & = \mu^2\left(x\right) + \sigma_{\varepsilon}^{2}(x)
			\leq \mu^2\left(x\right) + \overline{\sigma}_{\varepsilon}^{2}
		\end{aligned}
	\end{equation}
	Similarly, in the CATE estimation setup, we can make the following observations.
	\begin{equation}
		\begin{aligned}
			\lim_{s \rightarrow \infty} \E_{1}\left[m\left(Z_{1}; \mu, \pi\right) s \E_{2:s}\left[\kappa(x; Z_1, D)\right]\right]
			 & = \mu_1\left(x\right) - \mu_0\left(x\right)
		\end{aligned}
	\end{equation}
	\begin{equation}
		\begin{aligned}
			\lim_{s \rightarrow \infty} \E_{1}\left[m^2\left(Z_{1}; \mu, \pi\right) s \E_{2:s}\left[\kappa(x; Z_1, D)\right]\right]
			 & = \left(\mu_1\left(x\right) - \mu_0\left(x\right)\right)^2 + \frac{\sigma_{\varepsilon}^2(x)}{\pi\left(x\right)\left(1 - \pi\left(x\right)\right)}  \\
			%
			 & \leq \left(\mu_1\left(x\right) - \mu_0\left(x\right)\right)^2 + \frac{\overline{\sigma}_{\varepsilon}^2}{\mathfrak{p}\left(1 - \mathfrak{p}\right)}
		\end{aligned}
	\end{equation}
\end{lem}

\hrule

\begin{proof}[Proof of Lemma~\ref{lem:limit_res}]
	Starting with the first limit, we find the following.
	\begin{equation}
		\begin{aligned}
			\E_{1}\left[Y_1 s \E_{2:s}\left[\kappa(x; Z_1, D)\right]\right]
			 & = \E_{1}\left[\left(\mu\left(X_1\right) + \varepsilon_1\right) s \E_{2:s}\left[\kappa(x; Z_1, D)\right]\right]                                   \\
			%
			 & = \E_{1}\left[\left(\mu\left(X_1\right) + \E\left[\varepsilon_1 \, \middle| \, X_1\right]\right) s \E_{2:s}\left[\kappa(x; Z_1, D)\right]\right] \\
			%
			 & = \E_{1}\left[\mu\left(X_1\right) s \E_{2:s}\left[\kappa(x; Z_1, D)\right]\right]
			\overset{\text{(Lem~\ref{lem:dem13})}}{\longrightarrow} \mu\left(x\right)
			\quad \text{as} \quad s \rightarrow \infty
		\end{aligned}
	\end{equation}
	Similarly, when considering the second limit, we can make the following observation.
	\begin{equation}
		\begin{aligned}
			\E_{1}\left[Y_1^2 s \E_{2:s}\left[\kappa(x; Z_1, D)\right]\right]
			 & = \E_{1}\left[\left(\mu\left(X_1\right) + \varepsilon_1\right)^2 s \E_{2:s}\left[\kappa(x; Z_1, D)\right]\right]                                                            \\
			%
			 & = \E_{1}\left[\left(\mu^2\left(X_1\right) + 2\mu\left(X_1\right)\varepsilon_1 + \varepsilon_1^2\right)s \E_{2:s}\left[\kappa(x; Z_1, D)\right]\right]                       \\
			%
			 & = \E_{1}\left[\left(\mu^2\left(X_1\right) + 2\mu\left(X_1\right) \E\left[\varepsilon_1 \, \middle| \, X_1\right] + \E\left[\varepsilon_1^2 \, \middle| \, X_1\right]\right)
			s \E_{2:s}\left[\kappa(x; Z_1, D)\right]\right]                                                                                                                                \\
			%
			 & = \E_{1}\left[\left(\mu^2\left(X_1\right) +\sigma_{\varepsilon}^{2}(X_1)\right) s \E_{2:s}\left[\kappa(x; Z_1, D)\right]\right]                                             \\
			%
			 & \overset{\text{(Lem~\ref{lem:dem13})}}{\longrightarrow} \mu^2\left(x\right) +\sigma_{\varepsilon}^{2}(x)
			\quad \text{as} \quad s \rightarrow \infty
		\end{aligned}
	\end{equation}
	In the CATE estimation setting, we can proceed analogously.
	\begin{equation}
		\begin{aligned}
			\E_{1}\left[m\left(Z_{1}; \mu, \pi\right) s \E_{2:s}\left[\kappa(x; Z_1, D)\right]\right]
			 & = \E_{1}\left[\left(\mu_1\left(X_{1}\right) - \mu_0\left(X_{1}\right) + \beta\left(W_{1}, X_{1}\right)\varepsilon_{i}\right) s \E_{2:s}\left[\kappa(x; Z_1, D)\right]\right]                                \\
			%
			 & = \E_{1}\left[\left(\mu_1\left(X_{1}\right) - \mu_0\left(X_{1}\right) + \beta\left(W_{1}, X_{1}\right)\E\left[\varepsilon_{i} \, \middle| \, \right]\right) s \E_{2:s}\left[\kappa(x; Z_1, D)\right]\right] \\
			%
			 & = \E_{1}\left[\left(\mu_1\left(X_{1}\right) - \mu_0\left(X_{1}\right)\right) s \E_{2:s}\left[\kappa(x; Z_1, D)\right]\right]                                                                                \\
			%
			 & \overset{\text{(Lem~\ref{lem:dem13})}}{\longrightarrow} \mu_1\left(x\right) - \mu_0\left(x\right)
			\quad \text{as} \quad s \rightarrow \infty
		\end{aligned}
	\end{equation}
	Similarly, we can find the following.
	\begin{equation}
		\begin{aligned}
			 & \E_{1}\left[m^2\left(Z_{i}; \mu, \pi\right) s \E_{2:s}\left[\kappa(x; Z_1, D)\right]\right]
			= \E_{1}\left[\left(\mu_1\left(X_{1}\right) - \mu_0\left(X_{1}\right) + \beta\left(W_{1}, X_{1}\right)\varepsilon_{i}\right)^2 s \E_{2:s}\left[\kappa(x; Z_1, D)\right]\right]                                                                                             \\
			%
			 & \quad = \E_{1}\left[\left(\mu_1\left(X_{i}\right) - \mu_0\left(X_{1}\right)\right)^2 s \E_{2:s}\left[\kappa(x; Z_1, D)\right]\right]
			+ \underbrace{\E_{1}\left[\left(\mu_1\left(X_{i}\right) - \mu_0\left(X_{1}\right)\right)\beta\left(W_{1}, X_{1}\right)\E\left[\varepsilon_{1} \, \middle| \, X_1\right] s \E_{2:s}\left[\kappa(x; Z_1, D)\right]\right]}_{=0}                                              \\
			 & \quad \quad + \E_{1}\left[\left(\beta\left(W_{1}, X_{1}\right)\varepsilon_{i}\right)^2 s \E_{2:s}\left[\kappa(x; Z_1, D)\right]\right]                                                                                                                                  \\
			%
			 & \quad =  \E_{1}\left[\left(\mu_1\left(X_{1}\right) - \mu_0\left(X_{1}\right)\right)^2 s \E_{2:s}\left[\kappa(x; Z_1, D)\right]\right]
			+ \E_{1}\left[\left(\frac{W_{1}}{\pi\left(X_1\right)} - \frac{1 - W_{1}}{1 - \pi\left(X_1\right)}\right)^2 \varepsilon_{1}^2 s \E_{2:s}\left[\kappa(x; Z_1, D)\right]\right]                                                                                               \\
			%
			 & \quad = \underbrace{\E_{1}\left[\left(\mu_1\left(X_{1}\right) - \mu_0\left(X_{1}\right)\right)^2 s \E_{2:s}\left[\kappa(x; Z_1, D)\right]\right]}_{\longrightarrow \left(\mu_1\left(x\right) - \mu_0\left(x\right)\right)^2 \quad \text{as} \quad s \rightarrow \infty}
			+ \underbrace{\E_{1}\left[\E\left[\left(\frac{W_{1}}{\pi\left(X_1\right)} - \frac{1 - W_{1}}{1 - \pi\left(X_1\right)}\right)^2 \varepsilon_{1}^2 \, \middle| \, X_1\right] s \E_{2:s}\left[\kappa(x; Z_1, D)\right]\right]}_{(B)}
		\end{aligned}
	\end{equation}
	Continuing with the second term, marked by $(B)$, we find the following.
	\begin{equation}
		\begin{aligned}
			(B)
			 & = \E_{1}\left[\E\left[\left(\frac{W_{1}}{\pi\left(X_1\right)} - \frac{1 - W_{1}}{1 - \pi\left(X_1\right)}\right)^2 \varepsilon_{1}^2 \, \middle| \, X_1\right] s \E_{2:s}\left[\kappa(x; Z_1, D)\right]\right] \\
			%
			 & = \E_{1}\left[\frac{\sigma_{\varepsilon}^2(X_1) \cdot s \E_{2:s}\left[\kappa(x; Z_1, D)\right]}{\pi^2\left(X_1\right)\left(1 - \pi\left(X_1\right)\right)^2} \cdot
			\E\left[\left(W_{1}\left(1 - \pi\left(X_1\right)\right) - \left(1 - W_{1}\right)\pi\left(X_1\right)\right)^2 \, \middle| \, X_1\right] \right]                                                                    \\
		\end{aligned}
	\end{equation}
	Observe that $W_1(1-W_1) = 0$, $W_1^2 = W_1$, and $(1-W_1)^2 = 1 - W_1$, which allows us to use the following simplification.
	\begin{equation}
		\begin{aligned}
			(B)
			 & = \E_{1}\left[\frac{\sigma_{\varepsilon}^2(X_1) \cdot s \E_{2:s}\left[\kappa(x; Z_1, D)\right]}{\pi^2\left(X_1\right)\left(1 - \pi\left(X_1\right)\right)^2} \cdot
			\E\left[W_{1}\left(1 - \pi\left(X_1\right)\right)^2 + \left(1 - W_{1}\right)\pi^2\left(X_1\right) \, \middle| \, X_1\right] \right]                                   \\
			%
			 & = \E_{1}\left[\frac{\sigma_{\varepsilon}^2(X_1) \cdot s \E_{2:s}\left[\kappa(x; Z_1, D)\right]}{\pi^2\left(X_1\right)\left(1 - \pi\left(X_1\right)\right)^2} \cdot
			\left(\pi(X_1)\left(1 - \pi\left(X_1\right)\right)^2 + (1 - \pi(X_1))\pi^2\left(X_1\right) \right)\right]                                                             \\
			%
			 & = \E_{1}\left[\frac{\sigma_{\varepsilon}^2(X_1) \cdot s \E_{2:s}\left[\kappa(x; Z_1, D)\right]}{\pi\left(X_1\right)\left(1 - \pi\left(X_1\right)\right)}\right]
			\overset{\text{(Lem~\ref{lem:dem13})}}{\longrightarrow} \frac{\sigma_{\varepsilon}^2(x)}{\pi\left(x\right)\left(1 - \pi\left(x\right)\right)}
			\quad \text{as} \quad s \rightarrow \infty
		\end{aligned}
	\end{equation}
	Recombining the terms of interest, we find the desired limit bound.
	\begin{equation}
		\begin{aligned}
			\E_{1}\left[m^2\left(Z_{i}; \mu, \pi\right) s \E_{2:s}\left[\kappa(x; Z_1, D)\right]\right]
			\overset{\text{(Lem~\ref{lem:dem13})}}{\longrightarrow} \left(\mu_1\left(x\right) - \mu_0\left(x\right)\right)^2 + \frac{\sigma_{\varepsilon}^2(x)}{\pi\left(x\right)\left(1 - \pi\left(x\right)\right)}
			\quad \text{as} \quad s \rightarrow \infty
		\end{aligned}
	\end{equation}
\end{proof}

\hrule

\begin{lem}\label{lem:npr_kern_ineq1}\mbox{}\\*
	Fix sample size $n$, subsampling scale $s$, and $c$ such that $0 < c \leq s \leq n$.
	Let $D = \left\{Z_1, Z_2, \dotsc, Z_c, Z_{c+1}, \dotsc Z_s \right\}$ be an i.i.d.\ data set drawn from $P$ as described in Setup~\ref{asm:npr_dgp}.
	Let $D^{\prime} = \left\{Z_1, Z_2, \dotsc, Z_c, Z_{c+1}^{\prime}, \dotsc Z_s^{\prime} \right\}$ be a second data set that shares the first $c$ observations with $D$.
	The remaining $s - c$ observations of $D^{\prime}$, i.e.\ $\left\{Z_{c+1}^{\prime}, \dotsc Z_s^{\prime} \right\}$, are i.i.d.\ draws from $P$ that are independent of $D$.

	Then, the following inequalities holds for sufficiently large $s$
	\begin{equation}
		\begin{aligned}
			\E_{D, D^{\prime}}\left[Y_{1}Y_{c+1}^{\prime} \, c(s-c) \, \kappa\left(x; Z_{1}, D\right)\kappa\left(x; Z_{c+1}^{\prime}, D^{\prime}\right)\right]
			& \lesssim  \frac{c(s-c)}{s^2}\mu^2(x) + o(1) 	
		\end{aligned}
	\end{equation}
\end{lem}

\hrule

\begin{proof}[Proof of Lemma~\ref{lem:npr_kern_ineq1}]\mbox{}\\*
	Consider first the following argument.
	\begin{equation}
		\begin{aligned}
			 & \E_{D, D^{\prime}}\left[Y_{1}Y_{c+1}^{\prime} \, c(s-c) \, \kappa\left(x; Z_{1}, D\right)\kappa\left(x; Z_{c+1}^{\prime}, D^{\prime}\right)\right]          \\
			 & \quad = \E_{1, c+1}\left[\mu(X_1) \mu(X_{c+1}^{\prime}) \, c(s-c) \,
			\E\left[\kappa\left(x; Z_{1}, D\right)\kappa\left(x; Z_{c+1}^{\prime}, D^{\prime}\right) \; \middle| \; X_1, X_{c+1}^{\prime}\right]\right]                    \\
			%
			 & \leq \frac{c(s-c)}{s^2} \cdot \E_{1, c+1}\left[\left|\mu(X_1)\right| \left|\mu(X_{c+1}^{\prime})\right| \, s^2 \,
			\underbrace{\E\left[\kappa\left(x; Z_{1}, D\right)\kappa\left(x; Z_{c+1}^{\prime}, D^{\prime}\right) \; \middle| \; X_1, X_{c+1}^{\prime}\right]}_{(A)}\right] \\
		\end{aligned}
	\end{equation}
	Analyzing term $(A)$ individually, we find the following.
	\begin{equation}
		\begin{aligned}
			(A)
			 & = \E\left[\kappa\left(x; Z_{1}, D\right)\kappa\left(x; Z_{c+1}^{\prime}, D^{\prime}\right) \; \middle| \; X_1, X_{c+1}^{\prime}\right]                                                                                       \\
			%
			 & = \E\left[\left(\prod_{i = 2}^{s}\1\left(\|X_1 - x\| < \|X_{i} - x\|\right)\right)
				\left(\prod_{i = 1}^{c}\1\left(\|X_i - x\| > \|X_{c+1}^{\prime} - x\|\right)\right)
			\left(\prod_{i = c+2}^{s}\1\left(\|X_i^{\prime} - x\| > \|X_{c+1}^{\prime} - x\|\right)\right) \; \middle| \; X_1, X_{c+1}^{\prime}\right]                                                                                      \\
			%
			 & = \1\left(\|X_1 - x\| > \|X_{c+1}^{\prime} - x\|\right) \cdot \E\left[\prod_{i = 2}^{c}\1\left(\|X_{i} - x\| > \max\left\{\|X_{1} - x\|, \|X_{c+1}^{\prime} - x\|\right\}\right) \; \middle| \; X_1, X_{c+1}^{\prime}\right] \\
			 & \quad \quad  \cdot \E\left[\prod_{i = c+1}^{s}\1\left(\|X_i - x\| > \|X_{1} - x\|\right) \; \middle| \; X_{1}\right]
			\cdot \E\left[\prod_{i = c+2}^{s}\1\left(\|X_i^{\prime} - x\| > \|X_{c+1}^{\prime} - x\|\right) \; \middle| \; X_{c+1}^{\prime}\right]                                                                                          \\
			%
			 & \leq \E\left[\kappa\left(x; Z_{1}, D\right) \; \middle| \; X_1\right]
			\cdot \E\left[\kappa\left(x; Z_{c+1}^{\prime}, D^{\prime}\right) \; \middle| \; X_{c+1}^{\prime}\right]
		\end{aligned}
	\end{equation}
	Plugging back into the expression of interest, we find the desired result.
	\begin{equation}
		\begin{aligned}
			 & \E_{D, D^{\prime}}\left[Y_{1}Y_{c+1}^{\prime} \, c(s-c) \, \kappa\left(x; Z_{1}, D\right)\kappa\left(x; Z_{c+1}^{\prime}, D^{\prime}\right)\right]                \\
			%
			 & \quad \leq \frac{c(s-c)}{s^2} \cdot \E_{1, c+1}\left[\left|\mu(X_1)\right| \left|\mu(X_{c+1}^{\prime})\right| \, s^2 \,
				\E\left[\kappa\left(x; Z_{1}, D\right) \; \middle| \; X_1\right]
			\cdot \E\left[\kappa\left(x; Z_{c+1}^{\prime}, D^{\prime}\right) \; \middle| \; X_{c+1}^{\prime}\right]\right]                                                       \\
			%
			 & \quad = \frac{c(s-c)}{s^2} \cdot \E_{1}\left[\left|\mu(X_1)\right| \, s \, \E\left[\kappa\left(x; Z_{1}, D\right) \; \middle| \; X_1\right]\right]
			\E_{c+1}\left[\left|\mu(X_{c+1})\right| \, s \, \E\left[\kappa\left(x; Z_{c+1}^{\prime}, D\right) \; \middle| \; X_{c+1}^{\prime}\right]\right]                      \\
			%
			 & \quad = \frac{c(s-c)}{s^2} \cdot \left(\E_{1}\left[\left|\mu(X_1)\right| \, s \, \E\left[\kappa\left(x; Z_{1}, D\right) \; \middle| \; X_1\right]\right]\right)^2
			\overset{\text{(Lem~\ref{lem:limit_res})}}{\lesssim}  \frac{c(s-c)}{s^2}\mu^2(x) + o(1)
			\quad \text{as} \quad s \rightarrow \infty.
		\end{aligned}
	\end{equation}
\end{proof}

\hrule

\begin{lem}\label{lem:npr_kern_ineq2}\mbox{}\\*
	Fix sample size $n$, subsampling scale $s$, and $c$ such that $0 < c \leq s \leq n$.
	Let $D = \left\{Z_1, Z_2, \dotsc, Z_c, Z_{c+1}, \dotsc Z_s \right\}$ be an i.i.d.\ data set drawn from $P$ as described in Setup~\ref{asm:npr_dgp}.
	Let $D^{\prime} = \left\{Z_1, Z_2, \dotsc, Z_c, Z_{c+1}^{\prime}, \dotsc Z_s^{\prime} \right\}$ be a second data set that shares the first $c$ observations with $D$.
	The remaining $s - c$ observations of $D^{\prime}$, i.e.\ $\left\{Z_{c+1}^{\prime}, \dotsc Z_s^{\prime} \right\}$, are i.i.d.\ draws from $P$ that are independent of $D$.

	Then, the following inequalities holds for sufficiently large $s$
	\begin{equation}
		\begin{aligned}
			\E_{D, D^{\prime}}\left[Y^{2}_{1} \, c \, \kappa\left(x; Z_{1}, D\right)\kappa\left(x; Z_{1}^{\prime}, D^{\prime}\right)\right]
		 & \lesssim  \frac{c}{2s - c} \left(\mu^2(x) + \sigma^2_{\varepsilon}(x)\right) + o(1)         
		 \leq \frac{c}{2s - c} \left(\mu^2(x) + \overline{\sigma}^2_{\varepsilon}\right) + o(1)          \\
		\end{aligned}
	\end{equation}
\end{lem}

\hrule
\begin{proof}[Proof of Lemma~\ref{lem:npr_kern_ineq2}]\mbox{}\\*
	We can make the following observation.
	\begin{equation}
		\begin{aligned}
			 \E_{D, D^{\prime}}\left[Y^2_{1} \, c \, \kappa\left(x; Z_{1}, D\right)\kappa\left(x; Z_{1}^{\prime}, D^{\prime}\right)\right] 
			 & = \E_{1}\left[\E\left[\left(\mu(X_1) + \varepsilon_1\right)^2 \; \middle| \; X_1\right] \, c^2 \, \E\left[\kappa\left(x; Z_{1}, D\right)\kappa\left(x; Z_{1}, D^{\prime}\right) \; \middle| \; X_1\right]\right]\\
			 %
			 & = \frac{c}{2s - c} \cdot \E_{1}\left[\left(\mu^2(X_1) + \sigma^2_{\varepsilon}(X_1)\right) \, \left(2s - c\right) \, 
			 \E\left[\kappa\left(x; Z_{1}, D\right)\kappa\left(x; Z_{1}, D^{\prime}\right) \; \middle| \; X_1\right]\right]\\
			 %
			 & \overset{\text{(Lem~\ref{lem:limit_res})}}{\lesssim} \frac{c}{2s - c} \left(\mu^2(x) + \sigma^2_{\varepsilon}(x)\right) + o(1)
		\end{aligned}
	\end{equation}
\end{proof}

\hrule

\begin{lem}\label{lem:npr_kern_ineq3}\mbox{}\\*
	Fix sample size $n$, subsampling scale $s$, and $c$ such that $0 < c \leq s \leq n$.
	Let $D = \left\{Z_1, Z_2, \dotsc, Z_c, Z_{c+1}, \dotsc Z_s \right\}$ be an i.i.d.\ data set drawn from $P$ as described in Setup~\ref{asm:npr_dgp}.
	Let $D^{\prime} = \left\{Z_1, Z_2, \dotsc, Z_c, Z_{c+1}^{\prime}, \dotsc Z_s^{\prime} \right\}$ be a second data set that shares the first $c$ observations with $D$.
	The remaining $s - c$ observations of $D^{\prime}$, i.e.\ $\left\{Z_{c+1}^{\prime}, \dotsc Z_s^{\prime} \right\}$, are i.i.d.\ draws from $P$ that are independent of $D$.

	Then, the following inequalities holds for sufficiently large $s$
	\begin{equation}
		\begin{aligned}
			\E_{D, D^{\prime}}\left[Y_{c+1}Y_{c+1}^{\prime} \, (s-c)^2 \, \kappa\left(x; Z_{c+1}, D\right)\kappa\left(x; Z_{c+1}^{\prime}, D^{\prime}\right)\right]
		 & \lesssim  {\color{red} LOREM IPSUM}	
		\end{aligned}
	\end{equation}
\end{lem}

\hrule

\begin{proof}[Proof of Lemma~\ref{lem:npr_kern_ineq3}]\mbox{}\\*
	We can make a similar argument as before.
	\begin{equation}
		\begin{aligned}
			& \E_{D, D^{\prime}}\left[Y_{c+1}Y_{c+1}^{\prime} \, (s-c)^2 \, \kappa\left(x; Z_{c+1}, D\right)\kappa\left(x; Z_{c+1}^{\prime}, D^{\prime}\right)\right]\\
			%
			& \quad = \E_{D, D^{\prime}}\left[\E\left[\left(\mu(X_{c+1}) + \varepsilon_{c+1}\right) \cdot \left(\mu(X_{c+1}^{\prime}) + \varepsilon_{c+1}^{\prime}\right) \, \middle| \, X_{c+1}, X_{c+1}^{\prime}\right] \, (s-c)^2 \, \kappa\left(x; Z_{c+1}, D\right)\kappa\left(x; Z_{c+1}^{\prime}, D^{\prime}\right)\right] \\
			% 
			& \quad = \E_{D, D^{\prime}}\left[\mu(X_{c+1})\mu(X_{c+1}^{\prime}) \, (s-c)^2 \, 
			\E\left[\kappa\left(x; Z_{c+1}, D\right)\kappa\left(x; Z_{c+1}^{\prime}, D^{\prime}\right) \; \middle| \; X_{c+1}, X_{c+1}^{\prime}\right]\right] \\
			%
			& = {\color{red} LOREM IPSUM}
		\end{aligned}
	\end{equation}
\end{proof}

\hrule

\begin{lem}\label{lem:CATE_kern_ineqs}\mbox{}\\*
	Fix sample size $n$, subsampling scale $s$, and $c$ such that $0 < c \leq s \leq n$.
	Let $D = \left\{Z_1, Z_2, \dotsc, Z_c, Z_{c+1}, \dotsc Z_s \right\}$ be an i.i.d.\ data set drawn from $Q$ as described in Setup~\ref{asm:CATE_dgp}.
	Let $D^{\prime} = \left\{Z_1, Z_2, \dotsc, Z_c, Z_{c+1}^{\prime}, \dotsc Z_s^{\prime} \right\}$ be a second data set that shares the first $c$ observations with $D$.
	The remaining $s - c$ observations of $D^{\prime}$, i.e.\ $\left\{Z_{c+1}^{\prime}, \dotsc Z_s^{\prime} \right\}$, are i.i.d.\ draws from $Q$ that are independent of $D$.

	Then, the following three inequalities hold.

		
	Similarly, consider the CATE estimation setting (Setup~\ref{asm:CATE_dgp}),
	i.e.\ replacing observations drawn from $P$ by observations drawn from $Q$.
	Then, analogous inequalities hold, where we replace\dots
	\begin{multicols}{2}
		\begin{itemize}
			\item $Y_i$ by $m(Z_{i}, \mu, \pi)$
			\item $\mu^2(x)$ by $\left(\mu_1\left(x\right) - \mu_{0}\left(x\right)\right)^2$
			\item $\sigma^{2}_{\varepsilon}(x)$ by $\frac{\sigma_{\varepsilon}^2\left(x\right)}{\pi(x)\left(1 - \pi(x)\right)}$
			\item $\overline{\sigma}^{2}_{\varepsilon}$ by $\frac{\overline{\sigma}^2_{\varepsilon}}{\mathfrak{p}\left(1 - \mathfrak{p}\right)}$
		\end{itemize}
	\end{multicols}
\end{lem}

\hrule

\begin{proof}[Proof of Lemma~\ref{lem:CATE_kern_ineqs}]\mbox{}\\*
	The inequalities follow analogous to the proofs of Lemma~\ref{lem:npr_kern_ineq1}, Lemma~\ref{lem:npr_kern_ineq2}, and Lemma~\ref{lem:npr_kern_ineq3}.
\end{proof}

\hrule

\begin{lem}[\citet{peng_bias_2021} - Lemma 1]\label{lem:peng1}\mbox{}\\*
	Suppose that $\sum X_{i}^2 \xrightarrow{p} 1, \sum \E\left[X_{i}^2\right] \rightarrow 1$, and $\sum_{i=1}^n \E\left[Y_{i}^2\right] \rightarrow 0$, then
	\begin{equation}
		\sum\left[X_{i}+Y_{i}\right]^2 \xrightarrow{p} 1 \quad \text { and } \E\left[\sum\left(X_{i}+Y_{i}\right)^2\right] \rightarrow 1.
	\end{equation}
\end{lem}

\hrule

\begin{lem}[Honesty of the DNN/TDNN Estimators]\label{lem:honesty}\mbox{}\\*
	The DNN and TDNN estimator kernels $\kappa\left(\cdot, \cdot, D_{\ell}\right)$ are Honest in the sense of \citet{wager_estimation_2018}.
	\begin{equation*}
		\kappa\left(x, X_{i}, D_{\ell}\right) \indep Y_{i} \mid X_{i}, D_{\ell,-i},
	\end{equation*}
	where $\indep$ denotes conditional independence and $D_{\ell,-i} = \{Z_l \, | \, l \in \ell \backslash \{i\}\}$.
\end{lem}