\section{Pointwise Inference for the TDNN Estimator}\label{sec:pw_inf}
\hrule
To perform inference in the regression setup, \citet{demirkaya_optimal_2024} introduce variance estimators based on the Jackknife and Bootstrap.
However, as they point out, their consistency results rely on a likely suboptimal rate condition for the subsampling scale.
While Theorem~\ref{thm:dem3} allows $s_2$ to be of the order $o(n)$, the variance estimators rely on the considerably stronger condition that $s_2 = o(n^{1/3})$.
Establishing consistent variance estimation under weaker assumptions on the subsampling rates could broaden the scope of the TDNN estimator for inferential purposes considerably.
Furthermore, it can contribute to a better balance between variance and bias as the choice of the kernel orders is crucial when considering the finite sample properties of the estimator.
In this paper, we will focus specifically on variance estimators based on the Jackknife and show consistency results under $s = o(n)$.
This is motivated by the closed form representation of the estimators in question leading to computationally simple formulas for the exact Jackknife variance estimators.

\subsection{Jackknife Variance Estimators for Nonparametric Regression}\label{Var_Ests}
\hrule

Define the following variance we need to estimate to perform pointwise inference at a point of interest $x$.
\begin{equation}\label{eq:TDNN_Var}
	\omega^{2}\left(x\right)
	= \Var_{D}\left(\hat{\mu}_{s_1, s_2}\left(x; \mathbf{D}_n\right)\right)
\end{equation}
We denote by $\mathbf{D}_{n, -i}$ the data set $\mathbf{D}_n$ after removing the $i$'th observation.
Then, the proposed Jackknife variance estimator takes the following form.
\begin{equation}\label{eq:JK_Var_Est}
	\hat{\omega}_{JK}^2\left(x; \mathbf{D}_n\right)
	= \frac{n-1}{n} \sum_{i = 1}^{n}\left(\hat{\mu}_{s_1, s_2}\left(x; \mathbf{D}_{n, -i}\right) - \hat{\mu}_{s_1, s_2}\left(x; \mathbf{D}_{n}\right)\right)^2
\end{equation}

\begin{boxD}
	\begin{thm}[Closed Form Expression for the Jackknife-Variance Estimator]\label{thm:JK_closed_form}\mbox{}\\*
		The Jackknife variance estimator for the DNN estimator has the following convenient closed-form representations.
		\begin{equation}
			{\color{red} LOREM IPSUM}
		\end{equation}
		Similarly, the Jackknife variance estimator for the TDNN estimator admits the following representation.
		\begin{equation}
			{\color{red} LOREM IPSUM}
		\end{equation}
	\end{thm}
\end{boxD}

As a generalization to the Jackknife, we can also consider the delete-d Jackknife that builds on the same working principle.
Instead of removing one observation at a time, we remove $d$ observations and average over all possible d-subsets removals.
This leads to the following representation.
\begin{equation}\label{eq:JKD_Var_Est}
	\hat{\omega}_{JKD}^2\left(x; d, \mathbf{D}_n\right)
	= \frac{n-d}{d}\binom{n}{d}^{-1} \sum_{\ell \in L_{n,d}}
	\left(\hat{\mu}_{s_1, s_2}\left(x; \mathbf{D}_{n, -\ell}\right)
	- \hat{\mu}_{s_1, s_2}\left(x; \mathbf{D}_{n}\right)
	\right)^2
\end{equation}
Similar to the Jackknife, it is possible to derive a closed form representation for the delete-d Jackknife.
The derivation would proceed along the exact same lines as in the Jackknife case.
However, due to the unwieldiness of the closed form, we refrain from deriving it.

In this section, we will loosen that restrictive condition to make use of the attractive performance of U-statistics with large subsampling rates in the context of inference.
The PIJK variance estimator applied to the TDNN estimator is as follows.
\begin{equation}\label{eq:PIJK_Var_Est}
	\hat{\omega}_{PI}^2\left(x; \mathbf{D}_n\right)
	= \frac{s_2^2}{n^2}\sum_{i = 1}^{n}\left[\left(
		\binom{n-1}{s-1}^{-1} \sum_{\ell \in L_{s_2-1}([n]\backslash\{i\})} h_{s_1, s_2}\left(x; D_{\ell \cup \{i\}}\right)\right)
		- \hat{\mu}_{s_1, s_2}\left(x; \mathbf{D}_n\right)\right]^2
\end{equation}

{\color{red} LOREM IPSUM}

Analyzing the kernel of the TDNN estimators, it can be shown that the conditions of Theorem 6 of \citet{peng_bias_2021} apply under the regime $s_2 = o(n)$.
Thus, we obtain the following result.
% \begin{boxD}
% 	\begin{lem}[\citet{peng_bias_2021} - Theorem 6]\label{thm:peng21_6}\mbox{}\\*
% 		Let $X_1, \ldots, X_n$ be i.i.d.\ from $\mathrm{P}$ and $\mathrm{U}_{n, k, \omega}$ be a generalized complete $U$ statistic with kernel $s\left(X_1, \ldots, X_k ; \omega\right)$.
	
% 		Let $\theta=\E[s], \zeta_{1, \omega}=\Var\left(\E\left[s \mid X_1\right]\right)$ and $\zeta_k=\Var(s)$.
% 		If $\frac{k}{n}\left(\frac{\zeta_k}{k \zeta 1, \omega}-1\right) \rightarrow 0$, then
% 		\begin{equation}
% 			\frac{\mathrm{ps-IJ}_{U}^\omega}{\Var\left(\mathrm{U}_{n, k, \omega}\right)}
% 			\longrightarrow_{p} 1 .
% 		\end{equation}
% 	\end{lem}
% \end{boxD}
\begin{boxD}
	\begin{thm}[Pseudo-Infinitesimal Jackknife Variance Estimator Consistency]\label{thm:PI_JK_Cons}\mbox{}\\*
		Let $0 < \mathfrak{c} \leq s_1/s_2 \leq 1 - \mathfrak{c} < 1$ and $s_2 = o(n)$, then
		\begin{equation}
			\frac{\hat{\omega}_{PI}^2\left(x; \mathbf{D}_n\right)}{\omega^{2}\left(x; \mathbf{D}_n\right)} \longrightarrow_{p} 1.
		\end{equation}
	\end{thm}
\end{boxD}


In an analogous fashion to Theorems 5 and 6 from \citet{demirkaya_optimal_2024}, we furthermore obtain the following consistency results for the presented variance estimators.
As they point out, proving these results goes beyond the techniques presented in \citet{arvesen_jackknifing_1969}, instead relying on results for infinite-order U-statistics.
Following the ideas from \citet{peng_bias_2021}, we then obtain the following results on the Jackknife and Bootstrap variance estimators respectively.
As part of the proof of these results, we obtain general results on the consistency of Jackknife and Bootstrap variance estimators for infinite-order U-statistics beyond the TDNN estimator.
\begin{boxD}
	\begin{thm}[Jackknife Variance Estimator Consistency]\label{thm:JK_Cons}\mbox{}\\*
		Let $0 < \mathfrak{c} \leq s_1/s_2 \leq 1 - \mathfrak{c} < 1$ and $s_2 = o(n)$, then
		\begin{equation}
			\frac{\hat{\omega}_{JK}^2\left(x; \mathbf{D}_n\right)}{\omega^{2}\left(x; \mathbf{D}_n\right)} \longrightarrow_{p} 1.
		\end{equation}
	\end{thm}
\end{boxD}

\begin{boxD}
	\begin{thm}[delete-d Jackknife Variance Estimator Consistency]\label{thm:JKD_Cons}\mbox{}\\*
		Let $0 < \mathfrak{c} \leq s_1/s_2 \leq 1 - \mathfrak{c} < 1$, $s_2 = o(n)$, and $d = o(n)$, then
		\begin{equation}
			\frac{\hat{\omega}_{JKD}^2\left(x; d, \mathbf{D}_n\right)}{\omega^{2}\left(x; \mathbf{D}_n\right)} \longrightarrow_{p} 1.
		\end{equation}
	\end{thm}
\end{boxD}

\subsection{Variance Estimation for the (T)DNN-DML2 CATE Estimator}\label{CATE_Var_Ests}
\hrule

{\color{red}
Ideas:
\begin{itemize}
	\item Ignoring the occurrence of left-out observation in nuisance parameter estimation and do basic Jackknife - does this lead to bias?
	\item Leave Fold-Out Bootstrap with slowly diverging number of folds ($k \rightarrow \infty, \;m = o(n)$) - Effectively a variant of delete-d bootstrap
	\item Leave out two folds in the estimator's first step. 
	Then use each previously left out fold for Jackknife construction to eliminate contamination from nuisance parameters
	\item Modify approach presented in \citet{ritzwoller_uniform_2024} Appendix F.4 - modified half-sample k-fold cross-split bootstrap root
\end{itemize}
}

A fitting variance estimator given the context of this paper in the literature can be obtained by modifying a construction presented in \citet{ritzwoller_uniform_2024}.
Specifically, the procedure is based on a variation of the approach presented in Appendix F.4 of the aforementioned paper and makes use of a carefully constructed bootstrap-root.
Thus, we need to introduce some additional notation, where, for simplicity, we assume that $m$, i.e.\ the number of observations in each $I_k$, is even.
\begin{boxD}
	\begin{dfn}[Crossfitting Half-Sample]\label{def:CF_HSample}\mbox{}\\*
		Given a K-fold partition $\mathcal{I} = \left(I_k\right)_{k = 1}^{K}$ of $[n]$, a corresponding half sample of $\mathcal{I}$ is a collection of subsets $\mathcal{H} = \left(H_{k}\right)_{k = 1}^{K}$ such that for all $k \in [K]$, the following holds.
		\begin{equation}
			|H_k| = \frac{|I_k|}{2}= m/2
			\quad \text{and} \quad
			H_k \subset I_k
		\end{equation}
		The set of all such half-samples of $\mathcal{I}$ is denoted by $\mathfrak{H}\left(\mathcal{I}\right)$.
	\end{dfn}
\end{boxD}
This bootstrap root will take the following structure.
\begin{equation}
	R^{*}_{n,s}\left(x; \mathbf{D}_{[n]}, \mathcal{I}\right)
	= \overline{\operatorname{CATE}}_{\mathcal{H}}\left(x\right)
	- \widehat{\operatorname{CATE}}\left(x\right)
\end{equation}
Here, $\overline{\operatorname{CATE}}_{\mathcal{H}}\left(x\right)$ is the solution to the following equation, where $\mathcal{I}$ is a fixed partition of $[n]$ and $\mathcal{H}$ is a fixed half-sample corresponding to $\mathcal{I}$.
In analogy to the previously established notation, we let $K(x, X_{i} \, | \, \mathcal{H})$ denote the kernel as previously established but with respect to the chosen half-sample and $Z_{(i \, | \, \mathcal{H})}$ denote the $i$'th closest observation to the point of interest $\mathbf{x}$ that is contained in $\mathcal{H}$.
Furthermore, $k(i \, | \, \mathcal{H})$ denotes the fold $k \in [K]$ that the $i$'th closest observation in $\mathcal{H}$ is contained in.
\begin{equation}
	\begin{aligned}
		0 & = \sum_{k = 1}^{K} \sum_{i \in H_k} K(x, X_{i} \, | \, \mathcal{H}) m\left(Z_{i}; \overline{CATE}_{\mathcal{H}}\left(x\right), \hat{\mu}_{k,s}, \hat{\pi}_{k,s}\right)\\
		%
	 	& = \sum_{i = 1}^{n/2 - s + 1} \left[\frac{\binom{n/2-i}{s-1}}{\binom{n/2}{s}} m\left(Z_{(i \, | \, \mathcal{H})}; \overline{\operatorname{CATE}}_{\mathcal{H}}\left(x\right), \hat{\mu}_{k(i \, | \, \mathcal{H}),s}, \hat{\pi}_{k(i\, | \, \mathcal{H}),s}\right)\right]
	\end{aligned}
\end{equation}
Plugging in for the moment under consideration once more, we find the following.
\begin{equation}
	\begin{aligned}
		\overline{\operatorname{CATE}}_{\mathcal{H}}\left(x\right) 
		& = \sum_{i = 1}^{n/2 - s + 1} \frac{\binom{n/2-i}{s-1}}{\binom{n/2}{s}} 
		\left[\hat{\mu}_{k(i \, | \, \mathcal{H}),s}^{1}\left(X_{(i \, | \, \mathcal{H})}\right) 
		- \hat{\mu}_{k(i \, | \, \mathcal{H}),s}^{0}\left(X_{(i \, | \, \mathcal{H})}\right) \right.\\
		& \left. \quad + \hat{\beta}_{k(i \, | \, \mathcal{H}),s}\left(W_{(i \, | \, \mathcal{H})}, X_{(i \, | \, \mathcal{H})}\right)\left(Y_{(i \, | \, \mathcal{H})} - \mu_{W_{(i \, | \, \mathcal{H})}}\left(X_{(i \, | \, \mathcal{H})}\right)\right)\right]
	\end{aligned}
\end{equation}
Recognizing the similarity to $\widehat{\operatorname{CATE}}\left(x\right)$, we can further simplify in the following way.
\begin{equation}
	\begin{aligned}
		R^{*}_{n,s}\left(x; \mathbf{D}_{[n]}, \mathcal{I}\right)
		& = {\color{red} LOREM IPSUM}
	\end{aligned}
\end{equation}

\begin{boxD}
	\begin{thm}[Consistent Variance Estimation for the (T)DNN-DML2 CATE Estimator]\label{thm:CATE_DNN_DML_Var_Est}\mbox{}\\*
		{\color{red} LOREM IPSUM}
	\end{thm}
\end{boxD}

\subsection{Pointwise Inference with the TDNN Estimator}
\hrule
\begin{boxD}
	\begin{thm}[Pointwise Inference in Nonparametric Regression]\label{thm:pw_inf_TDNN}\mbox{}\\*
		{\color{red} LOREM IPSUM}
	\end{thm}
\end{boxD}

\begin{boxD}
	\begin{thm}[Pointwise Inference in Heterogeneous Treatment Effect Estimation]\label{thm:pw_inf_TDNN_HTE}\mbox{}\\*
		{\color{red} LOREM IPSUM}
	\end{thm}
\end{boxD}
